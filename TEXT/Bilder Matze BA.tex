\documentclass[pagesize,paper=A4,fontsize=12pt,utf8,numbers=noenddot,bibliography=totoc,listof=totoc,DIV=11,BCOR=1mm]{scrreprt}
% BCOR ist die Bindekorrektur (verlorener Rand am linken Blattrand)! Wert haengt von der Art der Heftung ab!!
% DIV ist eine Satzspiegeleinstellung von KOMA-Script / sccreprt.

\pagestyle{headings}

\usepackage[T1]{fontenc}
\usepackage[utf8]{inputenc} % Font Encoding fuer europaeische Schriften mit Umlauten (Unterstuetzung der Worttrennung)
\usepackage{lmodern} % PostScript-Varianten der TeX Computer Modern-Schriften laden
\usepackage[english,ngerman]{babel} % Spracheinstellungen fuer Englisch und Neudeutsch laden

\usepackage{graphicx} % Grafikeinbindung (fuer .JPG, .JPEG, .PNG und .PDF, falls pdflatex benutzt wird)
\usepackage{float}

\usepackage{booktabs} % fuer schoene Tabellen

\usepackage{amsmath}


% Fonteinstellungen fuer Bildunterschriften: Unterschrift serifenlos, "Abbildung" fett (bfseries = bold face series)
\setkomafont{captionlabel}{\sffamily\bfseries}
\setkomafont{caption}{\sffamily}




\begin{document}

\chapter{Anhang}

\section{Vormessung}


\subsection{Arri Sun 5}

\begin{figure}[htp]     % h=here, t=top, b=bottom, p=page
\centering
\includegraphics[width=0.9\textwidth]{vormessung/Arrisun5vorspec} 
% Bilddatei aus dem Unterverzeichnis bilder holen, skalieren auf 0.8*Satzspiegel
\caption {Arri Sun 5: Spektrum} 
\end{figure}

\begin{figure}[htp]     % h=here, t=top, b=bottom, p=page
\centering
\includegraphics[width=0.5\textwidth]{vormessung/Arrisun5vorcct} 
% Bilddatei aus dem Unterverzeichnis bilder holen, skalieren auf 0.8*Satzspiegel
\caption {Arri Sun 5: Beleuchtungsstärke, CCT, $\Delta$UV und Farbortkoordinaten} 
\end{figure}

\begin{figure}[htp]     % h=here, t=top, b=bottom, p=page
\centering
\includegraphics[width=0.9\textwidth]{vormessung/Arrisun5vorxy} 
% Bilddatei aus dem Unterverzeichnis bilder holen, skalieren auf 0.8*Satzspiegel
\caption {Arri Sun 5: XYZ-Farbraum Detailansicht} 
\end{figure}

\begin{figure}[htp]     % h=here, t=top, b=bottom, p=page
\centering
\includegraphics[width=0.9\textwidth]{vormessung/Arrisun5vorcri} 
% Bilddatei aus dem Unterverzeichnis bilder holen, skalieren auf 0.8*Satzspiegel
\caption {Arri Sun 5: CRI} 
\end{figure}

\begin{figure}[htp]     % h=here, t=top, b=bottom, p=page
\centering
\includegraphics[width=0.9\textwidth]{vormessung/Arrisun5vorcqs} 
% Bilddatei aus dem Unterverzeichnis bilder holen, skalieren auf 0.8*Satzspiegel
\caption {Arri Sun 5: CQS} 
\end{figure}

\begin{figure}[htp]     % h=here, t=top, b=bottom, p=page
\centering
\includegraphics[width=0.9\textwidth]{vormessung/Arrisun5vortlci} 
% Bilddatei aus dem Unterverzeichnis bilder holen, skalieren auf 0.8*Satzspiegel
\caption {Arri Sun 5: TLCI} 
\end{figure}


\begin{figure}[htp]     % h=here, t=top, b=bottom, p=page
\centering
\includegraphics[width=0.9\textwidth]{vormessung/Arrisun5vortm} 
% Bilddatei aus dem Unterverzeichnis bilder holen, skalieren auf 0.8*Satzspiegel
\caption {Arri Sun 5: TM-30} 
\end{figure}

\subsection{Rottöne}

\begin{figure}[htp]     % h=here, t=top, b=bottom, p=page
\centering
\includegraphics[width=0.9\textwidth]{vormessung/1kWvor027spec} 
% Bilddatei aus dem Unterverzeichnis bilder holen, skalieren auf 0.8*Satzspiegel
\caption {027 Rot: Spektrum} 
\end{figure}

\begin{figure}[htp]     % h=here, t=top, b=bottom, p=page
\centering
\includegraphics[width=0.9\textwidth]{vormessung/1kWvor787spec} 
% Bilddatei aus dem Unterverzeichnis bilder holen, skalieren auf 0.8*Satzspiegel
\caption {787 Rot: Spektrum} 
\end{figure}

\begin{figure}[htp]     % h=here, t=top, b=bottom, p=page
\centering
\includegraphics[width=0.9\textwidth]{vormessung/1kWvor789spec} 
% Bilddatei aus dem Unterverzeichnis bilder holen, skalieren auf 0.8*Satzspiegel
\caption {789 Rot: Spektrum} 
\end{figure}



\subsection{Encore}

\subsubsection{027 Rot in 10p Schritten}

\begin{figure}[htp]     % h=here, t=top, b=bottom, p=page
\centering
\includegraphics[width=0.9\textwidth]{vormessung/encorevor02710spec} 
% Bilddatei aus dem Unterverzeichnis bilder holen, skalieren auf 0.8*Satzspiegel
\caption {Encore mit 10p 027 Rot: Spektrum} 
\end{figure}

\begin{figure}[htp]     % h=here, t=top, b=bottom, p=page
\centering
\includegraphics[width=0.5\textwidth]{vormessung/encorevor02710cct} 
% Bilddatei aus dem Unterverzeichnis bilder holen, skalieren auf 0.8*Satzspiegel
\caption {Encore mit 10p 027 Rot: Beleuchtungsstärke, CCT, $\Delta$UV und Farbortkoordinaten} 
\end{figure}

\begin{figure}[htp]     % h=here, t=top, b=bottom, p=page
\centering
\includegraphics[width=0.9\textwidth]{vormessung/encorevor02710xy} 
% Bilddatei aus dem Unterverzeichnis bilder holen, skalieren auf 0.8*Satzspiegel
\caption {Encore mit 10p 027 Rot: XYZ-Farbraum Detailansicht} 
\end{figure}

\begin{figure}[htp]     % h=here, t=top, b=bottom, p=page
\centering
\includegraphics[width=0.9\textwidth]{vormessung/encorevor02710cri} 
% Bilddatei aus dem Unterverzeichnis bilder holen, skalieren auf 0.8*Satzspiegel
\caption {Encore mit 10p 027 Rot: CRI} 
\end{figure}

\begin{figure}[htp]     % h=here, t=top, b=bottom, p=page
\centering
\includegraphics[width=0.9\textwidth]{vormessung/encorevor02710cqs} 
% Bilddatei aus dem Unterverzeichnis bilder holen, skalieren auf 0.8*Satzspiegel
\caption {Encore mit 10p 027 Rot: CQS} 
\end{figure}

\begin{figure}[htp]     % h=here, t=top, b=bottom, p=page
\centering
\includegraphics[width=0.9\textwidth]{vormessung/encorevor02710tlci} 
% Bilddatei aus dem Unterverzeichnis bilder holen, skalieren auf 0.8*Satzspiegel
\caption {Encore mit 10p 027 Rot: TLCI} 
\end{figure}

\begin{figure}[htp]     % h=here, t=top, b=bottom, p=page
\centering
\includegraphics[width=0.9\textwidth]{vormessung/encorevor02710tm} 
% Bilddatei aus dem Unterverzeichnis bilder holen, skalieren auf 0.8*Satzspiegel
\caption {Encore mit 10p 027 Rot: TM-30} 
\end{figure}




\begin{figure}[htp]     % h=here, t=top, b=bottom, p=page
\centering
\includegraphics[width=0.9\textwidth]{vormessung/encorevor02720spec} 
% Bilddatei aus dem Unterverzeichnis bilder holen, skalieren auf 0.8*Satzspiegel
\caption {Encore mit 20p 027 Rot: Spektrum} 
\end{figure}

\begin{figure}[htp]     % h=here, t=top, b=bottom, p=page
\centering
\includegraphics[width=0.5\textwidth]{vormessung/encorevor02720cct} 
% Bilddatei aus dem Unterverzeichnis bilder holen, skalieren auf 0.8*Satzspiegel
\caption {Encore mit 20p 027 Rot: Beleuchtungsstärke, CCT, $\Delta$UV und Farbortkoordinaten} 
\end{figure}

\begin{figure}[htp]     % h=here, t=top, b=bottom, p=page
\centering
\includegraphics[width=0.9\textwidth]{vormessung/encorevor02720xy} 
% Bilddatei aus dem Unterverzeichnis bilder holen, skalieren auf 0.8*Satzspiegel
\caption {Encore mit 20p 027 Rot: XYZ-Farbraum Detailansicht} 
\end{figure}

\begin{figure}[htp]     % h=here, t=top, b=bottom, p=page
\centering
\includegraphics[width=0.9\textwidth]{vormessung/encorevor02720cri} 
% Bilddatei aus dem Unterverzeichnis bilder holen, skalieren auf 0.8*Satzspiegel
\caption {Encore mit 20p 027 Rot: CRI} 
\end{figure}

\begin{figure}[htp]     % h=here, t=top, b=bottom, p=page
\centering
\includegraphics[width=0.9\textwidth]{vormessung/encorevor02720cqs} 
% Bilddatei aus dem Unterverzeichnis bilder holen, skalieren auf 0.8*Satzspiegel
\caption {Encore mit 20p 027 Rot: CQS} 
\end{figure}

\begin{figure}[htp]     % h=here, t=top, b=bottom, p=page
\centering
\includegraphics[width=0.9\textwidth]{vormessung/encorevor02720tlci} 
% Bilddatei aus dem Unterverzeichnis bilder holen, skalieren auf 0.8*Satzspiegel
\caption {Encore mit 20p 027 Rot: TLCI} 
\end{figure}

\begin{figure}[htp]     % h=here, t=top, b=bottom, p=page
\centering
\includegraphics[width=0.9\textwidth]{vormessung/encorevor02720tm} 
% Bilddatei aus dem Unterverzeichnis bilder holen, skalieren auf 0.8*Satzspiegel
\caption {Encore mit 20p 027 Rot: TM-30} 
\end{figure}



\begin{figure}[htp]     % h=here, t=top, b=bottom, p=page
\centering
\includegraphics[width=0.9\textwidth]{vormessung/encorevor02730spec} 
% Bilddatei aus dem Unterverzeichnis bilder holen, skalieren auf 0.8*Satzspiegel
\caption {Encore mit 30p 027 Rot: Spektrum} 
\end{figure}

\begin{figure}[htp]     % h=here, t=top, b=bottom, p=page
\centering
\includegraphics[width=0.5\textwidth]{vormessung/encorevor02730cct} 
% Bilddatei aus dem Unterverzeichnis bilder holen, skalieren auf 0.8*Satzspiegel
\caption {Encore mit 30p 027 Rot: Beleuchtungsstärke, CCT, $\Delta$UV und Farbortkoordinaten} 
\end{figure}

\begin{figure}[htp]     % h=here, t=top, b=bottom, p=page
\centering
\includegraphics[width=0.9\textwidth]{vormessung/encorevor02730xy} 
% Bilddatei aus dem Unterverzeichnis bilder holen, skalieren auf 0.8*Satzspiegel
\caption {Encore mit 30p 027 Rot: XYZ-Farbraum Detailansicht} 
\end{figure}

\begin{figure}[htp]     % h=here, t=top, b=bottom, p=page
\centering
\includegraphics[width=0.9\textwidth]{vormessung/encorevor02730cri} 
% Bilddatei aus dem Unterverzeichnis bilder holen, skalieren auf 0.8*Satzspiegel
\caption {Encore mit 30p 027 Rot: CRI} 
\end{figure}

\begin{figure}[htp]     % h=here, t=top, b=bottom, p=page
\centering
\includegraphics[width=0.9\textwidth]{vormessung/encorevor02730cqs} 
% Bilddatei aus dem Unterverzeichnis bilder holen, skalieren auf 0.8*Satzspiegel
\caption {Encore mit 30p 027 Rot: CQS} 
\end{figure}

\begin{figure}[htp]     % h=here, t=top, b=bottom, p=page
\centering
\includegraphics[width=0.9\textwidth]{vormessung/encorevor02730tlci} 
% Bilddatei aus dem Unterverzeichnis bilder holen, skalieren auf 0.8*Satzspiegel
\caption {Encore mit 30p 027 Rot: TLCI} 
\end{figure}

\begin{figure}[htp]     % h=here, t=top, b=bottom, p=page
\centering
\includegraphics[width=0.9\textwidth]{vormessung/encorevor02730tm} 
% Bilddatei aus dem Unterverzeichnis bilder holen, skalieren auf 0.8*Satzspiegel
\caption {Encore mit 30p 027 Rot: TM-30} 
\end{figure}



\begin{figure}[htp]     % h=here, t=top, b=bottom, p=page
\centering
\includegraphics[width=0.9\textwidth]{vormessung/encorevor02740spec} 
% Bilddatei aus dem Unterverzeichnis bilder holen, skalieren auf 0.8*Satzspiegel
\caption {Encore mit 40p 027 Rot: Spektrum} 
\end{figure}

\begin{figure}[htp]     % h=here, t=top, b=bottom, p=page
\centering
\includegraphics[width=0.5\textwidth]{vormessung/encorevor02740cct} 
% Bilddatei aus dem Unterverzeichnis bilder holen, skalieren auf 0.8*Satzspiegel
\caption {Encore mit 40p 027 Rot: Beleuchtungsstärke, CCT, $\Delta$UV und Farbortkoordinaten} 
\end{figure}

\begin{figure}[htp]     % h=here, t=top, b=bottom, p=page
\centering
\includegraphics[width=0.9\textwidth]{vormessung/encorevor02740xy} 
% Bilddatei aus dem Unterverzeichnis bilder holen, skalieren auf 0.8*Satzspiegel
\caption {Encore mit 40p 027 Rot: XYZ-Farbraum Detailansicht} 
\end{figure}

\begin{figure}[htp]     % h=here, t=top, b=bottom, p=page
\centering
\includegraphics[width=0.9\textwidth]{vormessung/encorevor02740cri} 
% Bilddatei aus dem Unterverzeichnis bilder holen, skalieren auf 0.8*Satzspiegel
\caption {Encore mit 40p 027 Rot: CRI} 
\end{figure}

\begin{figure}[htp]     % h=here, t=top, b=bottom, p=page
\centering
\includegraphics[width=0.9\textwidth]{vormessung/encorevor02740cqs} 
% Bilddatei aus dem Unterverzeichnis bilder holen, skalieren auf 0.8*Satzspiegel
\caption {Encore mit 40p 027 Rot: CQS} 
\end{figure}

\begin{figure}[htp]     % h=here, t=top, b=bottom, p=page
\centering
\includegraphics[width=0.9\textwidth]{vormessung/encorevor02740tlci} 
% Bilddatei aus dem Unterverzeichnis bilder holen, skalieren auf 0.8*Satzspiegel
\caption {Encore mit 40p 027 Rot: TLCI} 
\end{figure}

\begin{figure}[htp]     % h=here, t=top, b=bottom, p=page
\centering
\includegraphics[width=0.9\textwidth]{vormessung/encorevor02740tm} 
% Bilddatei aus dem Unterverzeichnis bilder holen, skalieren auf 0.8*Satzspiegel
\caption {Encore mit 40p 027 Rot: TM-30} 
\end{figure}

\begin{figure}[htp]     % h=here, t=top, b=bottom, p=page
\centering
\includegraphics[width=0.9\textwidth]{vormessung/encorevor02750spec} 
% Bilddatei aus dem Unterverzeichnis bilder holen, skalieren auf 0.8*Satzspiegel
\caption {Encore mit 50p 027 Rot: Spektrum} 
\end{figure}

\begin{figure}[htp]     % h=here, t=top, b=bottom, p=page
\centering
\includegraphics[width=0.5\textwidth]{vormessung/encorevor02750cct} 
% Bilddatei aus dem Unterverzeichnis bilder holen, skalieren auf 0.8*Satzspiegel
\caption {Encore mit 50p 027 Rot: Beleuchtungsstärke, CCT, $\Delta$UV und Farbortkoordinaten} 
\end{figure}

\begin{figure}[htp]     % h=here, t=top, b=bottom, p=page
\centering
\includegraphics[width=0.9\textwidth]{vormessung/encorevor02750xy} 
% Bilddatei aus dem Unterverzeichnis bilder holen, skalieren auf 0.8*Satzspiegel
\caption {Encore mit 50p 027 Rot: XYZ-Farbraum Detailansicht} 
\end{figure}

\begin{figure}[htp]     % h=here, t=top, b=bottom, p=page
\centering
\includegraphics[width=0.9\textwidth]{vormessung/encorevor02750cri} 
% Bilddatei aus dem Unterverzeichnis bilder holen, skalieren auf 0.8*Satzspiegel
\caption {Encore mit 50p 027 Rot: CRI} 
\end{figure}

\begin{figure}[htp]     % h=here, t=top, b=bottom, p=page
\centering
\includegraphics[width=0.9\textwidth]{vormessung/encorevor02750cqs} 
% Bilddatei aus dem Unterverzeichnis bilder holen, skalieren auf 0.8*Satzspiegel
\caption {Encore mit 50p 027 Rot: CQS} 
\end{figure}

\begin{figure}[htp]     % h=here, t=top, b=bottom, p=page
\centering
\includegraphics[width=0.9\textwidth]{vormessung/encorevor02750tlci} 
% Bilddatei aus dem Unterverzeichnis bilder holen, skalieren auf 0.8*Satzspiegel
\caption {Encore mit 50p 027 Rot: TLCI} 
\end{figure}

\begin{figure}[htp]     % h=here, t=top, b=bottom, p=page
\centering
\includegraphics[width=0.9\textwidth]{vormessung/encorevor02750tm} 
% Bilddatei aus dem Unterverzeichnis bilder holen, skalieren auf 0.8*Satzspiegel
\caption {Encore mit 50p 027 Rot: TM-30} 
\end{figure}




\begin{figure}[htp]     % h=here, t=top, b=bottom, p=page
\centering
\includegraphics[width=0.9\textwidth]{vormessung/encorevor02760spec} 
% Bilddatei aus dem Unterverzeichnis bilder holen, skalieren auf 0.8*Satzspiegel
\caption {Encore mit 60p 027 Rot: Spektrum} 
\end{figure}

\begin{figure}[htp]     % h=here, t=top, b=bottom, p=page
\centering
\includegraphics[width=0.5\textwidth]{vormessung/encorevor02760cct} 
% Bilddatei aus dem Unterverzeichnis bilder holen, skalieren auf 0.8*Satzspiegel
\caption {Encore mit 60p 027 Rot: Beleuchtungsstärke, CCT, $\Delta$UV und Farbortkoordinaten} 
\end{figure}

\begin{figure}[htp]     % h=here, t=top, b=bottom, p=page
\centering
\includegraphics[width=0.9\textwidth]{vormessung/encorevor02760xy} 
% Bilddatei aus dem Unterverzeichnis bilder holen, skalieren auf 0.8*Satzspiegel
\caption {Encore mit 60p 027 Rot: XYZ-Farbraum Detailansicht} 
\end{figure}

\begin{figure}[htp]     % h=here, t=top, b=bottom, p=page
\centering
\includegraphics[width=0.9\textwidth]{vormessung/encorevor02760cri} 
% Bilddatei aus dem Unterverzeichnis bilder holen, skalieren auf 0.8*Satzspiegel
\caption {Encore mit 60p 027 Rot: CRI} 
\end{figure}

\begin{figure}[htp]     % h=here, t=top, b=bottom, p=page
\centering
\includegraphics[width=0.9\textwidth]{vormessung/encorevor02760cqs} 
% Bilddatei aus dem Unterverzeichnis bilder holen, skalieren auf 0.8*Satzspiegel
\caption {Encore mit 60p 027 Rot: CQS} 
\end{figure}

\begin{figure}[htp]     % h=here, t=top, b=bottom, p=page
\centering
\includegraphics[width=0.9\textwidth]{vormessung/encorevor02760tlci} 
% Bilddatei aus dem Unterverzeichnis bilder holen, skalieren auf 0.8*Satzspiegel
\caption {Encore mit 60p 027 Rot: TLCI} 
\end{figure}

\begin{figure}[htp]     % h=here, t=top, b=bottom, p=page
\centering
\includegraphics[width=0.9\textwidth]{vormessung/encorevor02760tm} 
% Bilddatei aus dem Unterverzeichnis bilder holen, skalieren auf 0.8*Satzspiegel
\caption {Encore mit 60p 027 Rot: TM-30} 
\end{figure}




\begin{figure}[htp]     % h=here, t=top, b=bottom, p=page
\centering
\includegraphics[width=0.9\textwidth]{vormessung/encorevor02770spec} 
% Bilddatei aus dem Unterverzeichnis bilder holen, skalieren auf 0.8*Satzspiegel
\caption {Encore mit 70p 027 Rot: Spektrum} 
\end{figure}

\begin{figure}[htp]     % h=here, t=top, b=bottom, p=page
\centering
\includegraphics[width=0.5\textwidth]{vormessung/encorevor02770cct} 
% Bilddatei aus dem Unterverzeichnis bilder holen, skalieren auf 0.8*Satzspiegel
\caption {Encore mit 70p 027 Rot: Beleuchtungsstärke, CCT, $\Delta$UV und Farbortkoordinaten} 
\end{figure}

\begin{figure}[htp]     % h=here, t=top, b=bottom, p=page
\centering
\includegraphics[width=0.9\textwidth]{vormessung/encorevor02770xy} 
% Bilddatei aus dem Unterverzeichnis bilder holen, skalieren auf 0.8*Satzspiegel
\caption {Encore mit 70p 027 Rot: XYZ-Farbraum Detailansicht} 
\end{figure}

\begin{figure}[htp]     % h=here, t=top, b=bottom, p=page
\centering
\includegraphics[width=0.9\textwidth]{vormessung/encorevor02770cri} 
% Bilddatei aus dem Unterverzeichnis bilder holen, skalieren auf 0.8*Satzspiegel
\caption {Encore mit 70p 027 Rot: CRI} 
\end{figure}

\begin{figure}[htp]     % h=here, t=top, b=bottom, p=page
\centering
\includegraphics[width=0.9\textwidth]{vormessung/encorevor02770cqs} 
% Bilddatei aus dem Unterverzeichnis bilder holen, skalieren auf 0.8*Satzspiegel
\caption {Encore mit 70p 027 Rot: CQS} 
\end{figure}

\begin{figure}[htp]     % h=here, t=top, b=bottom, p=page
\centering
\includegraphics[width=0.9\textwidth]{vormessung/encorevor02770tlci} 
% Bilddatei aus dem Unterverzeichnis bilder holen, skalieren auf 0.8*Satzspiegel
\caption {Encore mit 70p 027 Rot: TLCI} 
\end{figure}

\begin{figure}[htp]     % h=here, t=top, b=bottom, p=page
\centering
\includegraphics[width=0.9\textwidth]{vormessung/encorevor02770tm} 
% Bilddatei aus dem Unterverzeichnis bilder holen, skalieren auf 0.8*Satzspiegel
\caption {Encore mit 70p 027 Rot: TM-30} 
\end{figure}




\begin{figure}[htp]     % h=here, t=top, b=bottom, p=page
\centering
\includegraphics[width=0.9\textwidth]{vormessung/encorevor02780spec} 
% Bilddatei aus dem Unterverzeichnis bilder holen, skalieren auf 0.8*Satzspiegel
\caption {Encore mit 80p 027 Rot: Spektrum} 
\end{figure}

\begin{figure}[htp]     % h=here, t=top, b=bottom, p=page
\centering
\includegraphics[width=0.5\textwidth]{vormessung/encorevor02780cct} 
% Bilddatei aus dem Unterverzeichnis bilder holen, skalieren auf 0.8*Satzspiegel
\caption {Encore mit 80p 027 Rot: Beleuchtungsstärke, CCT, $\Delta$UV und Farbortkoordinaten} 
\end{figure}

\begin{figure}[htp]     % h=here, t=top, b=bottom, p=page
\centering
\includegraphics[width=0.9\textwidth]{vormessung/encorevor02780xy} 
% Bilddatei aus dem Unterverzeichnis bilder holen, skalieren auf 0.8*Satzspiegel
\caption {Encore mit 80p 027 Rot: XYZ-Farbraum Detailansicht} 
\end{figure}

\begin{figure}[htp]     % h=here, t=top, b=bottom, p=page
\centering
\includegraphics[width=0.9\textwidth]{vormessung/encorevor02780cri} 
% Bilddatei aus dem Unterverzeichnis bilder holen, skalieren auf 0.8*Satzspiegel
\caption {Encore mit 80p 027 Rot: CRI} 
\end{figure}

\begin{figure}[htp]     % h=here, t=top, b=bottom, p=page
\centering
\includegraphics[width=0.9\textwidth]{vormessung/encorevor02780cqs} 
% Bilddatei aus dem Unterverzeichnis bilder holen, skalieren auf 0.8*Satzspiegel
\caption {Encore mit 80p 027 Rot: CQS} 
\end{figure}

\begin{figure}[htp]     % h=here, t=top, b=bottom, p=page
\centering
\includegraphics[width=0.9\textwidth]{vormessung/encorevor02780tlci} 
% Bilddatei aus dem Unterverzeichnis bilder holen, skalieren auf 0.8*Satzspiegel
\caption {Encore mit 80p 027 Rot: TLCI} 
\end{figure}

\begin{figure}[htp]     % h=here, t=top, b=bottom, p=page
\centering
\includegraphics[width=0.9\textwidth]{vormessung/encorevor02780tm} 
% Bilddatei aus dem Unterverzeichnis bilder holen, skalieren auf 0.8*Satzspiegel
\caption {Encore mit 80p 027 Rot: TM-30} 
\end{figure}




\begin{figure}[htp]     % h=here, t=top, b=bottom, p=page
\centering
\includegraphics[width=0.9\textwidth]{vormessung/encorevor02790spec} 
% Bilddatei aus dem Unterverzeichnis bilder holen, skalieren auf 0.8*Satzspiegel
\caption {Encore mit 90p 027 Rot: Spektrum} 
\end{figure}

\begin{figure}[htp]     % h=here, t=top, b=bottom, p=page
\centering
\includegraphics[width=0.5\textwidth]{vormessung/encorevor02790cct} 
% Bilddatei aus dem Unterverzeichnis bilder holen, skalieren auf 0.8*Satzspiegel
\caption {Encore mit 90p 027 Rot: Beleuchtungsstärke, CCT, $\Delta$UV und Farbortkoordinaten} 
\end{figure}

\begin{figure}[htp]     % h=here, t=top, b=bottom, p=page
\centering
\includegraphics[width=0.9\textwidth]{vormessung/encorevor02790xy} 
% Bilddatei aus dem Unterverzeichnis bilder holen, skalieren auf 0.8*Satzspiegel
\caption {Encore mit 90p 027 Rot: XYZ-Farbraum Detailansicht} 
\end{figure}

\begin{figure}[htp]     % h=here, t=top, b=bottom, p=page
\centering
\includegraphics[width=0.9\textwidth]{vormessung/encorevor02790cri} 
% Bilddatei aus dem Unterverzeichnis bilder holen, skalieren auf 0.8*Satzspiegel
\caption {Encore mit 90p 027 Rot: CRI} 
\end{figure}

\begin{figure}[htp]     % h=here, t=top, b=bottom, p=page
\centering
\includegraphics[width=0.9\textwidth]{vormessung/encorevor02790cqs} 
% Bilddatei aus dem Unterverzeichnis bilder holen, skalieren auf 0.8*Satzspiegel
\caption {Encore mit 90p 027 Rot: CQS} 
\end{figure}

\begin{figure}[htp]     % h=here, t=top, b=bottom, p=page
\centering
\includegraphics[width=0.9\textwidth]{vormessung/encorevor02790tlci} 
% Bilddatei aus dem Unterverzeichnis bilder holen, skalieren auf 0.8*Satzspiegel
\caption {Encore mit 90p 027 Rot: TLCI} 
\end{figure}

\begin{figure}[htp]     % h=here, t=top, b=bottom, p=page
\centering
\includegraphics[width=0.9\textwidth]{vormessung/encorevor02790tm} 
% Bilddatei aus dem Unterverzeichnis bilder holen, skalieren auf 0.8*Satzspiegel
\caption {Encore mit 90p 027 Rot: TM-30} 
\end{figure}




\begin{figure}[htp]     % h=here, t=top, b=bottom, p=page
\centering
\includegraphics[width=0.9\textwidth]{vormessung/encorevor027100spec} 
% Bilddatei aus dem Unterverzeichnis bilder holen, skalieren auf 0.8*Satzspiegel
\caption {Encore mit 20p 027 Rot: Spektrum} 
\end{figure}

\begin{figure}[htp]     % h=here, t=top, b=bottom, p=page
\centering
\includegraphics[width=0.5\textwidth]{vormessung/encorevor02720cct} 
% Bilddatei aus dem Unterverzeichnis bilder holen, skalieren auf 0.8*Satzspiegel
\caption {Encore mit 100p 027 Rot: Beleuchtungsstärke, CCT, $\Delta$UV und Farbortkoordinaten} 
\end{figure}

\begin{figure}[htp]     % h=here, t=top, b=bottom, p=page
\centering
\includegraphics[width=0.9\textwidth]{vormessung/encorevor027100xy} 
% Bilddatei aus dem Unterverzeichnis bilder holen, skalieren auf 0.8*Satzspiegel
\caption {Encore mit 100p 027 Rot: XYZ-Farbraum Detailansicht} 
\end{figure}

\begin{figure}[htp]     % h=here, t=top, b=bottom, p=page
\centering
\includegraphics[width=0.9\textwidth]{vormessung/encorevor027100cri} 
% Bilddatei aus dem Unterverzeichnis bilder holen, skalieren auf 0.8*Satzspiegel
\caption {Encore mit 100p 027 Rot: CRI} 
\end{figure}

\begin{figure}[htp]     % h=here, t=top, b=bottom, p=page
\centering
\includegraphics[width=0.9\textwidth]{vormessung/encorevor027100cqs} 
% Bilddatei aus dem Unterverzeichnis bilder holen, skalieren auf 0.8*Satzspiegel
\caption {Encore mit 100p 027 Rot: CQS} 
\end{figure}

\begin{figure}[htp]     % h=here, t=top, b=bottom, p=page
\centering
\includegraphics[width=0.9\textwidth]{vormessung/encorevor027100tlci} 
% Bilddatei aus dem Unterverzeichnis bilder holen, skalieren auf 0.8*Satzspiegel
\caption {Encore mit 100p 027 Rot: TLCI} 
\end{figure}

\begin{figure}[htp]     % h=here, t=top, b=bottom, p=page
\centering
\includegraphics[width=0.9\textwidth]{vormessung/encorevor027100tm} 
% Bilddatei aus dem Unterverzeichnis bilder holen, skalieren auf 0.8*Satzspiegel
\caption {Encore mit 100p 027 Rot: TM-30} 
\end{figure}


\subsubsection{787 Rot in 10p Schritten}

\begin{figure}[htp]     % h=here, t=top, b=bottom, p=page
\centering
\includegraphics[width=0.9\textwidth]{vormessung/encorevor78710spec} 
% Bilddatei aus dem Unterverzeichnis bilder holen, skalieren auf 0.8*Satzspiegel
\caption {Encore mit 10p 787 Rot: Spektrum} 
\end{figure}

\begin{figure}[htp]     % h=here, t=top, b=bottom, p=page
\centering
\includegraphics[width=0.5\textwidth]{vormessung/encorevor78710cct} 
% Bilddatei aus dem Unterverzeichnis bilder holen, skalieren auf 0.8*Satzspiegel
\caption {Encore mit 10p 787 Rot: Beleuchtungsstärke, CCT, $\Delta$UV und Farbortkoordinaten} 
\end{figure}

\begin{figure}[htp]     % h=here, t=top, b=bottom, p=page
\centering
\includegraphics[width=0.9\textwidth]{vormessung/encorevor07870xy} 
% Bilddatei aus dem Unterverzeichnis bilder holen, skalieren auf 0.8*Satzspiegel
\caption {Encore mit 10p 787 Rot: XYZ-Farbraum Detailansicht} 
\end{figure}

\begin{figure}[htp]     % h=here, t=top, b=bottom, p=page
\centering
\includegraphics[width=0.9\textwidth]{vormessung/encorevor78710cri} 
% Bilddatei aus dem Unterverzeichnis bilder holen, skalieren auf 0.8*Satzspiegel
\caption {Encore mit 10p 787 Rot: CRI} 
\end{figure}

\begin{figure}[htp]     % h=here, t=top, b=bottom, p=page
\centering
\includegraphics[width=0.9\textwidth]{vormessung/encorevor78710cqs} 
% Bilddatei aus dem Unterverzeichnis bilder holen, skalieren auf 0.8*Satzspiegel
\caption {Encore mit 10p 787 Rot: CQS} 
\end{figure}

\begin{figure}[htp]     % h=here, t=top, b=bottom, p=page
\centering
\includegraphics[width=0.9\textwidth]{vormessung/encorevor78710tlci} 
% Bilddatei aus dem Unterverzeichnis bilder holen, skalieren auf 0.8*Satzspiegel
\caption {Encore mit 10p 787 Rot: TLCI} 
\end{figure}

\begin{figure}[htp]     % h=here, t=top, b=bottom, p=page
\centering
\includegraphics[width=0.9\textwidth]{vormessung/encorevor78710tm} 
% Bilddatei aus dem Unterverzeichnis bilder holen, skalieren auf 0.8*Satzspiegel
\caption {Encore mit 10p 787 Rot: TM-30} 
\end{figure}




\begin{figure}[htp]     % h=here, t=top, b=bottom, p=page
\centering
\includegraphics[width=0.9\textwidth]{vormessung/encorevor78720spec} 
% Bilddatei aus dem Unterverzeichnis bilder holen, skalieren auf 0.8*Satzspiegel
\caption {Encore mit 20p 787 Rot: Spektrum} 
\end{figure}

\begin{figure}[htp]     % h=here, t=top, b=bottom, p=page
\centering
\includegraphics[width=0.5\textwidth]{vormessung/encorevor78720cct} 
% Bilddatei aus dem Unterverzeichnis bilder holen, skalieren auf 0.8*Satzspiegel
\caption {Encore mit 20p 027 Rot: Beleuchtungsstärke, CCT, $\Delta$UV und Farbortkoordinaten} 
\end{figure}

\begin{figure}[htp]     % h=here, t=top, b=bottom, p=page
\centering
\includegraphics[width=0.9\textwidth]{vormessung/encorevor02720xy} 
% Bilddatei aus dem Unterverzeichnis bilder holen, skalieren auf 0.8*Satzspiegel
\caption {Encore mit 20p 787 Rot: XYZ-Farbraum Detailansicht} 
\end{figure}

\begin{figure}[htp]     % h=here, t=top, b=bottom, p=page
\centering
\includegraphics[width=0.9\textwidth]{vormessung/encorevor78720cri} 
% Bilddatei aus dem Unterverzeichnis bilder holen, skalieren auf 0.8*Satzspiegel
\caption {Encore mit 20p 787 Rot: CRI} 
\end{figure}

\begin{figure}[htp]     % h=here, t=top, b=bottom, p=page
\centering
\includegraphics[width=0.9\textwidth]{vormessung/encorevor78720cqs} 
% Bilddatei aus dem Unterverzeichnis bilder holen, skalieren auf 0.8*Satzspiegel
\caption {Encore mit 20p 787 Rot: CQS} 
\end{figure}

\begin{figure}[htp]     % h=here, t=top, b=bottom, p=page
\centering
\includegraphics[width=0.9\textwidth]{vormessung/encorevor78720tlci} 
% Bilddatei aus dem Unterverzeichnis bilder holen, skalieren auf 0.8*Satzspiegel
\caption {Encore mit 20p 787 Rot: TLCI} 
\end{figure}

\begin{figure}[htp]     % h=here, t=top, b=bottom, p=page
\centering
\includegraphics[width=0.9\textwidth]{vormessung/encorevor78720tm} 
% Bilddatei aus dem Unterverzeichnis bilder holen, skalieren auf 0.8*Satzspiegel
\caption {Encore mit 20p 787 Rot: TM-30} 
\end{figure}



\begin{figure}[htp]     % h=here, t=top, b=bottom, p=page
\centering
\includegraphics[width=0.9\textwidth]{vormessung/encorevor78730spec} 
% Bilddatei aus dem Unterverzeichnis bilder holen, skalieren auf 0.8*Satzspiegel
\caption {Encore mit 30p 787 Rot: Spektrum} 
\end{figure}

\begin{figure}[htp]     % h=here, t=top, b=bottom, p=page
\centering
\includegraphics[width=0.5\textwidth]{vormessung/encorevor78730cct} 
% Bilddatei aus dem Unterverzeichnis bilder holen, skalieren auf 0.8*Satzspiegel
\caption {Encore mit 30p 787 Rot: Beleuchtungsstärke, CCT, $\Delta$UV und Farbortkoordinaten} 
\end{figure}

\begin{figure}[htp]     % h=here, t=top, b=bottom, p=page
\centering
\includegraphics[width=0.9\textwidth]{vormessung/encorevor78730xy} 
% Bilddatei aus dem Unterverzeichnis bilder holen, skalieren auf 0.8*Satzspiegel
\caption {Encore mit 30p 787 Rot: XYZ-Farbraum Detailansicht} 
\end{figure}

\begin{figure}[htp]     % h=here, t=top, b=bottom, p=page
\centering
\includegraphics[width=0.9\textwidth]{vormessung/encorevor78730cri} 
% Bilddatei aus dem Unterverzeichnis bilder holen, skalieren auf 0.8*Satzspiegel
\caption {Encore mit 30p 787 Rot: CRI} 
\end{figure}

\begin{figure}[htp]     % h=here, t=top, b=bottom, p=page
\centering
\includegraphics[width=0.9\textwidth]{vormessung/encorevor78730cqs} 
% Bilddatei aus dem Unterverzeichnis bilder holen, skalieren auf 0.8*Satzspiegel
\caption {Encore mit 30p 787 Rot: CQS} 
\end{figure}

\begin{figure}[htp]     % h=here, t=top, b=bottom, p=page
\centering
\includegraphics[width=0.9\textwidth]{vormessung/encorevor78730tlci} 
% Bilddatei aus dem Unterverzeichnis bilder holen, skalieren auf 0.8*Satzspiegel
\caption {Encore mit 30p 787 Rot: TLCI} 
\end{figure}

\begin{figure}[htp]     % h=here, t=top, b=bottom, p=page
\centering
\includegraphics[width=0.9\textwidth]{vormessung/encorevor78730tm} 
% Bilddatei aus dem Unterverzeichnis bilder holen, skalieren auf 0.8*Satzspiegel
\caption {Encore mit 30p 787 Rot: TM-30} 
\end{figure}



\begin{figure}[htp]     % h=here, t=top, b=bottom, p=page
\centering
\includegraphics[width=0.9\textwidth]{vormessung/encorevor78740spec} 
% Bilddatei aus dem Unterverzeichnis bilder holen, skalieren auf 0.8*Satzspiegel
\caption {Encore mit 40p 787 Rot: Spektrum} 
\end{figure}

\begin{figure}[htp]     % h=here, t=top, b=bottom, p=page
\centering
\includegraphics[width=0.5\textwidth]{vormessung/encorevor78740cct} 
% Bilddatei aus dem Unterverzeichnis bilder holen, skalieren auf 0.8*Satzspiegel
\caption {Encore mit 40p 787 Rot: Beleuchtungsstärke, CCT, $\Delta$UV und Farbortkoordinaten} 
\end{figure}

\begin{figure}[htp]     % h=here, t=top, b=bottom, p=page
\centering
\includegraphics[width=0.9\textwidth]{vormessung/encorevor78740xy} 
% Bilddatei aus dem Unterverzeichnis bilder holen, skalieren auf 0.8*Satzspiegel
\caption {Encore mit 40p 787 Rot: XYZ-Farbraum Detailansicht} 
\end{figure}

\begin{figure}[htp]     % h=here, t=top, b=bottom, p=page
\centering
\includegraphics[width=0.9\textwidth]{vormessung/encorevor78740cri} 
% Bilddatei aus dem Unterverzeichnis bilder holen, skalieren auf 0.8*Satzspiegel
\caption {Encore mit 40p 787 Rot: CRI} 
\end{figure}

\begin{figure}[htp]     % h=here, t=top, b=bottom, p=page
\centering
\includegraphics[width=0.9\textwidth]{vormessung/encorevor78740cqs} 
% Bilddatei aus dem Unterverzeichnis bilder holen, skalieren auf 0.8*Satzspiegel
\caption {Encore mit 40p 787 Rot: CQS} 
\end{figure}

\begin{figure}[htp]     % h=here, t=top, b=bottom, p=page
\centering
\includegraphics[width=0.9\textwidth]{vormessung/encorevor78740tlci} 
% Bilddatei aus dem Unterverzeichnis bilder holen, skalieren auf 0.8*Satzspiegel
\caption {Encore mit 40p 787 Rot: TLCI} 
\end{figure}

\begin{figure}[htp]     % h=here, t=top, b=bottom, p=page
\centering
\includegraphics[width=0.9\textwidth]{vormessung/encorevor78740tm} 
% Bilddatei aus dem Unterverzeichnis bilder holen, skalieren auf 0.8*Satzspiegel
\caption {Encore mit 40p 787 Rot: TM-30} 
\end{figure}




\begin{figure}[htp]     % h=here, t=top, b=bottom, p=page
\centering
\includegraphics[width=0.9\textwidth]{vormessung/encorevor78750spec} 
% Bilddatei aus dem Unterverzeichnis bilder holen, skalieren auf 0.8*Satzspiegel
\caption {Encore mit 50p 787 Rot: Spektrum} 
\end{figure}

\begin{figure}[htp]     % h=here, t=top, b=bottom, p=page
\centering
\includegraphics[width=0.5\textwidth]{vormessung/encorevor78750cct} 
% Bilddatei aus dem Unterverzeichnis bilder holen, skalieren auf 0.8*Satzspiegel
\caption {Encore mit 50p 787 Rot: Beleuchtungsstärke, CCT, $\Delta$UV und Farbortkoordinaten} 
\end{figure}

\begin{figure}[htp]     % h=here, t=top, b=bottom, p=page
\centering
\includegraphics[width=0.9\textwidth]{vormessung/encorevor78750xy} 
% Bilddatei aus dem Unterverzeichnis bilder holen, skalieren auf 0.8*Satzspiegel
\caption {Encore mit 50p 787 Rot: XYZ-Farbraum Detailansicht} 
\end{figure}

\begin{figure}[htp]     % h=here, t=top, b=bottom, p=page
\centering
\includegraphics[width=0.9\textwidth]{vormessung/encorevor78750cri} 
% Bilddatei aus dem Unterverzeichnis bilder holen, skalieren auf 0.8*Satzspiegel
\caption {Encore mit 50p 787 Rot: CRI} 
\end{figure}

\begin{figure}[htp]     % h=here, t=top, b=bottom, p=page
\centering
\includegraphics[width=0.9\textwidth]{vormessung/encorevor78750cqs} 
% Bilddatei aus dem Unterverzeichnis bilder holen, skalieren auf 0.8*Satzspiegel
\caption {Encore mit 50p 787 Rot: CQS} 
\end{figure}

\begin{figure}[htp]     % h=here, t=top, b=bottom, p=page
\centering
\includegraphics[width=0.9\textwidth]{vormessung/encorevor78750tlci} 
% Bilddatei aus dem Unterverzeichnis bilder holen, skalieren auf 0.8*Satzspiegel
\caption {Encore mit 50p 787 Rot: TLCI} 
\end{figure}

\begin{figure}[htp]     % h=here, t=top, b=bottom, p=page
\centering
\includegraphics[width=0.9\textwidth]{vormessung/encorevor78750tm} 
% Bilddatei aus dem Unterverzeichnis bilder holen, skalieren auf 0.8*Satzspiegel
\caption {Encore mit 50p 787 Rot: TM-30} 
\end{figure}




\begin{figure}[htp]     % h=here, t=top, b=bottom, p=page
\centering
\includegraphics[width=0.9\textwidth]{vormessung/encorevor78760spec} 
% Bilddatei aus dem Unterverzeichnis bilder holen, skalieren auf 0.8*Satzspiegel
\caption {Encore mit 60p 787 Rot: Spektrum} 
\end{figure}

\begin{figure}[htp]     % h=here, t=top, b=bottom, p=page
\centering
\includegraphics[width=0.5\textwidth]{vormessung/encorevor78760cct} 
% Bilddatei aus dem Unterverzeichnis bilder holen, skalieren auf 0.8*Satzspiegel
\caption {Encore mit 60p 787 Rot: Beleuchtungsstärke, CCT, $\Delta$UV und Farbortkoordinaten} 
\end{figure}

\begin{figure}[htp]     % h=here, t=top, b=bottom, p=page
\centering
\includegraphics[width=0.9\textwidth]{vormessung/encorevor78760xy} 
% Bilddatei aus dem Unterverzeichnis bilder holen, skalieren auf 0.8*Satzspiegel
\caption {Encore mit 60p 787 Rot: XYZ-Farbraum Detailansicht} 
\end{figure}

\begin{figure}[htp]     % h=here, t=top, b=bottom, p=page
\centering
\includegraphics[width=0.9\textwidth]{vormessung/encorevor78760cri} 
% Bilddatei aus dem Unterverzeichnis bilder holen, skalieren auf 0.8*Satzspiegel
\caption {Encore mit 60p 787 Rot: CRI} 
\end{figure}

\begin{figure}[htp]     % h=here, t=top, b=bottom, p=page
\centering
\includegraphics[width=0.9\textwidth]{vormessung/encorevor78760cqs} 
% Bilddatei aus dem Unterverzeichnis bilder holen, skalieren auf 0.8*Satzspiegel
\caption {Encore mit 60p 787 Rot: CQS} 
\end{figure}

\begin{figure}[htp]     % h=here, t=top, b=bottom, p=page
\centering
\includegraphics[width=0.9\textwidth]{vormessung/encorevor78760tlci} 
% Bilddatei aus dem Unterverzeichnis bilder holen, skalieren auf 0.8*Satzspiegel
\caption {Encore mit 60p 787 Rot: TLCI} 
\end{figure}

\begin{figure}[htp]     % h=here, t=top, b=bottom, p=page
\centering
\includegraphics[width=0.9\textwidth]{vormessung/encorevor78760tm} 
% Bilddatei aus dem Unterverzeichnis bilder holen, skalieren auf 0.8*Satzspiegel
\caption {Encore mit 60p 787 Rot: TM-30} 
\end{figure}




\begin{figure}[htp]     % h=here, t=top, b=bottom, p=page
\centering
\includegraphics[width=0.9\textwidth]{vormessung/encorevor78770spec} 
% Bilddatei aus dem Unterverzeichnis bilder holen, skalieren auf 0.8*Satzspiegel
\caption {Encore mit 70p 787 Rot: Spektrum} 
\end{figure}

\begin{figure}[htp]     % h=here, t=top, b=bottom, p=page
\centering
\includegraphics[width=0.5\textwidth]{vormessung/encorevor78770cct} 
% Bilddatei aus dem Unterverzeichnis bilder holen, skalieren auf 0.8*Satzspiegel
\caption {Encore mit 70p 787 Rot: Beleuchtungsstärke, CCT, $\Delta$UV und Farbortkoordinaten} 
\end{figure}

\begin{figure}[htp]     % h=here, t=top, b=bottom, p=page
\centering
\includegraphics[width=0.9\textwidth]{vormessung/encorevor78770xy} 
% Bilddatei aus dem Unterverzeichnis bilder holen, skalieren auf 0.8*Satzspiegel
\caption {Encore mit 70p 787 Rot: XYZ-Farbraum Detailansicht} 
\end{figure}

\begin{figure}[htp]     % h=here, t=top, b=bottom, p=page
\centering
\includegraphics[width=0.9\textwidth]{vormessung/encorevor78770cri} 
% Bilddatei aus dem Unterverzeichnis bilder holen, skalieren auf 0.8*Satzspiegel
\caption {Encore mit 70p 787 Rot: CRI} 
\end{figure}

\begin{figure}[htp]     % h=here, t=top, b=bottom, p=page
\centering
\includegraphics[width=0.9\textwidth]{vormessung/encorevor78770cqs} 
% Bilddatei aus dem Unterverzeichnis bilder holen, skalieren auf 0.8*Satzspiegel
\caption {Encore mit 70p 787 Rot: CQS} 
\end{figure}

\begin{figure}[htp]     % h=here, t=top, b=bottom, p=page
\centering
\includegraphics[width=0.9\textwidth]{vormessung/encorevor78770tlci} 
% Bilddatei aus dem Unterverzeichnis bilder holen, skalieren auf 0.8*Satzspiegel
\caption {Encore mit 70p 787 Rot: TLCI} 
\end{figure}

\begin{figure}[htp]     % h=here, t=top, b=bottom, p=page
\centering
\includegraphics[width=0.9\textwidth]{vormessung/encorevor78770tm} 
% Bilddatei aus dem Unterverzeichnis bilder holen, skalieren auf 0.8*Satzspiegel
\caption {Encore mit 70p 787 Rot: TM-30} 
\end{figure}




\begin{figure}[htp]     % h=here, t=top, b=bottom, p=page
\centering
\includegraphics[width=0.9\textwidth]{vormessung/encorevor78780spec} 
% Bilddatei aus dem Unterverzeichnis bilder holen, skalieren auf 0.8*Satzspiegel
\caption {Encore mit 80p 787 Rot: Spektrum} 
\end{figure}

\begin{figure}[htp]     % h=here, t=top, b=bottom, p=page
\centering
\includegraphics[width=0.5\textwidth]{vormessung/encorevor78780cct} 
% Bilddatei aus dem Unterverzeichnis bilder holen, skalieren auf 0.8*Satzspiegel
\caption {Encore mit 80p 787 Rot: Beleuchtungsstärke, CCT, $\Delta$UV und Farbortkoordinaten} 
\end{figure}

\begin{figure}[htp]     % h=here, t=top, b=bottom, p=page
\centering
\includegraphics[width=0.9\textwidth]{vormessung/encorevor78780xy} 
% Bilddatei aus dem Unterverzeichnis bilder holen, skalieren auf 0.8*Satzspiegel
\caption {Encore mit 80p 787 Rot: XYZ-Farbraum Detailansicht} 
\end{figure}

\begin{figure}[htp]     % h=here, t=top, b=bottom, p=page
\centering
\includegraphics[width=0.9\textwidth]{vormessung/encorevor78780cri} 
% Bilddatei aus dem Unterverzeichnis bilder holen, skalieren auf 0.8*Satzspiegel
\caption {Encore mit 80p 787 Rot: CRI} 
\end{figure}

\begin{figure}[htp]     % h=here, t=top, b=bottom, p=page
\centering
\includegraphics[width=0.9\textwidth]{vormessung/encorevor78780cqs} 
% Bilddatei aus dem Unterverzeichnis bilder holen, skalieren auf 0.8*Satzspiegel
\caption {Encore mit 80p 787 Rot: CQS} 
\end{figure}

\begin{figure}[htp]     % h=here, t=top, b=bottom, p=page
\centering
\includegraphics[width=0.9\textwidth]{vormessung/encorevor78780tlci} 
% Bilddatei aus dem Unterverzeichnis bilder holen, skalieren auf 0.8*Satzspiegel
\caption {Encore mit 80p 787 Rot: TLCI} 
\end{figure}

\begin{figure}[htp]     % h=here, t=top, b=bottom, p=page
\centering
\includegraphics[width=0.9\textwidth]{vormessung/encorevor78780tm} 
% Bilddatei aus dem Unterverzeichnis bilder holen, skalieren auf 0.8*Satzspiegel
\caption {Encore mit 80p 787 Rot: TM-30} 
\end{figure}




\begin{figure}[htp]     % h=here, t=top, b=bottom, p=page
\centering
\includegraphics[width=0.9\textwidth]{vormessung/encorevor78790spec} 
% Bilddatei aus dem Unterverzeichnis bilder holen, skalieren auf 0.8*Satzspiegel
\caption {Encore mit 90p 787 Rot: Spektrum} 
\end{figure}

\begin{figure}[htp]     % h=here, t=top, b=bottom, p=page
\centering
\includegraphics[width=0.5\textwidth]{vormessung/encorevor78790cct} 
% Bilddatei aus dem Unterverzeichnis bilder holen, skalieren auf 0.8*Satzspiegel
\caption {Encore mit 90p 787 Rot: Beleuchtungsstärke, CCT, $\Delta$UV und Farbortkoordinaten} 
\end{figure}

\begin{figure}[htp]     % h=here, t=top, b=bottom, p=page
\centering
\includegraphics[width=0.9\textwidth]{vormessung/encorevor78790xy} 
% Bilddatei aus dem Unterverzeichnis bilder holen, skalieren auf 0.8*Satzspiegel
\caption {Encore mit 90p 787 Rot: XYZ-Farbraum Detailansicht} 
\end{figure}

\begin{figure}[htp]     % h=here, t=top, b=bottom, p=page
\centering
\includegraphics[width=0.9\textwidth]{vormessung/encorevor78790cri} 
% Bilddatei aus dem Unterverzeichnis bilder holen, skalieren auf 0.8*Satzspiegel
\caption {Encore mit 90p 787 Rot: CRI} 
\end{figure}

\begin{figure}[htp]     % h=here, t=top, b=bottom, p=page
\centering
\includegraphics[width=0.9\textwidth]{vormessung/encorevor78790cqs} 
% Bilddatei aus dem Unterverzeichnis bilder holen, skalieren auf 0.8*Satzspiegel
\caption {Encore mit 90p 787 Rot: CQS} 
\end{figure}

\begin{figure}[htp]     % h=here, t=top, b=bottom, p=page
\centering
\includegraphics[width=0.9\textwidth]{vormessung/encorevor78790tlci} 
% Bilddatei aus dem Unterverzeichnis bilder holen, skalieren auf 0.8*Satzspiegel
\caption {Encore mit 90p 787 Rot: TLCI} 
\end{figure}

\begin{figure}[htp]     % h=here, t=top, b=bottom, p=page
\centering
\includegraphics[width=0.9\textwidth]{vormessung/encorevor78790tm} 
% Bilddatei aus dem Unterverzeichnis bilder holen, skalieren auf 0.8*Satzspiegel
\caption {Encore mit 90p 787 Rot: TM-30} 
\end{figure}




\begin{figure}[htp]     % h=here, t=top, b=bottom, p=page
\centering
\includegraphics[width=0.9\textwidth]{vormessung/encorevor787100spec} 
% Bilddatei aus dem Unterverzeichnis bilder holen, skalieren auf 0.8*Satzspiegel
\caption {Encore mit 20p 787 Rot: Spektrum} 
\end{figure}

\begin{figure}[htp]     % h=here, t=top, b=bottom, p=page
\centering
\includegraphics[width=0.5\textwidth]{vormessung/encorevor78720cct} 
% Bilddatei aus dem Unterverzeichnis bilder holen, skalieren auf 0.8*Satzspiegel
\caption {Encore mit 100p 787 Rot: Beleuchtungsstärke, CCT, $\Delta$UV und Farbortkoordinaten} 
\end{figure}

\begin{figure}[htp]     % h=here, t=top, b=bottom, p=page
\centering
\includegraphics[width=0.9\textwidth]{vormessung/encorevor787100xy} 
% Bilddatei aus dem Unterverzeichnis bilder holen, skalieren auf 0.8*Satzspiegel
\caption {Encore mit 100p 787 Rot: XYZ-Farbraum Detailansicht} 
\end{figure}

\begin{figure}[htp]     % h=here, t=top, b=bottom, p=page
\centering
\includegraphics[width=0.9\textwidth]{vormessung/encorevor787100cri} 
% Bilddatei aus dem Unterverzeichnis bilder holen, skalieren auf 0.8*Satzspiegel
\caption {Encore mit 100p 787 Rot: CRI} 
\end{figure}

\begin{figure}[htp]     % h=here, t=top, b=bottom, p=page
\centering
\includegraphics[width=0.9\textwidth]{vormessung/encorevor787100cqs} 
% Bilddatei aus dem Unterverzeichnis bilder holen, skalieren auf 0.8*Satzspiegel
\caption {Encore mit 100p 787 Rot: CQS} 
\end{figure}

\begin{figure}[htp]     % h=here, t=top, b=bottom, p=page
\centering
\includegraphics[width=0.9\textwidth]{vormessung/encorevor787100tlci} 
% Bilddatei aus dem Unterverzeichnis bilder holen, skalieren auf 0.8*Satzspiegel
\caption {Encore mit 100p 787 Rot: TLCI} 
\end{figure}

\begin{figure}[htp]     % h=here, t=top, b=bottom, p=page
\centering
\includegraphics[width=0.9\textwidth]{vormessung/encorevor787100tm} 
% Bilddatei aus dem Unterverzeichnis bilder holen, skalieren auf 0.8*Satzspiegel
\caption {Encore mit 100p 787 Rot: TM-30} 
\end{figure}



\subsection{K-Eye}

\subsubsection{027 Rot in 10p Schritten}

\begin{figure}[htp]     % h=here, t=top, b=bottom, p=page
\centering
\includegraphics[width=0.9\textwidth]{vormessung/keyevor02710spec} 
% Bilddatei aus dem Unterverzeichnis bilder holen, skalieren auf 0.8*Satzspiegel
\caption {K-Eye mit 10p 027 Rot: Spektrum} 
\end{figure}

\begin{figure}[htp]     % h=here, t=top, b=bottom, p=page
\centering
\includegraphics[width=0.5\textwidth]{vormessung/keyevor02710cct} 
% Bilddatei aus dem Unterverzeichnis bilder holen, skalieren auf 0.8*Satzspiegel
\caption {K-Eye mit 10p 027 Rot: Beleuchtungsstärke, CCT, $\Delta$UV und Farbortkoordinaten} 
\end{figure}

\begin{figure}[htp]     % h=here, t=top, b=bottom, p=page
\centering
\includegraphics[width=0.9\textwidth]{vormessung/keyevor02710xy} 
% Bilddatei aus dem Unterverzeichnis bilder holen, skalieren auf 0.8*Satzspiegel
\caption {K-Eye mit 10p 027 Rot: XYZ-Farbraum Detailansicht} 
\end{figure}

\begin{figure}[htp]     % h=here, t=top, b=bottom, p=page
\centering
\includegraphics[width=0.9\textwidth]{vormessung/keyevor02710cri} 
% Bilddatei aus dem Unterverzeichnis bilder holen, skalieren auf 0.8*Satzspiegel
\caption {K-Eye mit 10p 027 Rot: CRI} 
\end{figure}

\begin{figure}[htp]     % h=here, t=top, b=bottom, p=page
\centering
\includegraphics[width=0.9\textwidth]{vormessung/keyevor02710cqs} 
% Bilddatei aus dem Unterverzeichnis bilder holen, skalieren auf 0.8*Satzspiegel
\caption {K-Eye mit 10p 027 Rot: CQS} 
\end{figure}

\begin{figure}[htp]     % h=here, t=top, b=bottom, p=page
\centering
\includegraphics[width=0.9\textwidth]{vormessung/keyevor02710tlci} 
% Bilddatei aus dem Unterverzeichnis bilder holen, skalieren auf 0.8*Satzspiegel
\caption {K-Eye mit 10p 027 Rot: TLCI} 
\end{figure}

\begin{figure}[htp]     % h=here, t=top, b=bottom, p=page
\centering
\includegraphics[width=0.9\textwidth]{vormessung/keyevor02710tm} 
% Bilddatei aus dem Unterverzeichnis bilder holen, skalieren auf 0.8*Satzspiegel
\caption {K-Eye mit 10p 027 Rot: TM-30} 
\end{figure}




\begin{figure}[htp]     % h=here, t=top, b=bottom, p=page
\centering
\includegraphics[width=0.9\textwidth]{vormessung/keyevor02720spec} 
% Bilddatei aus dem Unterverzeichnis bilder holen, skalieren auf 0.8*Satzspiegel
\caption {K-Eye mit 20p 027 Rot: Spektrum} 
\end{figure}

\begin{figure}[htp]     % h=here, t=top, b=bottom, p=page
\centering
\includegraphics[width=0.5\textwidth]{vormessung/keyevor02720cct} 
% Bilddatei aus dem Unterverzeichnis bilder holen, skalieren auf 0.8*Satzspiegel
\caption {K-Eye mit 20p 027 Rot: Beleuchtungsstärke, CCT, $\Delta$UV und Farbortkoordinaten} 
\end{figure}

\begin{figure}[htp]     % h=here, t=top, b=bottom, p=page
\centering
\includegraphics[width=0.9\textwidth]{vormessung/keyevor02720xy} 
% Bilddatei aus dem Unterverzeichnis bilder holen, skalieren auf 0.8*Satzspiegel
\caption {K-Eye mit 20p 027 Rot: XYZ-Farbraum Detailansicht} 
\end{figure}

\begin{figure}[htp]     % h=here, t=top, b=bottom, p=page
\centering
\includegraphics[width=0.9\textwidth]{vormessung/keyevor02720cri} 
% Bilddatei aus dem Unterverzeichnis bilder holen, skalieren auf 0.8*Satzspiegel
\caption {K-Eye mit 20p 027 Rot: CRI} 
\end{figure}

\begin{figure}[htp]     % h=here, t=top, b=bottom, p=page
\centering
\includegraphics[width=0.9\textwidth]{vormessung/keyevor02720cqs} 
% Bilddatei aus dem Unterverzeichnis bilder holen, skalieren auf 0.8*Satzspiegel
\caption {K-Eye mit 20p 027 Rot: CQS} 
\end{figure}

\begin{figure}[htp]     % h=here, t=top, b=bottom, p=page
\centering
\includegraphics[width=0.9\textwidth]{vormessung/keyevor02720tlci} 
% Bilddatei aus dem Unterverzeichnis bilder holen, skalieren auf 0.8*Satzspiegel
\caption {K-Eye mit 20p 027 Rot: TLCI} 
\end{figure}

\begin{figure}[htp]     % h=here, t=top, b=bottom, p=page
\centering
\includegraphics[width=0.9\textwidth]{vormessung/keyevor02720tm} 
% Bilddatei aus dem Unterverzeichnis bilder holen, skalieren auf 0.8*Satzspiegel
\caption {K-Eye mit 20p 027 Rot: TM-30} 
\end{figure}



\begin{figure}[htp]     % h=here, t=top, b=bottom, p=page
\centering
\includegraphics[width=0.9\textwidth]{vormessung/keyevor02730spec} 
% Bilddatei aus dem Unterverzeichnis bilder holen, skalieren auf 0.8*Satzspiegel
\caption {K-Eye mit 30p 027 Rot: Spektrum} 
\end{figure}

\begin{figure}[htp]     % h=here, t=top, b=bottom, p=page
\centering
\includegraphics[width=0.5\textwidth]{vormessung/keyevor02730cct} 
% Bilddatei aus dem Unterverzeichnis bilder holen, skalieren auf 0.8*Satzspiegel
\caption {K-Eye mit 30p 027 Rot: Beleuchtungsstärke, CCT, $\Delta$UV und Farbortkoordinaten} 
\end{figure}

\begin{figure}[htp]     % h=here, t=top, b=bottom, p=page
\centering
\includegraphics[width=0.9\textwidth]{vormessung/keyevor02730xy} 
% Bilddatei aus dem Unterverzeichnis bilder holen, skalieren auf 0.8*Satzspiegel
\caption {K-Eye mit 30p 027 Rot: XYZ-Farbraum Detailansicht} 
\end{figure}

\begin{figure}[htp]     % h=here, t=top, b=bottom, p=page
\centering
\includegraphics[width=0.9\textwidth]{vormessung/keyevor02730cri} 
% Bilddatei aus dem Unterverzeichnis bilder holen, skalieren auf 0.8*Satzspiegel
\caption {K-Eye mit 30p 027 Rot: CRI} 
\end{figure}

\begin{figure}[htp]     % h=here, t=top, b=bottom, p=page
\centering
\includegraphics[width=0.9\textwidth]{vormessung/keyevor02730cqs} 
% Bilddatei aus dem Unterverzeichnis bilder holen, skalieren auf 0.8*Satzspiegel
\caption {K-Eye mit 30p 027 Rot: CQS} 
\end{figure}

\begin{figure}[htp]     % h=here, t=top, b=bottom, p=page
\centering
\includegraphics[width=0.9\textwidth]{vormessung/keyevor02730tlci} 
% Bilddatei aus dem Unterverzeichnis bilder holen, skalieren auf 0.8*Satzspiegel
\caption {K-Eye mit 30p 027 Rot: TLCI} 
\end{figure}

\begin{figure}[htp]     % h=here, t=top, b=bottom, p=page
\centering
\includegraphics[width=0.9\textwidth]{vormessung/keyevor02730tm} 
% Bilddatei aus dem Unterverzeichnis bilder holen, skalieren auf 0.8*Satzspiegel
\caption {K-Eye mit 30p 027 Rot: TM-30} 
\end{figure}



\begin{figure}[htp]     % h=here, t=top, b=bottom, p=page
\centering
\includegraphics[width=0.9\textwidth]{vormessung/keyevor02740spec} 
% Bilddatei aus dem Unterverzeichnis bilder holen, skalieren auf 0.8*Satzspiegel
\caption {K-Eye mit 40p 027 Rot: Spektrum} 
\end{figure}

\begin{figure}[htp]     % h=here, t=top, b=bottom, p=page
\centering
\includegraphics[width=0.5\textwidth]{vormessung/keyevor02740cct} 
% Bilddatei aus dem Unterverzeichnis bilder holen, skalieren auf 0.8*Satzspiegel
\caption {K-Eye mit 40p 027 Rot: Beleuchtungsstärke, CCT, $\Delta$UV und Farbortkoordinaten} 
\end{figure}

\begin{figure}[htp]     % h=here, t=top, b=bottom, p=page
\centering
\includegraphics[width=0.9\textwidth]{vormessung/keyevor02740xy} 
% Bilddatei aus dem Unterverzeichnis bilder holen, skalieren auf 0.8*Satzspiegel
\caption {K-Eye mit 40p 027 Rot: XYZ-Farbraum Detailansicht} 
\end{figure}

\begin{figure}[htp]     % h=here, t=top, b=bottom, p=page
\centering
\includegraphics[width=0.9\textwidth]{vormessung/keyevor02740cri} 
% Bilddatei aus dem Unterverzeichnis bilder holen, skalieren auf 0.8*Satzspiegel
\caption {K-Eye mit 40p 027 Rot: CRI} 
\end{figure}

\begin{figure}[htp]     % h=here, t=top, b=bottom, p=page
\centering
\includegraphics[width=0.9\textwidth]{vormessung/keyevor02740cqs} 
% Bilddatei aus dem Unterverzeichnis bilder holen, skalieren auf 0.8*Satzspiegel
\caption {K-Eye mit 40p 027 Rot: CQS} 
\end{figure}

\begin{figure}[htp]     % h=here, t=top, b=bottom, p=page
\centering
\includegraphics[width=0.9\textwidth]{vormessung/keyevor02740tlci} 
% Bilddatei aus dem Unterverzeichnis bilder holen, skalieren auf 0.8*Satzspiegel
\caption {K-Eye mit 40p 027 Rot: TLCI} 
\end{figure}

\begin{figure}[htp]     % h=here, t=top, b=bottom, p=page
\centering
\includegraphics[width=0.9\textwidth]{vormessung/keyevor02740tm} 
% Bilddatei aus dem Unterverzeichnis bilder holen, skalieren auf 0.8*Satzspiegel
\caption {K-Eye mit 40p 027 Rot: TM-30} 
\end{figure}




\begin{figure}[htp]     % h=here, t=top, b=bottom, p=page
\centering
\includegraphics[width=0.9\textwidth]{vormessung/keyevor02750spec} 
% Bilddatei aus dem Unterverzeichnis bilder holen, skalieren auf 0.8*Satzspiegel
\caption {K-Eye mit 50p 027 Rot: Spektrum} 
\end{figure}

\begin{figure}[htp]     % h=here, t=top, b=bottom, p=page
\centering
\includegraphics[width=0.5\textwidth]{vormessung/keyevor02750cct} 
% Bilddatei aus dem Unterverzeichnis bilder holen, skalieren auf 0.8*Satzspiegel
\caption {K-Eye mit 50p 027 Rot: Beleuchtungsstärke, CCT, $\Delta$UV und Farbortkoordinaten} 
\end{figure}

\begin{figure}[htp]     % h=here, t=top, b=bottom, p=page
\centering
\includegraphics[width=0.9\textwidth]{vormessung/keyevor02750xy} 
% Bilddatei aus dem Unterverzeichnis bilder holen, skalieren auf 0.8*Satzspiegel
\caption {K-Eye mit 50p 027 Rot: XYZ-Farbraum Detailansicht} 
\end{figure}

\begin{figure}[htp]     % h=here, t=top, b=bottom, p=page
\centering
\includegraphics[width=0.9\textwidth]{vormessung/keyevor02750cri} 
% Bilddatei aus dem Unterverzeichnis bilder holen, skalieren auf 0.8*Satzspiegel
\caption {K-Eye mit 50p 027 Rot: CRI} 
\end{figure}

\begin{figure}[htp]     % h=here, t=top, b=bottom, p=page
\centering
\includegraphics[width=0.9\textwidth]{vormessung/keyevor02750cqs} 
% Bilddatei aus dem Unterverzeichnis bilder holen, skalieren auf 0.8*Satzspiegel
\caption {K-Eye mit 50p 027 Rot: CQS} 
\end{figure}

\begin{figure}[htp]     % h=here, t=top, b=bottom, p=page
\centering
\includegraphics[width=0.9\textwidth]{vormessung/keyevor02750tlci} 
% Bilddatei aus dem Unterverzeichnis bilder holen, skalieren auf 0.8*Satzspiegel
\caption {K-Eye mit 50p 027 Rot: TLCI} 
\end{figure}

\begin{figure}[htp]     % h=here, t=top, b=bottom, p=page
\centering
\includegraphics[width=0.9\textwidth]{vormessung/keyevor02750tm} 
% Bilddatei aus dem Unterverzeichnis bilder holen, skalieren auf 0.8*Satzspiegel
\caption {K-Eye mit 50p 027 Rot: TM-30} 
\end{figure}




\begin{figure}[htp]     % h=here, t=top, b=bottom, p=page
\centering
\includegraphics[width=0.9\textwidth]{vormessung/keyevor02760spec} 
% Bilddatei aus dem Unterverzeichnis bilder holen, skalieren auf 0.8*Satzspiegel
\caption {K-Eye mit 60p 027 Rot: Spektrum} 
\end{figure}

\begin{figure}[htp]     % h=here, t=top, b=bottom, p=page
\centering
\includegraphics[width=0.5\textwidth]{vormessung/keyevor02760cct} 
% Bilddatei aus dem Unterverzeichnis bilder holen, skalieren auf 0.8*Satzspiegel
\caption {K-Eye mit 60p 027 Rot: Beleuchtungsstärke, CCT, $\Delta$UV und Farbortkoordinaten} 
\end{figure}

\begin{figure}[htp]     % h=here, t=top, b=bottom, p=page
\centering
\includegraphics[width=0.9\textwidth]{vormessung/keyevor02760xy} 
% Bilddatei aus dem Unterverzeichnis bilder holen, skalieren auf 0.8*Satzspiegel
\caption {K-Eye mit 60p 027 Rot: XYZ-Farbraum Detailansicht} 
\end{figure}

\begin{figure}[htp]     % h=here, t=top, b=bottom, p=page
\centering
\includegraphics[width=0.9\textwidth]{vormessung/keyevor02760cri} 
% Bilddatei aus dem Unterverzeichnis bilder holen, skalieren auf 0.8*Satzspiegel
\caption {K-Eye mit 60p 027 Rot: CRI} 
\end{figure}

\begin{figure}[htp]     % h=here, t=top, b=bottom, p=page
\centering
\includegraphics[width=0.9\textwidth]{vormessung/keyevor02760cqs} 
% Bilddatei aus dem Unterverzeichnis bilder holen, skalieren auf 0.8*Satzspiegel
\caption {K-Eye mit 60p 027 Rot: CQS} 
\end{figure}

\begin{figure}[htp]     % h=here, t=top, b=bottom, p=page
\centering
\includegraphics[width=0.9\textwidth]{vormessung/keyevor02760tlci} 
% Bilddatei aus dem Unterverzeichnis bilder holen, skalieren auf 0.8*Satzspiegel
\caption {K-Eye mit 60p 027 Rot: TLCI} 
\end{figure}

\begin{figure}[htp]     % h=here, t=top, b=bottom, p=page
\centering
\includegraphics[width=0.9\textwidth]{vormessung/keyevor02760tm} 
% Bilddatei aus dem Unterverzeichnis bilder holen, skalieren auf 0.8*Satzspiegel
\caption {K-Eye mit 60p 027 Rot: TM-30} 
\end{figure}




\begin{figure}[htp]     % h=here, t=top, b=bottom, p=page
\centering
\includegraphics[width=0.9\textwidth]{vormessung/keyevor02770spec} 
% Bilddatei aus dem Unterverzeichnis bilder holen, skalieren auf 0.8*Satzspiegel
\caption {K-Eye mit 70p 027 Rot: Spektrum} 
\end{figure}

\begin{figure}[htp]     % h=here, t=top, b=bottom, p=page
\centering
\includegraphics[width=0.5\textwidth]{vormessung/keyevor02770cct} 
% Bilddatei aus dem Unterverzeichnis bilder holen, skalieren auf 0.8*Satzspiegel
\caption {K-Eye mit 70p 027 Rot: Beleuchtungsstärke, CCT, $\Delta$UV und Farbortkoordinaten} 
\end{figure}

\begin{figure}[htp]     % h=here, t=top, b=bottom, p=page
\centering
\includegraphics[width=0.9\textwidth]{vormessung/keyevor02770xy} 
% Bilddatei aus dem Unterverzeichnis bilder holen, skalieren auf 0.8*Satzspiegel
\caption {K-Eye mit 70p 027 Rot: XYZ-Farbraum Detailansicht} 
\end{figure}

\begin{figure}[htp]     % h=here, t=top, b=bottom, p=page
\centering
\includegraphics[width=0.9\textwidth]{vormessung/keyevor02770cri} 
% Bilddatei aus dem Unterverzeichnis bilder holen, skalieren auf 0.8*Satzspiegel
\caption {K-Eye mit 70p 027 Rot: CRI} 
\end{figure}

\begin{figure}[htp]     % h=here, t=top, b=bottom, p=page
\centering
\includegraphics[width=0.9\textwidth]{vormessung/keyevor02770cqs} 
% Bilddatei aus dem Unterverzeichnis bilder holen, skalieren auf 0.8*Satzspiegel
\caption {K-Eye mit 70p 027 Rot: CQS} 
\end{figure}

\begin{figure}[htp]     % h=here, t=top, b=bottom, p=page
\centering
\includegraphics[width=0.9\textwidth]{vormessung/keyevor02770tlci} 
% Bilddatei aus dem Unterverzeichnis bilder holen, skalieren auf 0.8*Satzspiegel
\caption {K-Eye mit 70p 027 Rot: TLCI} 
\end{figure}

\begin{figure}[htp]     % h=here, t=top, b=bottom, p=page
\centering
\includegraphics[width=0.9\textwidth]{vormessung/keyevor02770tm} 
% Bilddatei aus dem Unterverzeichnis bilder holen, skalieren auf 0.8*Satzspiegel
\caption {K-Eye mit 70p 027 Rot: TM-30} 
\end{figure}




\begin{figure}[htp]     % h=here, t=top, b=bottom, p=page
\centering
\includegraphics[width=0.9\textwidth]{vormessung/keyevor02780spec} 
% Bilddatei aus dem Unterverzeichnis bilder holen, skalieren auf 0.8*Satzspiegel
\caption {K-Eye mit 80p 027 Rot: Spektrum} 
\end{figure}

\begin{figure}[htp]     % h=here, t=top, b=bottom, p=page
\centering
\includegraphics[width=0.5\textwidth]{vormessung/keyevor02780cct} 
% Bilddatei aus dem Unterverzeichnis bilder holen, skalieren auf 0.8*Satzspiegel
\caption {K-Eye mit 80p 027 Rot: Beleuchtungsstärke, CCT, $\Delta$UV und Farbortkoordinaten} 
\end{figure}

\begin{figure}[htp]     % h=here, t=top, b=bottom, p=page
\centering
\includegraphics[width=0.9\textwidth]{vormessung/keyevor02780xy} 
% Bilddatei aus dem Unterverzeichnis bilder holen, skalieren auf 0.8*Satzspiegel
\caption {K-Eye mit 80p 027 Rot: XYZ-Farbraum Detailansicht} 
\end{figure}

\begin{figure}[htp]     % h=here, t=top, b=bottom, p=page
\centering
\includegraphics[width=0.9\textwidth]{vormessung/keyevor02780cri} 
% Bilddatei aus dem Unterverzeichnis bilder holen, skalieren auf 0.8*Satzspiegel
\caption {K-Eye mit 80p 027 Rot: CRI} 
\end{figure}

\begin{figure}[htp]     % h=here, t=top, b=bottom, p=page
\centering
\includegraphics[width=0.9\textwidth]{vormessung/keyevor02780cqs} 
% Bilddatei aus dem Unterverzeichnis bilder holen, skalieren auf 0.8*Satzspiegel
\caption {K-Eye mit 80p 027 Rot: CQS} 
\end{figure}

\begin{figure}[htp]     % h=here, t=top, b=bottom, p=page
\centering
\includegraphics[width=0.9\textwidth]{vormessung/keyevor02780tlci} 
% Bilddatei aus dem Unterverzeichnis bilder holen, skalieren auf 0.8*Satzspiegel
\caption {K-Eye mit 80p 027 Rot: TLCI} 
\end{figure}

\begin{figure}[htp]     % h=here, t=top, b=bottom, p=page
\centering
\includegraphics[width=0.9\textwidth]{vormessung/keyevor02780tm} 
% Bilddatei aus dem Unterverzeichnis bilder holen, skalieren auf 0.8*Satzspiegel
\caption {K-Eye mit 80p 027 Rot: TM-30} 
\end{figure}




\begin{figure}[htp]     % h=here, t=top, b=bottom, p=page
\centering
\includegraphics[width=0.9\textwidth]{vormessung/keyevor02790spec} 
% Bilddatei aus dem Unterverzeichnis bilder holen, skalieren auf 0.8*Satzspiegel
\caption {K-Eye mit 90p 027 Rot: Spektrum} 
\end{figure}

\begin{figure}[htp]     % h=here, t=top, b=bottom, p=page
\centering
\includegraphics[width=0.5\textwidth]{vormessung/keyevor02790cct} 
% Bilddatei aus dem Unterverzeichnis bilder holen, skalieren auf 0.8*Satzspiegel
\caption {K-Eye mit 90p 027 Rot: Beleuchtungsstärke, CCT, $\Delta$UV und Farbortkoordinaten} 
\end{figure}

\begin{figure}[htp]     % h=here, t=top, b=bottom, p=page
\centering
\includegraphics[width=0.9\textwidth]{vormessung/keyevor02790xy} 
% Bilddatei aus dem Unterverzeichnis bilder holen, skalieren auf 0.8*Satzspiegel
\caption {K-Eye mit 90p 027 Rot: XYZ-Farbraum Detailansicht} 
\end{figure}

\begin{figure}[htp]     % h=here, t=top, b=bottom, p=page
\centering
\includegraphics[width=0.9\textwidth]{vormessung/keyevor02790cri} 
% Bilddatei aus dem Unterverzeichnis bilder holen, skalieren auf 0.8*Satzspiegel
\caption {K-Eye mit 90p 027 Rot: CRI} 
\end{figure}

\begin{figure}[htp]     % h=here, t=top, b=bottom, p=page
\centering
\includegraphics[width=0.9\textwidth]{vormessung/keyevor02790cqs} 
% Bilddatei aus dem Unterverzeichnis bilder holen, skalieren auf 0.8*Satzspiegel
\caption {K-Eye mit 90p 027 Rot: CQS} 
\end{figure}

\begin{figure}[htp]     % h=here, t=top, b=bottom, p=page
\centering
\includegraphics[width=0.9\textwidth]{vormessung/keyevor02790tlci} 
% Bilddatei aus dem Unterverzeichnis bilder holen, skalieren auf 0.8*Satzspiegel
\caption {K-Eye mit 90p 027 Rot: TLCI} 
\end{figure}

\begin{figure}[htp]     % h=here, t=top, b=bottom, p=page
\centering
\includegraphics[width=0.9\textwidth]{vormessung/keyevor02790tm} 
% Bilddatei aus dem Unterverzeichnis bilder holen, skalieren auf 0.8*Satzspiegel
\caption {K-Eye mit 90p 027 Rot: TM-30} 
\end{figure}




\begin{figure}[htp]     % h=here, t=top, b=bottom, p=page
\centering
\includegraphics[width=0.9\textwidth]{vormessung/keyevor027100spec} 
% Bilddatei aus dem Unterverzeichnis bilder holen, skalieren auf 0.8*Satzspiegel
\caption {K-Eye mit 100p 027 Rot: Spektrum} 
\end{figure}

\begin{figure}[htp]     % h=here, t=top, b=bottom, p=page
\centering
\includegraphics[width=0.5\textwidth]{vormessung/keyevor027100cct} 
% Bilddatei aus dem Unterverzeichnis bilder holen, skalieren auf 0.8*Satzspiegel
\caption {K-Eye mit 100p 027 Rot: Beleuchtungsstärke, CCT, $\Delta$UV und Farbortkoordinaten} 
\end{figure}

\begin{figure}[htp]     % h=here, t=top, b=bottom, p=page
\centering
\includegraphics[width=0.9\textwidth]{vormessung/keyevor027100xy} 
% Bilddatei aus dem Unterverzeichnis bilder holen, skalieren auf 0.8*Satzspiegel
\caption {K-Eye mit 100p 027 Rot: XYZ-Farbraum Detailansicht} 
\end{figure}

\begin{figure}[htp]     % h=here, t=top, b=bottom, p=page
\centering
\includegraphics[width=0.9\textwidth]{vormessung/keyevor027100cri} 
% Bilddatei aus dem Unterverzeichnis bilder holen, skalieren auf 0.8*Satzspiegel
\caption {K-Eye mit 100p 027 Rot: CRI} 
\end{figure}

\begin{figure}[htp]     % h=here, t=top, b=bottom, p=page
\centering
\includegraphics[width=0.9\textwidth]{vormessung/keyevor027100cqs} 
% Bilddatei aus dem Unterverzeichnis bilder holen, skalieren auf 0.8*Satzspiegel
\caption {K-Eye mit 100p 027 Rot: CQS} 
\end{figure}

\begin{figure}[htp]     % h=here, t=top, b=bottom, p=page
\centering
\includegraphics[width=0.9\textwidth]{vormessung/keyevor027100tlci} 
% Bilddatei aus dem Unterverzeichnis bilder holen, skalieren auf 0.8*Satzspiegel
\caption {K-Eye mit 100p 027 Rot: TLCI} 
\end{figure}

\begin{figure}[htp]     % h=here, t=top, b=bottom, p=page
\centering
\includegraphics[width=0.9\textwidth]{vormessung/keyevor027100tm} 
% Bilddatei aus dem Unterverzeichnis bilder holen, skalieren auf 0.8*Satzspiegel
\caption {K-Eye mit 100p 027 Rot: TM-30} 
\end{figure}


\subsubsection{787 Rot in 10p Schritten}

\begin{figure}[htp]     % h=here, t=top, b=bottom, p=page
\centering
\includegraphics[width=0.9\textwidth]{vormessung/keyevor78710spec} 
% Bilddatei aus dem Unterverzeichnis bilder holen, skalieren auf 0.8*Satzspiegel
\caption {K-Eye mit 10p 787 Rot: Spektrum} 
\end{figure}

\begin{figure}[htp]     % h=here, t=top, b=bottom, p=page
\centering
\includegraphics[width=0.5\textwidth]{vormessung/keyevor78710cct} 
% Bilddatei aus dem Unterverzeichnis bilder holen, skalieren auf 0.8*Satzspiegel
\caption {K-Eye mit 10p 787 Rot: Beleuchtungsstärke, CCT, $\Delta$UV und Farbortkoordinaten} 
\end{figure}

\begin{figure}[htp]     % h=here, t=top, b=bottom, p=page
\centering
\includegraphics[width=0.9\textwidth]{vormessung/keyevor07870xy} 
% Bilddatei aus dem Unterverzeichnis bilder holen, skalieren auf 0.8*Satzspiegel
\caption {K-Eye mit 10p 787 Rot: XYZ-Farbraum Detailansicht} 
\end{figure}

\begin{figure}[htp]     % h=here, t=top, b=bottom, p=page
\centering
\includegraphics[width=0.9\textwidth]{vormessung/keyevor78710cri} 
% Bilddatei aus dem Unterverzeichnis bilder holen, skalieren auf 0.8*Satzspiegel
\caption {K-Eye mit 10p 787 Rot: CRI} 
\end{figure}

\begin{figure}[htp]     % h=here, t=top, b=bottom, p=page
\centering
\includegraphics[width=0.9\textwidth]{vormessung/keyevor78710cqs} 
% Bilddatei aus dem Unterverzeichnis bilder holen, skalieren auf 0.8*Satzspiegel
\caption {K-Eye mit 10p 787 Rot: CQS} 
\end{figure}

\begin{figure}[htp]     % h=here, t=top, b=bottom, p=page
\centering
\includegraphics[width=0.9\textwidth]{vormessung/keyevor78710tlci} 
% Bilddatei aus dem Unterverzeichnis bilder holen, skalieren auf 0.8*Satzspiegel
\caption {K-Eye mit 10p 787 Rot: TLCI} 
\end{figure}

\begin{figure}[htp]     % h=here, t=top, b=bottom, p=page
\centering
\includegraphics[width=0.9\textwidth]{vormessung/keyevor78710tm} 
% Bilddatei aus dem Unterverzeichnis bilder holen, skalieren auf 0.8*Satzspiegel
\caption {K-Eye mit 10p 787 Rot: TM-30} 
\end{figure}




\begin{figure}[htp]     % h=here, t=top, b=bottom, p=page
\centering
\includegraphics[width=0.9\textwidth]{vormessung/keyevor78720spec} 
% Bilddatei aus dem Unterverzeichnis bilder holen, skalieren auf 0.8*Satzspiegel
\caption {K-Eye mit 20p 787 Rot: Spektrum} 
\end{figure}

\begin{figure}[htp]     % h=here, t=top, b=bottom, p=page
\centering
\includegraphics[width=0.5\textwidth]{vormessung/keyevor78720cct} 
% Bilddatei aus dem Unterverzeichnis bilder holen, skalieren auf 0.8*Satzspiegel
\caption {K-Eye mit 20p 027 Rot: Beleuchtungsstärke, CCT, $\Delta$UV und Farbortkoordinaten} 
\end{figure}

\begin{figure}[htp]     % h=here, t=top, b=bottom, p=page
\centering
\includegraphics[width=0.9\textwidth]{vormessung/keyevor02720xy} 
% Bilddatei aus dem Unterverzeichnis bilder holen, skalieren auf 0.8*Satzspiegel
\caption {K-Eye mit 20p 787 Rot: XYZ-Farbraum Detailansicht} 
\end{figure}

\begin{figure}[htp]     % h=here, t=top, b=bottom, p=page
\centering
\includegraphics[width=0.9\textwidth]{vormessung/keyevor78720cri} 
% Bilddatei aus dem Unterverzeichnis bilder holen, skalieren auf 0.8*Satzspiegel
\caption {K-Eye mit 20p 787 Rot: CRI} 
\end{figure}

\begin{figure}[htp]     % h=here, t=top, b=bottom, p=page
\centering
\includegraphics[width=0.9\textwidth]{vormessung/keyevor78720cqs} 
% Bilddatei aus dem Unterverzeichnis bilder holen, skalieren auf 0.8*Satzspiegel
\caption {K-Eye mit 20p 787 Rot: CQS} 
\end{figure}

\begin{figure}[htp]     % h=here, t=top, b=bottom, p=page
\centering
\includegraphics[width=0.9\textwidth]{vormessung/keyevor78720tlci} 
% Bilddatei aus dem Unterverzeichnis bilder holen, skalieren auf 0.8*Satzspiegel
\caption {K-Eye mit 20p 787 Rot: TLCI} 
\end{figure}

\begin{figure}[htp]     % h=here, t=top, b=bottom, p=page
\centering
\includegraphics[width=0.9\textwidth]{vormessung/keyevor78720tm} 
% Bilddatei aus dem Unterverzeichnis bilder holen, skalieren auf 0.8*Satzspiegel
\caption {K-Eye mit 20p 787 Rot: TM-30} 
\end{figure}



\begin{figure}[htp]     % h=here, t=top, b=bottom, p=page
\centering
\includegraphics[width=0.9\textwidth]{vormessung/keyevor78730spec} 
% Bilddatei aus dem Unterverzeichnis bilder holen, skalieren auf 0.8*Satzspiegel
\caption {K-Eye mit 30p 787 Rot: Spektrum} 
\end{figure}

\begin{figure}[htp]     % h=here, t=top, b=bottom, p=page
\centering
\includegraphics[width=0.5\textwidth]{vormessung/keyevor78730cct} 
% Bilddatei aus dem Unterverzeichnis bilder holen, skalieren auf 0.8*Satzspiegel
\caption {K-Eye mit 30p 787 Rot: Beleuchtungsstärke, CCT, $\Delta$UV und Farbortkoordinaten} 
\end{figure}

\begin{figure}[htp]     % h=here, t=top, b=bottom, p=page
\centering
\includegraphics[width=0.9\textwidth]{vormessung/keyevor78730xy} 
% Bilddatei aus dem Unterverzeichnis bilder holen, skalieren auf 0.8*Satzspiegel
\caption {K-Eye mit 30p 787 Rot: XYZ-Farbraum Detailansicht} 
\end{figure}

\begin{figure}[htp]     % h=here, t=top, b=bottom, p=page
\centering
\includegraphics[width=0.9\textwidth]{vormessung/keyevor78730cri} 
% Bilddatei aus dem Unterverzeichnis bilder holen, skalieren auf 0.8*Satzspiegel
\caption {K-Eye mit 30p 787 Rot: CRI} 
\end{figure}

\begin{figure}[htp]     % h=here, t=top, b=bottom, p=page
\centering
\includegraphics[width=0.9\textwidth]{vormessung/keyevor78730cqs} 
% Bilddatei aus dem Unterverzeichnis bilder holen, skalieren auf 0.8*Satzspiegel
\caption {K-Eye mit 30p 787 Rot: CQS} 
\end{figure}

\begin{figure}[htp]     % h=here, t=top, b=bottom, p=page
\centering
\includegraphics[width=0.9\textwidth]{vormessung/keyevor78730tlci} 
% Bilddatei aus dem Unterverzeichnis bilder holen, skalieren auf 0.8*Satzspiegel
\caption {K-Eye mit 30p 787 Rot: TLCI} 
\end{figure}

\begin{figure}[htp]     % h=here, t=top, b=bottom, p=page
\centering
\includegraphics[width=0.9\textwidth]{vormessung/keyevor78730tm} 
% Bilddatei aus dem Unterverzeichnis bilder holen, skalieren auf 0.8*Satzspiegel
\caption {K-Eye mit 30p 787 Rot: TM-30} 
\end{figure}



\begin{figure}[htp]     % h=here, t=top, b=bottom, p=page
\centering
\includegraphics[width=0.9\textwidth]{vormessung/keyevor78740spec} 
% Bilddatei aus dem Unterverzeichnis bilder holen, skalieren auf 0.8*Satzspiegel
\caption {K-Eye mit 40p 787 Rot: Spektrum} 
\end{figure}

\begin{figure}[htp]     % h=here, t=top, b=bottom, p=page
\centering
\includegraphics[width=0.5\textwidth]{vormessung/keyevor78740cct} 
% Bilddatei aus dem Unterverzeichnis bilder holen, skalieren auf 0.8*Satzspiegel
\caption {K-Eye mit 40p 787 Rot: Beleuchtungsstärke, CCT, $\Delta$UV und Farbortkoordinaten} 
\end{figure}

\begin{figure}[htp]     % h=here, t=top, b=bottom, p=page
\centering
\includegraphics[width=0.9\textwidth]{vormessung/keyevor78740xy} 
% Bilddatei aus dem Unterverzeichnis bilder holen, skalieren auf 0.8*Satzspiegel
\caption {K-Eye mit 40p 787 Rot: XYZ-Farbraum Detailansicht} 
\end{figure}

\begin{figure}[htp]     % h=here, t=top, b=bottom, p=page
\centering
\includegraphics[width=0.9\textwidth]{vormessung/keyevor78740cri} 
% Bilddatei aus dem Unterverzeichnis bilder holen, skalieren auf 0.8*Satzspiegel
\caption {K-Eye mit 40p 787 Rot: CRI} 
\end{figure}

\begin{figure}[htp]     % h=here, t=top, b=bottom, p=page
\centering
\includegraphics[width=0.9\textwidth]{vormessung/keyevor78740cqs} 
% Bilddatei aus dem Unterverzeichnis bilder holen, skalieren auf 0.8*Satzspiegel
\caption {K-Eye mit 40p 787 Rot: CQS} 
\end{figure}

\begin{figure}[htp]     % h=here, t=top, b=bottom, p=page
\centering
\includegraphics[width=0.9\textwidth]{vormessung/keyevor78740tlci} 
% Bilddatei aus dem Unterverzeichnis bilder holen, skalieren auf 0.8*Satzspiegel
\caption {K-Eye mit 40p 787 Rot: TLCI} 
\end{figure}

\begin{figure}[htp]     % h=here, t=top, b=bottom, p=page
\centering
\includegraphics[width=0.9\textwidth]{vormessung/keyevor78740tm} 
% Bilddatei aus dem Unterverzeichnis bilder holen, skalieren auf 0.8*Satzspiegel
\caption {K-Eye mit 40p 787 Rot: TM-30} 
\end{figure}




\begin{figure}[htp]     % h=here, t=top, b=bottom, p=page
\centering
\includegraphics[width=0.9\textwidth]{vormessung/keyevor78750spec} 
% Bilddatei aus dem Unterverzeichnis bilder holen, skalieren auf 0.8*Satzspiegel
\caption {K-Eye mit 50p 787 Rot: Spektrum} 
\end{figure}

\begin{figure}[htp]     % h=here, t=top, b=bottom, p=page
\centering
\includegraphics[width=0.5\textwidth]{vormessung/keyevor78750cct} 
% Bilddatei aus dem Unterverzeichnis bilder holen, skalieren auf 0.8*Satzspiegel
\caption {K-Eye mit 50p 787 Rot: Beleuchtungsstärke, CCT, $\Delta$UV und Farbortkoordinaten} 
\end{figure}

\begin{figure}[htp]     % h=here, t=top, b=bottom, p=page
\centering
\includegraphics[width=0.9\textwidth]{vormessung/keyevor78750xy} 
% Bilddatei aus dem Unterverzeichnis bilder holen, skalieren auf 0.8*Satzspiegel
\caption {K-Eye mit 50p 787 Rot: XYZ-Farbraum Detailansicht} 
\end{figure}

\begin{figure}[htp]     % h=here, t=top, b=bottom, p=page
\centering
\includegraphics[width=0.9\textwidth]{vormessung/keyevor78750cri} 
% Bilddatei aus dem Unterverzeichnis bilder holen, skalieren auf 0.8*Satzspiegel
\caption {K-Eye mit 50p 787 Rot: CRI} 
\end{figure}

\begin{figure}[htp]     % h=here, t=top, b=bottom, p=page
\centering
\includegraphics[width=0.9\textwidth]{vormessung/keyevor78750cqs} 
% Bilddatei aus dem Unterverzeichnis bilder holen, skalieren auf 0.8*Satzspiegel
\caption {K-Eye mit 50p 787 Rot: CQS} 
\end{figure}

\begin{figure}[htp]     % h=here, t=top, b=bottom, p=page
\centering
\includegraphics[width=0.9\textwidth]{vormessung/keyevor78750tlci} 
% Bilddatei aus dem Unterverzeichnis bilder holen, skalieren auf 0.8*Satzspiegel
\caption {K-Eye mit 50p 787 Rot: TLCI} 
\end{figure}

\begin{figure}[htp]     % h=here, t=top, b=bottom, p=page
\centering
\includegraphics[width=0.9\textwidth]{vormessung/keyevor78750tm} 
% Bilddatei aus dem Unterverzeichnis bilder holen, skalieren auf 0.8*Satzspiegel
\caption {K-Eye mit 50p 787 Rot: TM-30} 
\end{figure}




\begin{figure}[htp]     % h=here, t=top, b=bottom, p=page
\centering
\includegraphics[width=0.9\textwidth]{vormessung/keyevor78760spec} 
% Bilddatei aus dem Unterverzeichnis bilder holen, skalieren auf 0.8*Satzspiegel
\caption {K-Eye mit 60p 787 Rot: Spektrum} 
\end{figure}

\begin{figure}[htp]     % h=here, t=top, b=bottom, p=page
\centering
\includegraphics[width=0.5\textwidth]{vormessung/keyevor78760cct} 
% Bilddatei aus dem Unterverzeichnis bilder holen, skalieren auf 0.8*Satzspiegel
\caption {K-Eye mit 60p 787 Rot: Beleuchtungsstärke, CCT, $\Delta$UV und Farbortkoordinaten} 
\end{figure}

\begin{figure}[htp]     % h=here, t=top, b=bottom, p=page
\centering
\includegraphics[width=0.9\textwidth]{vormessung/keyevor78760xy} 
% Bilddatei aus dem Unterverzeichnis bilder holen, skalieren auf 0.8*Satzspiegel
\caption {K-Eye mit 60p 787 Rot: XYZ-Farbraum Detailansicht} 
\end{figure}

\begin{figure}[htp]     % h=here, t=top, b=bottom, p=page
\centering
\includegraphics[width=0.9\textwidth]{vormessung/keyevor78760cri} 
% Bilddatei aus dem Unterverzeichnis bilder holen, skalieren auf 0.8*Satzspiegel
\caption {K-Eye mit 60p 787 Rot: CRI} 
\end{figure}

\begin{figure}[htp]     % h=here, t=top, b=bottom, p=page
\centering
\includegraphics[width=0.9\textwidth]{vormessung/keyevor78760cqs} 
% Bilddatei aus dem Unterverzeichnis bilder holen, skalieren auf 0.8*Satzspiegel
\caption {K-Eye mit 60p 787 Rot: CQS} 
\end{figure}

\begin{figure}[htp]     % h=here, t=top, b=bottom, p=page
\centering
\includegraphics[width=0.9\textwidth]{vormessung/keyevor78760tlci} 
% Bilddatei aus dem Unterverzeichnis bilder holen, skalieren auf 0.8*Satzspiegel
\caption {K-Eye mit 60p 787 Rot: TLCI} 
\end{figure}

\begin{figure}[htp]     % h=here, t=top, b=bottom, p=page
\centering
\includegraphics[width=0.9\textwidth]{vormessung/keyevor78760tm} 
% Bilddatei aus dem Unterverzeichnis bilder holen, skalieren auf 0.8*Satzspiegel
\caption {K-Eye mit 60p 787 Rot: TM-30} 
\end{figure}




\begin{figure}[htp]     % h=here, t=top, b=bottom, p=page
\centering
\includegraphics[width=0.9\textwidth]{vormessung/keyevor78770spec} 
% Bilddatei aus dem Unterverzeichnis bilder holen, skalieren auf 0.8*Satzspiegel
\caption {K-Eye mit 70p 787 Rot: Spektrum} 
\end{figure}

\begin{figure}[htp]     % h=here, t=top, b=bottom, p=page
\centering
\includegraphics[width=0.5\textwidth]{vormessung/keyevor78770cct} 
% Bilddatei aus dem Unterverzeichnis bilder holen, skalieren auf 0.8*Satzspiegel
\caption {K-Eye mit 70p 787 Rot: Beleuchtungsstärke, CCT, $\Delta$UV und Farbortkoordinaten} 
\end{figure}

\begin{figure}[htp]     % h=here, t=top, b=bottom, p=page
\centering
\includegraphics[width=0.9\textwidth]{vormessung/keyevor78770xy} 
% Bilddatei aus dem Unterverzeichnis bilder holen, skalieren auf 0.8*Satzspiegel
\caption {K-Eye mit 70p 787 Rot: XYZ-Farbraum Detailansicht} 
\end{figure}

\begin{figure}[htp]     % h=here, t=top, b=bottom, p=page
\centering
\includegraphics[width=0.9\textwidth]{vormessung/keyevor78770cri} 
% Bilddatei aus dem Unterverzeichnis bilder holen, skalieren auf 0.8*Satzspiegel
\caption {K-Eye mit 70p 787 Rot: CRI} 
\end{figure}

\begin{figure}[htp]     % h=here, t=top, b=bottom, p=page
\centering
\includegraphics[width=0.9\textwidth]{vormessung/keyevor78770cqs} 
% Bilddatei aus dem Unterverzeichnis bilder holen, skalieren auf 0.8*Satzspiegel
\caption {K-Eye mit 70p 787 Rot: CQS} 
\end{figure}

\begin{figure}[htp]     % h=here, t=top, b=bottom, p=page
\centering
\includegraphics[width=0.9\textwidth]{vormessung/keyevor78770tlci} 
% Bilddatei aus dem Unterverzeichnis bilder holen, skalieren auf 0.8*Satzspiegel
\caption {K-Eye mit 70p 787 Rot: TLCI} 
\end{figure}

\begin{figure}[htp]     % h=here, t=top, b=bottom, p=page
\centering
\includegraphics[width=0.9\textwidth]{vormessung/keyevor78770tm} 
% Bilddatei aus dem Unterverzeichnis bilder holen, skalieren auf 0.8*Satzspiegel
\caption {K-Eye mit 70p 787 Rot: TM-30} 
\end{figure}




\begin{figure}[htp]     % h=here, t=top, b=bottom, p=page
\centering
\includegraphics[width=0.9\textwidth]{vormessung/keyevor78780spec} 
% Bilddatei aus dem Unterverzeichnis bilder holen, skalieren auf 0.8*Satzspiegel
\caption {K-Eye mit 80p 787 Rot: Spektrum} 
\end{figure}

\begin{figure}[htp]     % h=here, t=top, b=bottom, p=page
\centering
\includegraphics[width=0.5\textwidth]{vormessung/keyevor78780cct} 
% Bilddatei aus dem Unterverzeichnis bilder holen, skalieren auf 0.8*Satzspiegel
\caption {K-Eye mit 80p 787 Rot: Beleuchtungsstärke, CCT, $\Delta$UV und Farbortkoordinaten} 
\end{figure}

\begin{figure}[htp]     % h=here, t=top, b=bottom, p=page
\centering
\includegraphics[width=0.9\textwidth]{vormessung/keyevor78780xy} 
% Bilddatei aus dem Unterverzeichnis bilder holen, skalieren auf 0.8*Satzspiegel
\caption {K-Eye mit 80p 787 Rot: XYZ-Farbraum Detailansicht} 
\end{figure}

\begin{figure}[htp]     % h=here, t=top, b=bottom, p=page
\centering
\includegraphics[width=0.9\textwidth]{vormessung/keyevor78780cri} 
% Bilddatei aus dem Unterverzeichnis bilder holen, skalieren auf 0.8*Satzspiegel
\caption {K-Eye mit 80p 787 Rot: CRI} 
\end{figure}

\begin{figure}[htp]     % h=here, t=top, b=bottom, p=page
\centering
\includegraphics[width=0.9\textwidth]{vormessung/keyevor78780cqs} 
% Bilddatei aus dem Unterverzeichnis bilder holen, skalieren auf 0.8*Satzspiegel
\caption {K-Eye mit 80p 787 Rot: CQS} 
\end{figure}

\begin{figure}[htp]     % h=here, t=top, b=bottom, p=page
\centering
\includegraphics[width=0.9\textwidth]{vormessung/keyevor78780tlci} 
% Bilddatei aus dem Unterverzeichnis bilder holen, skalieren auf 0.8*Satzspiegel
\caption {K-Eye mit 80p 787 Rot: TLCI} 
\end{figure}

\begin{figure}[htp]     % h=here, t=top, b=bottom, p=page
\centering
\includegraphics[width=0.9\textwidth]{vormessung/keyevor78780tm} 
% Bilddatei aus dem Unterverzeichnis bilder holen, skalieren auf 0.8*Satzspiegel
\caption {K-Eye mit 80p 787 Rot: TM-30} 
\end{figure}




\begin{figure}[htp]     % h=here, t=top, b=bottom, p=page
\centering
\includegraphics[width=0.9\textwidth]{vormessung/keyevor78790spec} 
% Bilddatei aus dem Unterverzeichnis bilder holen, skalieren auf 0.8*Satzspiegel
\caption {K-Eye mit 90p 787 Rot: Spektrum} 
\end{figure}

\begin{figure}[htp]     % h=here, t=top, b=bottom, p=page
\centering
\includegraphics[width=0.5\textwidth]{vormessung/keyevor78790cct} 
% Bilddatei aus dem Unterverzeichnis bilder holen, skalieren auf 0.8*Satzspiegel
\caption {K-Eye mit 90p 787 Rot: Beleuchtungsstärke, CCT, $\Delta$UV und Farbortkoordinaten} 
\end{figure}

\begin{figure}[htp]     % h=here, t=top, b=bottom, p=page
\centering
\includegraphics[width=0.9\textwidth]{vormessung/keyevor78790xy} 
% Bilddatei aus dem Unterverzeichnis bilder holen, skalieren auf 0.8*Satzspiegel
\caption {K-Eye mit 90p 787 Rot: XYZ-Farbraum Detailansicht} 
\end{figure}

\begin{figure}[htp]     % h=here, t=top, b=bottom, p=page
\centering
\includegraphics[width=0.9\textwidth]{vormessung/keyevor78790cri} 
% Bilddatei aus dem Unterverzeichnis bilder holen, skalieren auf 0.8*Satzspiegel
\caption {K-Eye mit 90p 787 Rot: CRI} 
\end{figure}

\begin{figure}[htp]     % h=here, t=top, b=bottom, p=page
\centering
\includegraphics[width=0.9\textwidth]{vormessung/keyevor78790cqs} 
% Bilddatei aus dem Unterverzeichnis bilder holen, skalieren auf 0.8*Satzspiegel
\caption {K-Eye mit 90p 787 Rot: CQS} 
\end{figure}

\begin{figure}[htp]     % h=here, t=top, b=bottom, p=page
\centering
\includegraphics[width=0.9\textwidth]{vormessung/keyevor78790tlci} 
% Bilddatei aus dem Unterverzeichnis bilder holen, skalieren auf 0.8*Satzspiegel
\caption {K-Eye mit 90p 787 Rot: TLCI} 
\end{figure}

\begin{figure}[htp]     % h=here, t=top, b=bottom, p=page
\centering
\includegraphics[width=0.9\textwidth]{vormessung/keyevor78790tm} 
% Bilddatei aus dem Unterverzeichnis bilder holen, skalieren auf 0.8*Satzspiegel
\caption {K-Eye mit 90p 787 Rot: TM-30} 
\end{figure}




\begin{figure}[htp]     % h=here, t=top, b=bottom, p=page
\centering
\includegraphics[width=0.9\textwidth]{vormessung/keyevor787100spec} 
% Bilddatei aus dem Unterverzeichnis bilder holen, skalieren auf 0.8*Satzspiegel
\caption {K-Eye mit 20p 787 Rot: Spektrum} 
\end{figure}

\begin{figure}[htp]     % h=here, t=top, b=bottom, p=page
\centering
\includegraphics[width=0.5\textwidth]{vormessung/keyevor78720cct} 
% Bilddatei aus dem Unterverzeichnis bilder holen, skalieren auf 0.8*Satzspiegel
\caption {K-Eye mit 100p 787 Rot: Beleuchtungsstärke, CCT, $\Delta$UV und Farbortkoordinaten} 
\end{figure}

\begin{figure}[htp]     % h=here, t=top, b=bottom, p=page
\centering
\includegraphics[width=0.9\textwidth]{vormessung/keyevor787100xy} 
% Bilddatei aus dem Unterverzeichnis bilder holen, skalieren auf 0.8*Satzspiegel
\caption {K-Eye mit 100p 787 Rot: XYZ-Farbraum Detailansicht} 
\end{figure}

\begin{figure}[htp]     % h=here, t=top, b=bottom, p=page
\centering
\includegraphics[width=0.9\textwidth]{vormessung/keyevor787100cri} 
% Bilddatei aus dem Unterverzeichnis bilder holen, skalieren auf 0.8*Satzspiegel
\caption {K-Eye mit 100p 787 Rot: CRI} 
\end{figure}

\begin{figure}[htp]     % h=here, t=top, b=bottom, p=page
\centering
\includegraphics[width=0.9\textwidth]{vormessung/keyevor787100cqs} 
% Bilddatei aus dem Unterverzeichnis bilder holen, skalieren auf 0.8*Satzspiegel
\caption {K-Eye mit 100p 787 Rot: CQS} 
\end{figure}

\begin{figure}[htp]     % h=here, t=top, b=bottom, p=page
\centering
\includegraphics[width=0.9\textwidth]{vormessung/keyevor787100tlci} 
% Bilddatei aus dem Unterverzeichnis bilder holen, skalieren auf 0.8*Satzspiegel
\caption {K-Eye mit 100p 787 Rot: TLCI} 
\end{figure}

\begin{figure}[htp]     % h=here, t=top, b=bottom, p=page
\centering
\includegraphics[width=0.9\textwidth]{vormessung/keyevor787100tm} 
% Bilddatei aus dem Unterverzeichnis bilder holen, skalieren auf 0.8*Satzspiegel
\caption {K-Eye mit 100p 787 Rot: TM-30} 
\end{figure}


\subsubsection{789 Rot in 10p Schritten}

\begin{figure}[htp]     % h=here, t=top, b=bottom, p=page
\centering
\includegraphics[width=0.9\textwidth]{vormessung/keyevor78910spec} 
% Bilddatei aus dem Unterverzeichnis bilder holen, skalieren auf 0.8*Satzspiegel
\caption {K-Eye mit 10p 789 Rot: Spektrum} 
\end{figure}

\begin{figure}[htp]     % h=here, t=top, b=bottom, p=page
\centering
\includegraphics[width=0.5\textwidth]{vormessung/keyevor78910cct} 
% Bilddatei aus dem Unterverzeichnis bilder holen, skalieren auf 0.8*Satzspiegel
\caption {K-Eye mit 10p 789 Rot: Beleuchtungsstärke, CCT, $\Delta$UV und Farbortkoordinaten} 
\end{figure}

\begin{figure}[htp]     % h=here, t=top, b=bottom, p=page
\centering
\includegraphics[width=0.9\textwidth]{vormessung/keyevor78910xy} 
% Bilddatei aus dem Unterverzeichnis bilder holen, skalieren auf 0.8*Satzspiegel
\caption {K-Eye mit 10p 789 Rot: XYZ-Farbraum Detailansicht} 
\end{figure}

\begin{figure}[htp]     % h=here, t=top, b=bottom, p=page
\centering
\includegraphics[width=0.9\textwidth]{vormessung/keyevor78910cri} 
% Bilddatei aus dem Unterverzeichnis bilder holen, skalieren auf 0.8*Satzspiegel
\caption {K-Eye mit 10p 789 Rot: CRI} 
\end{figure}

\begin{figure}[htp]     % h=here, t=top, b=bottom, p=page
\centering
\includegraphics[width=0.9\textwidth]{vormessung/keyevor78910cqs} 
% Bilddatei aus dem Unterverzeichnis bilder holen, skalieren auf 0.8*Satzspiegel
\caption {K-Eye mit 10p 789 Rot: CQS} 
\end{figure}

\begin{figure}[htp]     % h=here, t=top, b=bottom, p=page
\centering
\includegraphics[width=0.9\textwidth]{vormessung/keyevor78910tlci} 
% Bilddatei aus dem Unterverzeichnis bilder holen, skalieren auf 0.8*Satzspiegel
\caption {K-Eye mit 10p 789 Rot: TLCI} 
\end{figure}

\begin{figure}[htp]     % h=here, t=top, b=bottom, p=page
\centering
\includegraphics[width=0.9\textwidth]{vormessung/keyevor78910tm} 
% Bilddatei aus dem Unterverzeichnis bilder holen, skalieren auf 0.8*Satzspiegel
\caption {K-Eye mit 10p 789 Rot: TM-30} 
\end{figure}




\begin{figure}[htp]     % h=here, t=top, b=bottom, p=page
\centering
\includegraphics[width=0.9\textwidth]{vormessung/keyevor78920spec} 
% Bilddatei aus dem Unterverzeichnis bilder holen, skalieren auf 0.8*Satzspiegel
\caption {K-Eye mit 20p 789 Rot: Spektrum} 
\end{figure}

\begin{figure}[htp]     % h=here, t=top, b=bottom, p=page
\centering
\includegraphics[width=0.5\textwidth]{vormessung/keyevor78920cct} 
% Bilddatei aus dem Unterverzeichnis bilder holen, skalieren auf 0.8*Satzspiegel
\caption {K-Eye mit 20p 789 Rot: Beleuchtungsstärke, CCT, $\Delta$UV und Farbortkoordinaten} 
\end{figure}

\begin{figure}[htp]     % h=here, t=top, b=bottom, p=page
\centering
\includegraphics[width=0.9\textwidth]{vormessung/keyevor78920xy} 
% Bilddatei aus dem Unterverzeichnis bilder holen, skalieren auf 0.8*Satzspiegel
\caption {K-Eye mit 20p 789 Rot: XYZ-Farbraum Detailansicht} 
\end{figure}

\begin{figure}[htp]     % h=here, t=top, b=bottom, p=page
\centering
\includegraphics[width=0.9\textwidth]{vormessung/keyevor78920cri} 
% Bilddatei aus dem Unterverzeichnis bilder holen, skalieren auf 0.8*Satzspiegel
\caption {K-Eye mit 20p 789 Rot: CRI} 
\end{figure}

\begin{figure}[htp]     % h=here, t=top, b=bottom, p=page
\centering
\includegraphics[width=0.9\textwidth]{vormessung/keyevor78920cqs} 
% Bilddatei aus dem Unterverzeichnis bilder holen, skalieren auf 0.8*Satzspiegel
\caption {K-Eye mit 20p 789 Rot: CQS} 
\end{figure}

\begin{figure}[htp]     % h=here, t=top, b=bottom, p=page
\centering
\includegraphics[width=0.9\textwidth]{vormessung/keyevor78920tlci} 
% Bilddatei aus dem Unterverzeichnis bilder holen, skalieren auf 0.8*Satzspiegel
\caption {K-Eye mit 20p 789 Rot: TLCI} 
\end{figure}

\begin{figure}[htp]     % h=here, t=top, b=bottom, p=page
\centering
\includegraphics[width=0.9\textwidth]{vormessung/keyevor78920tm} 
% Bilddatei aus dem Unterverzeichnis bilder holen, skalieren auf 0.8*Satzspiegel
\caption {K-Eye mit 20p 789 Rot: TM-30} 
\end{figure}



\begin{figure}[htp]     % h=here, t=top, b=bottom, p=page
\centering
\includegraphics[width=0.9\textwidth]{vormessung/keyevor78930spec} 
% Bilddatei aus dem Unterverzeichnis bilder holen, skalieren auf 0.8*Satzspiegel
\caption {K-Eye mit 30p 789 Rot: Spektrum} 
\end{figure}

\begin{figure}[htp]     % h=here, t=top, b=bottom, p=page
\centering
\includegraphics[width=0.5\textwidth]{vormessung/keyevor78930cct} 
% Bilddatei aus dem Unterverzeichnis bilder holen, skalieren auf 0.8*Satzspiegel
\caption {K-Eye mit 30p 789 Rot: Beleuchtungsstärke, CCT, $\Delta$UV und Farbortkoordinaten} 
\end{figure}

\begin{figure}[htp]     % h=here, t=top, b=bottom, p=page
\centering
\includegraphics[width=0.9\textwidth]{vormessung/keyevor78930xy} 
% Bilddatei aus dem Unterverzeichnis bilder holen, skalieren auf 0.8*Satzspiegel
\caption {K-Eye mit 30p 789 Rot: XYZ-Farbraum Detailansicht} 
\end{figure}

\begin{figure}[htp]     % h=here, t=top, b=bottom, p=page
\centering
\includegraphics[width=0.9\textwidth]{vormessung/keyevor78930cri} 
% Bilddatei aus dem Unterverzeichnis bilder holen, skalieren auf 0.8*Satzspiegel
\caption {K-Eye mit 30p 789 Rot: CRI} 
\end{figure}

\begin{figure}[htp]     % h=here, t=top, b=bottom, p=page
\centering
\includegraphics[width=0.9\textwidth]{vormessung/keyevor78930cqs} 
% Bilddatei aus dem Unterverzeichnis bilder holen, skalieren auf 0.8*Satzspiegel
\caption {K-Eye mit 30p 789 Rot: CQS} 
\end{figure}

\begin{figure}[htp]     % h=here, t=top, b=bottom, p=page
\centering
\includegraphics[width=0.9\textwidth]{vormessung/keyevor78930tlci} 
% Bilddatei aus dem Unterverzeichnis bilder holen, skalieren auf 0.8*Satzspiegel
\caption {K-Eye mit 30p 789 Rot: TLCI} 
\end{figure}

\begin{figure}[htp]     % h=here, t=top, b=bottom, p=page
\centering
\includegraphics[width=0.9\textwidth]{vormessung/keyevor78930tm} 
% Bilddatei aus dem Unterverzeichnis bilder holen, skalieren auf 0.8*Satzspiegel
\caption {K-Eye mit 30p 789 Rot: TM-30} 
\end{figure}



\begin{figure}[htp]     % h=here, t=top, b=bottom, p=page
\centering
\includegraphics[width=0.9\textwidth]{vormessung/keyevor78940spec} 
% Bilddatei aus dem Unterverzeichnis bilder holen, skalieren auf 0.8*Satzspiegel
\caption {K-Eye mit 40p 789 Rot: Spektrum} 
\end{figure}

\begin{figure}[htp]     % h=here, t=top, b=bottom, p=page
\centering
\includegraphics[width=0.5\textwidth]{vormessung/keyevor78940cct} 
% Bilddatei aus dem Unterverzeichnis bilder holen, skalieren auf 0.8*Satzspiegel
\caption {K-Eye mit 40p 789 Rot: Beleuchtungsstärke, CCT, $\Delta$UV und Farbortkoordinaten} 
\end{figure}

\begin{figure}[htp]     % h=here, t=top, b=bottom, p=page
\centering
\includegraphics[width=0.9\textwidth]{vormessung/keyevor78940xy} 
% Bilddatei aus dem Unterverzeichnis bilder holen, skalieren auf 0.8*Satzspiegel
\caption {K-Eye mit 40p 789 Rot: XYZ-Farbraum Detailansicht} 
\end{figure}

\begin{figure}[htp]     % h=here, t=top, b=bottom, p=page
\centering
\includegraphics[width=0.9\textwidth]{vormessung/keyevor78940cri} 
% Bilddatei aus dem Unterverzeichnis bilder holen, skalieren auf 0.8*Satzspiegel
\caption {K-Eye mit 40p 789 Rot: CRI} 
\end{figure}

\begin{figure}[htp]     % h=here, t=top, b=bottom, p=page
\centering
\includegraphics[width=0.9\textwidth]{vormessung/keyevor78940cqs} 
% Bilddatei aus dem Unterverzeichnis bilder holen, skalieren auf 0.8*Satzspiegel
\caption {K-Eye mit 40p 789 Rot: CQS} 
\end{figure}

\begin{figure}[htp]     % h=here, t=top, b=bottom, p=page
\centering
\includegraphics[width=0.9\textwidth]{vormessung/keyevor78940tlci} 
% Bilddatei aus dem Unterverzeichnis bilder holen, skalieren auf 0.8*Satzspiegel
\caption {K-Eye mit 40p 789 Rot: TLCI} 
\end{figure}

\begin{figure}[htp]     % h=here, t=top, b=bottom, p=page
\centering
\includegraphics[width=0.9\textwidth]{vormessung/keyevor78940tm} 
% Bilddatei aus dem Unterverzeichnis bilder holen, skalieren auf 0.8*Satzspiegel
\caption {K-Eye mit 40p 789 Rot: TM-30} 
\end{figure}




\begin{figure}[htp]     % h=here, t=top, b=bottom, p=page
\centering
\includegraphics[width=0.9\textwidth]{vormessung/keyevor78950spec} 
% Bilddatei aus dem Unterverzeichnis bilder holen, skalieren auf 0.8*Satzspiegel
\caption {K-Eye mit 50p 789 Rot: Spektrum} 
\end{figure}

\begin{figure}[htp]     % h=here, t=top, b=bottom, p=page
\centering
\includegraphics[width=0.5\textwidth]{vormessung/keyevor78950cct} 
% Bilddatei aus dem Unterverzeichnis bilder holen, skalieren auf 0.8*Satzspiegel
\caption {K-Eye mit 50p 789 Rot: Beleuchtungsstärke, CCT, $\Delta$UV und Farbortkoordinaten} 
\end{figure}

\begin{figure}[htp]     % h=here, t=top, b=bottom, p=page
\centering
\includegraphics[width=0.9\textwidth]{vormessung/keyevor78950xy} 
% Bilddatei aus dem Unterverzeichnis bilder holen, skalieren auf 0.8*Satzspiegel
\caption {K-Eye mit 50p 789 Rot: XYZ-Farbraum Detailansicht} 
\end{figure}

\begin{figure}[htp]     % h=here, t=top, b=bottom, p=page
\centering
\includegraphics[width=0.9\textwidth]{vormessung/keyevor78950cri} 
% Bilddatei aus dem Unterverzeichnis bilder holen, skalieren auf 0.8*Satzspiegel
\caption {K-Eye mit 50p 789 Rot: CRI} 
\end{figure}

\begin{figure}[htp]     % h=here, t=top, b=bottom, p=page
\centering
\includegraphics[width=0.9\textwidth]{vormessung/keyevor78950cqs} 
% Bilddatei aus dem Unterverzeichnis bilder holen, skalieren auf 0.8*Satzspiegel
\caption {K-Eye mit 50p 789 Rot: CQS} 
\end{figure}

\begin{figure}[htp]     % h=here, t=top, b=bottom, p=page
\centering
\includegraphics[width=0.9\textwidth]{vormessung/keyevor78950tlci} 
% Bilddatei aus dem Unterverzeichnis bilder holen, skalieren auf 0.8*Satzspiegel
\caption {K-Eye mit 50p 789 Rot: TLCI} 
\end{figure}

\begin{figure}[htp]     % h=here, t=top, b=bottom, p=page
\centering
\includegraphics[width=0.9\textwidth]{vormessung/keyevor78950tm} 
% Bilddatei aus dem Unterverzeichnis bilder holen, skalieren auf 0.8*Satzspiegel
\caption {K-Eye mit 50p 789 Rot: TM-30} 
\end{figure}




\begin{figure}[htp]     % h=here, t=top, b=bottom, p=page
\centering
\includegraphics[width=0.9\textwidth]{vormessung/keyevor78960spec} 
% Bilddatei aus dem Unterverzeichnis bilder holen, skalieren auf 0.8*Satzspiegel
\caption {K-Eye mit 60p 789 Rot: Spektrum} 
\end{figure}

\begin{figure}[htp]     % h=here, t=top, b=bottom, p=page
\centering
\includegraphics[width=0.5\textwidth]{vormessung/keyevor78960cct} 
% Bilddatei aus dem Unterverzeichnis bilder holen, skalieren auf 0.8*Satzspiegel
\caption {K-Eye mit 60p 789 Rot: Beleuchtungsstärke, CCT, $\Delta$UV und Farbortkoordinaten} 
\end{figure}

\begin{figure}[htp]     % h=here, t=top, b=bottom, p=page
\centering
\includegraphics[width=0.9\textwidth]{vormessung/keyevor78960xy} 
% Bilddatei aus dem Unterverzeichnis bilder holen, skalieren auf 0.8*Satzspiegel
\caption {K-Eye mit 60p 789 Rot: XYZ-Farbraum Detailansicht} 
\end{figure}

\begin{figure}[htp]     % h=here, t=top, b=bottom, p=page
\centering
\includegraphics[width=0.9\textwidth]{vormessung/keyevor78960cri} 
% Bilddatei aus dem Unterverzeichnis bilder holen, skalieren auf 0.8*Satzspiegel
\caption {K-Eye mit 60p 789 Rot: CRI} 
\end{figure}

\begin{figure}[htp]     % h=here, t=top, b=bottom, p=page
\centering
\includegraphics[width=0.9\textwidth]{vormessung/keyevor78960cqs} 
% Bilddatei aus dem Unterverzeichnis bilder holen, skalieren auf 0.8*Satzspiegel
\caption {K-Eye mit 60p 789 Rot: CQS} 
\end{figure}

\begin{figure}[htp]     % h=here, t=top, b=bottom, p=page
\centering
\includegraphics[width=0.9\textwidth]{vormessung/keyevor78960tlci} 
% Bilddatei aus dem Unterverzeichnis bilder holen, skalieren auf 0.8*Satzspiegel
\caption {K-Eye mit 60p 789 Rot: TLCI} 
\end{figure}

\begin{figure}[htp]     % h=here, t=top, b=bottom, p=page
\centering
\includegraphics[width=0.9\textwidth]{vormessung/keyevor78960tm} 
% Bilddatei aus dem Unterverzeichnis bilder holen, skalieren auf 0.8*Satzspiegel
\caption {K-Eye mit 60p 789 Rot: TM-30} 
\end{figure}




\begin{figure}[htp]     % h=here, t=top, b=bottom, p=page
\centering
\includegraphics[width=0.9\textwidth]{vormessung/keyevor78970spec} 
% Bilddatei aus dem Unterverzeichnis bilder holen, skalieren auf 0.8*Satzspiegel
\caption {K-Eye mit 70p 789 Rot: Spektrum} 
\end{figure}

\begin{figure}[htp]     % h=here, t=top, b=bottom, p=page
\centering
\includegraphics[width=0.5\textwidth]{vormessung/keyevor78970cct} 
% Bilddatei aus dem Unterverzeichnis bilder holen, skalieren auf 0.8*Satzspiegel
\caption {K-Eye mit 70p 789 Rot: Beleuchtungsstärke, CCT, $\Delta$UV und Farbortkoordinaten} 
\end{figure}

\begin{figure}[htp]     % h=here, t=top, b=bottom, p=page
\centering
\includegraphics[width=0.9\textwidth]{vormessung/keyevor78970xy} 
% Bilddatei aus dem Unterverzeichnis bilder holen, skalieren auf 0.8*Satzspiegel
\caption {K-Eye mit 70p 789 Rot: XYZ-Farbraum Detailansicht} 
\end{figure}

\begin{figure}[htp]     % h=here, t=top, b=bottom, p=page
\centering
\includegraphics[width=0.9\textwidth]{vormessung/keyevor78970cri} 
% Bilddatei aus dem Unterverzeichnis bilder holen, skalieren auf 0.8*Satzspiegel
\caption {K-Eye mit 70p 789 Rot: CRI} 
\end{figure}

\begin{figure}[htp]     % h=here, t=top, b=bottom, p=page
\centering
\includegraphics[width=0.9\textwidth]{vormessung/keyevor78970cqs} 
% Bilddatei aus dem Unterverzeichnis bilder holen, skalieren auf 0.8*Satzspiegel
\caption {K-Eye mit 70p 789 Rot: CQS} 
\end{figure}

\begin{figure}[htp]     % h=here, t=top, b=bottom, p=page
\centering
\includegraphics[width=0.9\textwidth]{vormessung/keyevor78970tlci} 
% Bilddatei aus dem Unterverzeichnis bilder holen, skalieren auf 0.8*Satzspiegel
\caption {K-Eye mit 70p 789 Rot: TLCI} 
\end{figure}

\begin{figure}[htp]     % h=here, t=top, b=bottom, p=page
\centering
\includegraphics[width=0.9\textwidth]{vormessung/keyevor78970tm} 
% Bilddatei aus dem Unterverzeichnis bilder holen, skalieren auf 0.8*Satzspiegel
\caption {K-Eye mit 70p 789 Rot: TM-30} 
\end{figure}




\begin{figure}[htp]     % h=here, t=top, b=bottom, p=page
\centering
\includegraphics[width=0.9\textwidth]{vormessung/keyevor78980spec} 
% Bilddatei aus dem Unterverzeichnis bilder holen, skalieren auf 0.8*Satzspiegel
\caption {K-Eye mit 80p 789 Rot: Spektrum} 
\end{figure}

\begin{figure}[htp]     % h=here, t=top, b=bottom, p=page
\centering
\includegraphics[width=0.5\textwidth]{vormessung/keyevor78980cct} 
% Bilddatei aus dem Unterverzeichnis bilder holen, skalieren auf 0.8*Satzspiegel
\caption {K-Eye mit 80p 789 Rot: Beleuchtungsstärke, CCT, $\Delta$UV und Farbortkoordinaten} 
\end{figure}

\begin{figure}[htp]     % h=here, t=top, b=bottom, p=page
\centering
\includegraphics[width=0.9\textwidth]{vormessung/keyevor78980xy} 
% Bilddatei aus dem Unterverzeichnis bilder holen, skalieren auf 0.8*Satzspiegel
\caption {K-Eye mit 80p 789 Rot: XYZ-Farbraum Detailansicht} 
\end{figure}

\begin{figure}[htp]     % h=here, t=top, b=bottom, p=page
\centering
\includegraphics[width=0.9\textwidth]{vormessung/keyevor78980cri} 
% Bilddatei aus dem Unterverzeichnis bilder holen, skalieren auf 0.8*Satzspiegel
\caption {K-Eye mit 80p 789 Rot: CRI} 
\end{figure}

\begin{figure}[htp]     % h=here, t=top, b=bottom, p=page
\centering
\includegraphics[width=0.9\textwidth]{vormessung/keyevor78980cqs} 
% Bilddatei aus dem Unterverzeichnis bilder holen, skalieren auf 0.8*Satzspiegel
\caption {K-Eye mit 80p 789 Rot: CQS} 
\end{figure}

\begin{figure}[htp]     % h=here, t=top, b=bottom, p=page
\centering
\includegraphics[width=0.9\textwidth]{vormessung/keyevor78980tlci} 
% Bilddatei aus dem Unterverzeichnis bilder holen, skalieren auf 0.8*Satzspiegel
\caption {K-Eye mit 80p 789 Rot: TLCI} 
\end{figure}

\begin{figure}[htp]     % h=here, t=top, b=bottom, p=page
\centering
\includegraphics[width=0.9\textwidth]{vormessung/keyevor78980tm} 
% Bilddatei aus dem Unterverzeichnis bilder holen, skalieren auf 0.8*Satzspiegel
\caption {K-Eye mit 80p 789 Rot: TM-30} 
\end{figure}




\begin{figure}[htp]     % h=here, t=top, b=bottom, p=page
\centering
\includegraphics[width=0.9\textwidth]{vormessung/keyevor78990spec} 
% Bilddatei aus dem Unterverzeichnis bilder holen, skalieren auf 0.8*Satzspiegel
\caption {K-Eye mit 90p 789 Rot: Spektrum} 
\end{figure}

\begin{figure}[htp]     % h=here, t=top, b=bottom, p=page
\centering
\includegraphics[width=0.5\textwidth]{vormessung/keyevor78990cct} 
% Bilddatei aus dem Unterverzeichnis bilder holen, skalieren auf 0.8*Satzspiegel
\caption {K-Eye mit 90p 789 Rot: Beleuchtungsstärke, CCT, $\Delta$UV und Farbortkoordinaten} 
\end{figure}

\begin{figure}[htp]     % h=here, t=top, b=bottom, p=page
\centering
\includegraphics[width=0.9\textwidth]{vormessung/keyevor78990xy} 
% Bilddatei aus dem Unterverzeichnis bilder holen, skalieren auf 0.8*Satzspiegel
\caption {K-Eye mit 90p 789 Rot: XYZ-Farbraum Detailansicht} 
\end{figure}

\begin{figure}[htp]     % h=here, t=top, b=bottom, p=page
\centering
\includegraphics[width=0.9\textwidth]{vormessung/keyevor78990cri} 
% Bilddatei aus dem Unterverzeichnis bilder holen, skalieren auf 0.8*Satzspiegel
\caption {K-Eye mit 90p 789 Rot: CRI} 
\end{figure}

\begin{figure}[htp]     % h=here, t=top, b=bottom, p=page
\centering
\includegraphics[width=0.9\textwidth]{vormessung/keyevor78990cqs} 
% Bilddatei aus dem Unterverzeichnis bilder holen, skalieren auf 0.8*Satzspiegel
\caption {K-Eye mit 90p 789 Rot: CQS} 
\end{figure}

\begin{figure}[htp]     % h=here, t=top, b=bottom, p=page
\centering
\includegraphics[width=0.9\textwidth]{vormessung/keyevor78990tlci} 
% Bilddatei aus dem Unterverzeichnis bilder holen, skalieren auf 0.8*Satzspiegel
\caption {K-Eye mit 90p 789 Rot: TLCI} 
\end{figure}

\begin{figure}[htp]     % h=here, t=top, b=bottom, p=page
\centering
\includegraphics[width=0.9\textwidth]{vormessung/keyevor78990tm} 
% Bilddatei aus dem Unterverzeichnis bilder holen, skalieren auf 0.8*Satzspiegel
\caption {K-Eye mit 90p 789 Rot: TM-30} 
\end{figure}




\begin{figure}[htp]     % h=here, t=top, b=bottom, p=page
\centering
\includegraphics[width=0.9\textwidth]{vormessung/keyevor789100spec} 
% Bilddatei aus dem Unterverzeichnis bilder holen, skalieren auf 0.8*Satzspiegel
\caption {K-Eye mit 20p 789 Rot: Spektrum} 
\end{figure}

\begin{figure}[htp]     % h=here, t=top, b=bottom, p=page
\centering
\includegraphics[width=0.5\textwidth]{vormessung/keyevor789100cct} 
% Bilddatei aus dem Unterverzeichnis bilder holen, skalieren auf 0.8*Satzspiegel
\caption {K-Eye mit 100p 789 Rot: Beleuchtungsstärke, CCT, $\Delta$UV und Farbortkoordinaten} 
\end{figure}

\begin{figure}[htp]     % h=here, t=top, b=bottom, p=page
\centering
\includegraphics[width=0.9\textwidth]{vormessung/keyevor789100xy} 
% Bilddatei aus dem Unterverzeichnis bilder holen, skalieren auf 0.8*Satzspiegel
\caption {K-Eye mit 100p 789 Rot: XYZ-Farbraum Detailansicht} 
\end{figure}

\begin{figure}[htp]     % h=here, t=top, b=bottom, p=page
\centering
\includegraphics[width=0.9\textwidth]{vormessung/keyevor789100cri} 
% Bilddatei aus dem Unterverzeichnis bilder holen, skalieren auf 0.8*Satzspiegel
\caption {K-Eye mit 100p 789 Rot: CRI} 
\end{figure}

\begin{figure}[htp]     % h=here, t=top, b=bottom, p=page
\centering
\includegraphics[width=0.9\textwidth]{vormessung/keyevor789100cqs} 
% Bilddatei aus dem Unterverzeichnis bilder holen, skalieren auf 0.8*Satzspiegel
\caption {K-Eye mit 100p 789 Rot: CQS} 
\end{figure}

\begin{figure}[htp]     % h=here, t=top, b=bottom, p=page
\centering
\includegraphics[width=0.9\textwidth]{vormessung/keyevor789100tlci} 
% Bilddatei aus dem Unterverzeichnis bilder holen, skalieren auf 0.8*Satzspiegel
\caption {K-Eye mit 100p 789 Rot: TLCI} 
\end{figure}

\begin{figure}[htp]     % h=here, t=top, b=bottom, p=page
\centering
\includegraphics[width=0.9\textwidth]{vormessung/keyevor789100tm} 
% Bilddatei aus dem Unterverzeichnis bilder holen, skalieren auf 0.8*Satzspiegel
\caption {K-Eye mit 100p 789 Rot: TM-30} 
\end{figure}


\subsubsection{Umgebungslicht}

\begin{figure}[htp]     % h=here, t=top, b=bottom, p=page
\centering
\includegraphics[width=0.9\textwidth]{vormessung/umgebungslichtspec} 
% Bilddatei aus dem Unterverzeichnis bilder holen, skalieren auf 0.8*Satzspiegel
\caption {Umgebungslicht: Spektrum} 
\end{figure}

\begin{figure}[htp]     % h=here, t=top, b=bottom, p=page
\centering
\includegraphics[width=0.5\textwidth]{vormessung/umgebungslichtcct} 
% Bilddatei aus dem Unterverzeichnis bilder holen, skalieren auf 0.8*Satzspiegel
\caption {{Umgebungslicht: Beleuchtungsstärke, CCT, $\Delta$UV und Farbortkoordinaten} 
\end{figure}


\section{Hauptmessung}

\subsection{Arri D5}

\begin{figure}[htp]     % h=here, t=top, b=bottom, p=page
\centering
\includegraphics[width=0.9\textwidth]{hauptmessung/arriD5refspec} 
% Bilddatei aus dem Unterverzeichnis bilder holen, skalieren auf 0.8*Satzspiegel
\caption {Arri D5: Spektrum} 
\end{figure}

\begin{figure}[htp]     % h=here, t=top, b=bottom, p=page
\centering
\includegraphics[width=0.5\textwidth]{hauptmessung/arriD5refcct} 
% Bilddatei aus dem Unterverzeichnis bilder holen, skalieren auf 0.8*Satzspiegel
\caption {Arri D5: Beleuchtungsstärke, CCT, $\Delta$UV und Farbortkoordinaten} 
\end{figure}

\begin{figure}[htp]     % h=here, t=top, b=bottom, p=page
\centering
\includegraphics[width=0.9\textwidth]{hauptmessung/arriD5refxy} 
% Bilddatei aus dem Unterverzeichnis bilder holen, skalieren auf 0.8*Satzspiegel
\caption {Arri D5: XYZ-Farbraum Detailansicht} 
\end{figure}

\begin{figure}[htp]     % h=here, t=top, b=bottom, p=page
\centering
\includegraphics[width=0.9\textwidth]{hauptmessung/arriD5refcri} 
% Bilddatei aus dem Unterverzeichnis bilder holen, skalieren auf 0.8*Satzspiegel
\caption {Arri D5: CRI} 
\end{figure}

\begin{figure}[htp]     % h=here, t=top, b=bottom, p=page
\centering
\includegraphics[width=0.9\textwidth]{hauptmessung/arriD5refcqs} 
% Bilddatei aus dem Unterverzeichnis bilder holen, skalieren auf 0.8*Satzspiegel
\caption {Arri D5: CQS} 
\end{figure}

\begin{figure}[htp]     % h=here, t=top, b=bottom, p=page
\centering
\includegraphics[width=0.9\textwidth]{hauptmessung/arriD5reftlci} 
% Bilddatei aus dem Unterverzeichnis bilder holen, skalieren auf 0.8*Satzspiegel
\caption {Arri D5: TLCI} 
\end{figure}

\begin{figure}[htp]     % h=here, t=top, b=bottom, p=page
\centering
\includegraphics[width=0.9\textwidth]{hauptmessung/arriD5reftm} 
% Bilddatei aus dem Unterverzeichnis bilder holen, skalieren auf 0.8*Satzspiegel
\caption {Arri D5: TM-30} 
\end{figure}

\subsection{ Robe DL7F Wash}

\subsubsection{ Robe DL7F Wash - alle LED auf 100}

\begin{figure}[htp]     % h=here, t=top, b=bottom, p=page
\centering
\includegraphics[width=0.9\textwidth]{hauptmessung/dl7f100spec} 
% Bilddatei aus dem Unterverzeichnis bilder holen, skalieren auf 0.8*Satzspiegel
\caption { Robe DL7F Wash alle LED auf 100: Spektrum} 
\end{figure}

\begin{figure}[htp]     % h=here, t=top, b=bottom, p=page
\centering
\includegraphics[width=0.5\textwidth]{hauptmessung/dl7f100cct} 
% Bilddatei aus dem Unterverzeichnis bilder holen, skalieren auf 0.8*Satzspiegel
\caption { Robe DL7F Wash alle LED auf 100: Beleuchtungsstärke, CCT, $\Delta$UV und Farbortkoordinaten} 
\end{figure}

\begin{figure}[htp]     % h=here, t=top, b=bottom, p=page
\centering
\includegraphics[width=0.9\textwidth]{hauptmessung/dl7f100xy} 
% Bilddatei aus dem Unterverzeichnis bilder holen, skalieren auf 0.8*Satzspiegel
\caption { Robe DL7F Wash alle LED auf 100: XYZ-Farbraum Detailansicht} 
\end{figure}

\begin{figure}[htp]     % h=here, t=top, b=bottom, p=page
\centering
\includegraphics[width=0.9\textwidth]{hauptmessung/dl7f100cri} 
% Bilddatei aus dem Unterverzeichnis bilder holen, skalieren auf 0.8*Satzspiegel
\caption { Robe DL7F Wash alle LED auf 100: CRI} 
\end{figure}

\begin{figure}[htp]     % h=here, t=top, b=bottom, p=page
\centering
\includegraphics[width=0.9\textwidth]{hauptmessung/dl7f100cqs} 
% Bilddatei aus dem Unterverzeichnis bilder holen, skalieren auf 0.8*Satzspiegel
\caption { Robe DL7F Wash alle LED auf 100: CQS} 
\end{figure}

\begin{figure}[htp]     % h=here, t=top, b=bottom, p=page
\centering
\includegraphics[width=0.9\textwidth]{hauptmessung/dl7f100tlci} 
% Bilddatei aus dem Unterverzeichnis bilder holen, skalieren auf 0.8*Satzspiegel
\caption { Robe DL7F Wash alle LED auf 100: TLCI} 
\end{figure}

\begin{figure}[htp]     % h=here, t=top, b=bottom, p=page
\centering
\includegraphics[width=0.9\textwidth]{hauptmessung/dl7f100tm} 
% Bilddatei aus dem Unterverzeichnis bilder holen, skalieren auf 0.8*Satzspiegel
\caption { Robe DL7F Wash alle LED auf 100: TM-30} 
\end{figure}

\subsubsection{ Robe DL7F Wash ohne Rot}

\begin{figure}[htp]     % h=here, t=top, b=bottom, p=page
\centering
\includegraphics[width=0.9\textwidth]{hauptmessung/dl7fbestspec} 
% Bilddatei aus dem Unterverzeichnis bilder holen, skalieren auf 0.8*Satzspiegel
\caption { Robe DL7F Wash ohne Rot: Spektrum} 
\end{figure}

\begin{figure}[htp]     % h=here, t=top, b=bottom, p=page
\centering
\includegraphics[width=0.5\textwidth]{hauptmessung/dl7fbestcct} 
% Bilddatei aus dem Unterverzeichnis bilder holen, skalieren auf 0.8*Satzspiegel
\caption { Robe DL7F Wash ohne Rot: Beleuchtungsstärke, CCT, $\Delta$UV und Farbortkoordinaten} 
\end{figure}

\begin{figure}[htp]     % h=here, t=top, b=bottom, p=page
\centering
\includegraphics[width=0.9\textwidth]{hauptmessung/dl7fbestxy} 
% Bilddatei aus dem Unterverzeichnis bilder holen, skalieren auf 0.8*Satzspiegel
\caption { Robe DL7F Wash ohne Rot: XYZ-Farbraum Detailansicht} 
\end{figure}

\begin{figure}[htp]     % h=here, t=top, b=bottom, p=page
\centering
\includegraphics[width=0.9\textwidth]{hauptmessung/dl7fbestcri} 
% Bilddatei aus dem Unterverzeichnis bilder holen, skalieren auf 0.8*Satzspiegel
\caption { Robe DL7F Wash ohne Rot: CRI} 
\end{figure}

\begin{figure}[htp]     % h=here, t=top, b=bottom, p=page
\centering
\includegraphics[width=0.9\textwidth]{hauptmessung/dl7fbestcqs} 
% Bilddatei aus dem Unterverzeichnis bilder holen, skalieren auf 0.8*Satzspiegel
\caption { Robe DL7F Wash ohne Rot: CQS} 
\end{figure}

\begin{figure}[htp]     % h=here, t=top, b=bottom, p=page
\centering
\includegraphics[width=0.9\textwidth]{hauptmessung/dl7fbesttlci} 
% Bilddatei aus dem Unterverzeichnis bilder holen, skalieren auf 0.8*Satzspiegel
\caption { Robe DL7F Wash ohne Rot: TLCI} 
\end{figure}

\begin{figure}[htp]     % h=here, t=top, b=bottom, p=page
\centering
\includegraphics[width=0.9\textwidth]{hauptmessung/dl7fbesttm} 
% Bilddatei aus dem Unterverzeichnis bilder holen, skalieren auf 0.8*Satzspiegel
\caption { Robe DL7F Wash ohne Rot: TM-30} 
\end{figure}

\subsubsection{ Robe DL7F Wash mit 027 Rot}

\begin{figure}[htp]     % h=here, t=top, b=bottom, p=page
\centering
\includegraphics[width=0.9\textwidth]{hauptmessung/dl7f027spec} 
% Bilddatei aus dem Unterverzeichnis bilder holen, skalieren auf 0.8*Satzspiegel
\caption { Robe DL7F Wash mit 027 Rot: Spektrum} 
\end{figure}

\begin{figure}[htp]     % h=here, t=top, b=bottom, p=page
\centering
\includegraphics[width=0.5\textwidth]{hauptmessung/dl7f027cct} 
% Bilddatei aus dem Unterverzeichnis bilder holen, skalieren auf 0.8*Satzspiegel
\caption { Robe DL7F Wash mit 027 Rot: Beleuchtungsstärke, CCT, $\Delta$UV und Farbortkoordinaten} 
\end{figure}

\begin{figure}[htp]     % h=here, t=top, b=bottom, p=page
\centering
\includegraphics[width=0.9\textwidth]{hauptmessung/dl7f027xy} 
% Bilddatei aus dem Unterverzeichnis bilder holen, skalieren auf 0.8*Satzspiegel
\caption { Robe DL7F Wash mit 027 Rot: XYZ-Farbraum Detailansicht} 
\end{figure}

\begin{figure}[htp]     % h=here, t=top, b=bottom, p=page
\centering
\includegraphics[width=0.9\textwidth]{hauptmessung/dl7f027cri} 
% Bilddatei aus dem Unterverzeichnis bilder holen, skalieren auf 0.8*Satzspiegel
\caption { Robe DL7F Wash mit 027 Rot: CRI} 
\end{figure}

\begin{figure}[htp]     % h=here, t=top, b=bottom, p=page
\centering
\includegraphics[width=0.9\textwidth]{hauptmessung/dl7f027cqs} 
% Bilddatei aus dem Unterverzeichnis bilder holen, skalieren auf 0.8*Satzspiegel
\caption { Robe DL7F Wash mit 027 Rot: CQS} 
\end{figure}

\begin{figure}[htp]     % h=here, t=top, b=bottom, p=page
\centering
\includegraphics[width=0.9\textwidth]{hauptmessung/dl7f027tlci} 
% Bilddatei aus dem Unterverzeichnis bilder holen, skalieren auf 0.8*Satzspiegel
\caption { Robe DL7F Wash mit 027 Rot: TLCI} 
\end{figure}

\begin{figure}[htp]     % h=here, t=top, b=bottom, p=page
\centering
\includegraphics[width=0.9\textwidth]{hauptmessung/dl7f027tm} 
% Bilddatei aus dem Unterverzeichnis bilder holen, skalieren auf 0.8*Satzspiegel
\caption { Robe DL7F Wash mit 027 Rot: TM-30} 
\end{figure}

\subsubsection{ Robe DL7F Wash mit 787 Rot}

\begin{figure}[htp]     % h=here, t=top, b=bottom, p=page
\centering
\includegraphics[width=0.9\textwidth]{hauptmessung/dl7f787spec} 
% Bilddatei aus dem Unterverzeichnis bilder holen, skalieren auf 0.8*Satzspiegel
\caption { Robe DL7F Wash mit 787 Rot: Spektrum} 
\end{figure}

\begin{figure}[htp]     % h=here, t=top, b=bottom, p=page
\centering
\includegraphics[width=0.5\textwidth]{hauptmessung/dl7f787cct} 
% Bilddatei aus dem Unterverzeichnis bilder holen, skalieren auf 0.8*Satzspiegel
\caption { Robe DL7F Wash mit 787 Rot: Beleuchtungsstärke, CCT, $\Delta$UV und Farbortkoordinaten} 
\end{figure}

\begin{figure}[htp]     % h=here, t=top, b=bottom, p=page
\centering
\includegraphics[width=0.9\textwidth]{hauptmessung/dl7f787xy} 
% Bilddatei aus dem Unterverzeichnis bilder holen, skalieren auf 0.8*Satzspiegel
\caption { Robe DL7F Wash mit 787 Rot: XYZ-Farbraum Detailansicht} 
\end{figure}

\begin{figure}[htp]     % h=here, t=top, b=bottom, p=page
\centering
\includegraphics[width=0.9\textwidth]{hauptmessung/dl7f787cri} 
% Bilddatei aus dem Unterverzeichnis bilder holen, skalieren auf 0.8*Satzspiegel
\caption { Robe DL7F Wash mit 787 Rot: CRI} 
\end{figure}

\begin{figure}[htp]     % h=here, t=top, b=bottom, p=page
\centering
\includegraphics[width=0.9\textwidth]{hauptmessung/dl7f787cqs} 
% Bilddatei aus dem Unterverzeichnis bilder holen, skalieren auf 0.8*Satzspiegel
\caption { Robe DL7F Wash mit 787 Rot: CQS} 
\end{figure}

\begin{figure}[htp]     % h=here, t=top, b=bottom, p=page
\centering
\includegraphics[width=0.9\textwidth]{hauptmessung/dl7f787tlci} 
% Bilddatei aus dem Unterverzeichnis bilder holen, skalieren auf 0.8*Satzspiegel
\caption { Robe DL7F Wash mit 787 Rot: TLCI} 
\end{figure}

\begin{figure}[htp]     % h=here, t=top, b=bottom, p=page
\centering
\includegraphics[width=0.9\textwidth]{hauptmessung/dl7f787tm} 
% Bilddatei aus dem Unterverzeichnis bilder holen, skalieren auf 0.8*Satzspiegel
\caption { Robe DL7F Wash mit 787 Rot: TM-30} 
\end{figure}


\subsection{Martin MAC Encore Wash CLD}

\subsubsection{Martin MAC Encore Wash CLD - alle LED auf 100}

\begin{figure}[htp]     % h=here, t=top, b=bottom, p=page
\centering
\includegraphics[width=0.9\textwidth]{hauptmessung/encore100spec} 
% Bilddatei aus dem Unterverzeichnis bilder holen, skalieren auf 0.8*Satzspiegel
\caption {Martin MAC Encore Wash CLD alle LED auf 100: Spektrum} 
\end{figure}

\begin{figure}[htp]     % h=here, t=top, b=bottom, p=page
\centering
\includegraphics[width=0.5\textwidth]{hauptmessung/encore100cct} 
% Bilddatei aus dem Unterverzeichnis bilder holen, skalieren auf 0.8*Satzspiegel
\caption {Martin MAC Encore Wash CLD alle LED auf 100: Beleuchtungsstärke, CCT, $\Delta$UV und Farbortkoordinaten} 
\end{figure}

\begin{figure}[htp]     % h=here, t=top, b=bottom, p=page
\centering
\includegraphics[width=0.9\textwidth]{hauptmessung/encore100xy} 
% Bilddatei aus dem Unterverzeichnis bilder holen, skalieren auf 0.8*Satzspiegel
\caption {Martin MAC Encore Wash CLD alle LED auf 100: XYZ-Farbraum Detailansicht} 
\end{figure}

\begin{figure}[htp]     % h=here, t=top, b=bottom, p=page
\centering
\includegraphics[width=0.9\textwidth]{hauptmessung/encore100cri} 
% Bilddatei aus dem Unterverzeichnis bilder holen, skalieren auf 0.8*Satzspiegel
\caption {Martin MAC Encore Wash CLD alle LED auf 100: CRI} 
\end{figure}

\begin{figure}[htp]     % h=here, t=top, b=bottom, p=page
\centering
\includegraphics[width=0.9\textwidth]{hauptmessung/encore100cqs} 
% Bilddatei aus dem Unterverzeichnis bilder holen, skalieren auf 0.8*Satzspiegel
\caption {Martin MAC Encore Wash CLD alle LED auf 100: CQS} 
\end{figure}

\begin{figure}[htp]     % h=here, t=top, b=bottom, p=page
\centering
\includegraphics[width=0.9\textwidth]{hauptmessung/encore100tlci} 
% Bilddatei aus dem Unterverzeichnis bilder holen, skalieren auf 0.8*Satzspiegel
\caption {Martin MAC Encore Wash CLD alle LED auf 100: TLCI} 
\end{figure}

\begin{figure}[htp]     % h=here, t=top, b=bottom, p=page
\centering
\includegraphics[width=0.9\textwidth]{hauptmessung/encore100tm} 
% Bilddatei aus dem Unterverzeichnis bilder holen, skalieren auf 0.8*Satzspiegel
\caption {Martin MAC Encore Wash CLD alle LED auf 100: TM-30} 
\end{figure}

\subsubsection{Martin MAC Encore Wash CLD ohne Rot}

\begin{figure}[htp]     % h=here, t=top, b=bottom, p=page
\centering
\includegraphics[width=0.9\textwidth]{hauptmessung/encorebestspec} 
% Bilddatei aus dem Unterverzeichnis bilder holen, skalieren auf 0.8*Satzspiegel
\caption {Martin MAC Encore Wash CLD ohne Rot: Spektrum} 
\end{figure}

\begin{figure}[htp]     % h=here, t=top, b=bottom, p=page
\centering
\includegraphics[width=0.5\textwidth]{hauptmessung/encorebestcct} 
% Bilddatei aus dem Unterverzeichnis bilder holen, skalieren auf 0.8*Satzspiegel
\caption {Martin MAC Encore Wash CLD ohne Rot: Beleuchtungsstärke, CCT, $\Delta$UV und Farbortkoordinaten} 
\end{figure}

\begin{figure}[htp]     % h=here, t=top, b=bottom, p=page
\centering
\includegraphics[width=0.9\textwidth]{hauptmessung/encorebestxy} 
% Bilddatei aus dem Unterverzeichnis bilder holen, skalieren auf 0.8*Satzspiegel
\caption {Martin MAC Encore Wash CLD ohne Rot: XYZ-Farbraum Detailansicht} 
\end{figure}

\begin{figure}[htp]     % h=here, t=top, b=bottom, p=page
\centering
\includegraphics[width=0.9\textwidth]{hauptmessung/encorebestcri} 
% Bilddatei aus dem Unterverzeichnis bilder holen, skalieren auf 0.8*Satzspiegel
\caption {Martin MAC Encore Wash CLD ohne Rot: CRI} 
\end{figure}

\begin{figure}[htp]     % h=here, t=top, b=bottom, p=page
\centering
\includegraphics[width=0.9\textwidth]{hauptmessung/encorebestcqs} 
% Bilddatei aus dem Unterverzeichnis bilder holen, skalieren auf 0.8*Satzspiegel
\caption {Martin MAC Encore Wash CLD ohne Rot: CQS} 
\end{figure}

\begin{figure}[htp]     % h=here, t=top, b=bottom, p=page
\centering
\includegraphics[width=0.9\textwidth]{hauptmessung/encorebesttlci} 
% Bilddatei aus dem Unterverzeichnis bilder holen, skalieren auf 0.8*Satzspiegel
\caption {Martin MAC Encore Wash CLD ohne Rot: TLCI} 
\end{figure}

\begin{figure}[htp]     % h=here, t=top, b=bottom, p=page
\centering
\includegraphics[width=0.9\textwidth]{hauptmessung/encorebesttm} 
% Bilddatei aus dem Unterverzeichnis bilder holen, skalieren auf 0.8*Satzspiegel
\caption {Martin MAC Encore Wash CLD ohne Rot: TM-30} 
\end{figure}

\subsubsection{Martin MAC Encore Wash CLD mit 027 Rot}

\begin{figure}[htp]     % h=here, t=top, b=bottom, p=page
\centering
\includegraphics[width=0.9\textwidth]{hauptmessung/encore027spec} 
% Bilddatei aus dem Unterverzeichnis bilder holen, skalieren auf 0.8*Satzspiegel
\caption {Martin MAC Encore Wash CLD mit 027 Rot: Spektrum} 
\end{figure}

\begin{figure}[htp]     % h=here, t=top, b=bottom, p=page
\centering
\includegraphics[width=0.5\textwidth]{hauptmessung/encore027cct} 
% Bilddatei aus dem Unterverzeichnis bilder holen, skalieren auf 0.8*Satzspiegel
\caption {Martin MAC Encore Wash CLD mit 027 Rot: Beleuchtungsstärke, CCT, $\Delta$UV und Farbortkoordinaten} 
\end{figure}

\begin{figure}[htp]     % h=here, t=top, b=bottom, p=page
\centering
\includegraphics[width=0.9\textwidth]{hauptmessung/encore027xy} 
% Bilddatei aus dem Unterverzeichnis bilder holen, skalieren auf 0.8*Satzspiegel
\caption {Martin MAC Encore Wash CLD mit 027 Rot: XYZ-Farbraum Detailansicht} 
\end{figure}

\begin{figure}[htp]     % h=here, t=top, b=bottom, p=page
\centering
\includegraphics[width=0.9\textwidth]{hauptmessung/encore027cri} 
% Bilddatei aus dem Unterverzeichnis bilder holen, skalieren auf 0.8*Satzspiegel
\caption {Martin MAC Encore Wash CLD mit 027 Rot: CRI} 
\end{figure}

\begin{figure}[htp]     % h=here, t=top, b=bottom, p=page
\centering
\includegraphics[width=0.9\textwidth]{hauptmessung/encore027cqs} 
% Bilddatei aus dem Unterverzeichnis bilder holen, skalieren auf 0.8*Satzspiegel
\caption {Martin MAC Encore Wash CLD mit 027 Rot: CQS} 
\end{figure}

\begin{figure}[htp]     % h=here, t=top, b=bottom, p=page
\centering
\includegraphics[width=0.9\textwidth]{hauptmessung/encore027tlci} 
% Bilddatei aus dem Unterverzeichnis bilder holen, skalieren auf 0.8*Satzspiegel
\caption {Martin MAC Encore Wash CLD mit 027 Rot: TLCI} 
\end{figure}

\begin{figure}[htp]     % h=here, t=top, b=bottom, p=page
\centering
\includegraphics[width=0.9\textwidth]{hauptmessung/encore027tm} 
% Bilddatei aus dem Unterverzeichnis bilder holen, skalieren auf 0.8*Satzspiegel
\caption {Martin MAC Encore Wash CLD mit 027 Rot: TM-30} 
\end{figure}

\subsubsection{Martin MAC Encore Wash CLD mit 787 Rot}

\begin{figure}[htp]     % h=here, t=top, b=bottom, p=page
\centering
\includegraphics[width=0.9\textwidth]{hauptmessung/encore787spec} 
% Bilddatei aus dem Unterverzeichnis bilder holen, skalieren auf 0.8*Satzspiegel
\caption {Martin MAC Encore Wash CLD mit 787 Rot: Spektrum} 
\end{figure}

\begin{figure}[htp]     % h=here, t=top, b=bottom, p=page
\centering
\includegraphics[width=0.5\textwidth]{hauptmessung/encore787cct} 
% Bilddatei aus dem Unterverzeichnis bilder holen, skalieren auf 0.8*Satzspiegel
\caption {Martin MAC Encore Wash CLD mit 787 Rot: Beleuchtungsstärke, CCT, $\Delta$UV und Farbortkoordinaten} 
\end{figure}

\begin{figure}[htp]     % h=here, t=top, b=bottom, p=page
\centering
\includegraphics[width=0.9\textwidth]{hauptmessung/encore787xy} 
% Bilddatei aus dem Unterverzeichnis bilder holen, skalieren auf 0.8*Satzspiegel
\caption {Martin MAC Encore Wash CLD mit 787 Rot: XYZ-Farbraum Detailansicht} 
\end{figure}

\begin{figure}[htp]     % h=here, t=top, b=bottom, p=page
\centering
\includegraphics[width=0.9\textwidth]{hauptmessung/encore787cri} 
% Bilddatei aus dem Unterverzeichnis bilder holen, skalieren auf 0.8*Satzspiegel
\caption {Martin MAC Encore Wash CLD mit 787 Rot: CRI} 
\end{figure}

\begin{figure}[htp]     % h=here, t=top, b=bottom, p=page
\centering
\includegraphics[width=0.9\textwidth]{hauptmessung/encore787cqs} 
% Bilddatei aus dem Unterverzeichnis bilder holen, skalieren auf 0.8*Satzspiegel
\caption {Martin MAC Encore Wash CLD mit 787 Rot: CQS} 
\end{figure}

\begin{figure}[htp]     % h=here, t=top, b=bottom, p=page
\centering
\includegraphics[width=0.9\textwidth]{hauptmessung/encore787tlci} 
% Bilddatei aus dem Unterverzeichnis bilder holen, skalieren auf 0.8*Satzspiegel
\caption {Martin MAC Encore Wash CLD mit 787 Rot: TLCI} 
\end{figure}

\begin{figure}[htp]     % h=here, t=top, b=bottom, p=page
\centering
\includegraphics[width=0.9\textwidth]{hauptmessung/encore787tm} 
% Bilddatei aus dem Unterverzeichnis bilder holen, skalieren auf 0.8*Satzspiegel
\caption {Martin MAC Encore Wash CLD mit 787 Rot: TM-30} 
\end{figure}


\subsection{ETC Source Four LED Series 2 Lustr}

\subsubsection{ETC Source Four LED Series 2 Lustr - alle LED auf 100}

\begin{figure}[htp]     % h=here, t=top, b=bottom, p=page
\centering
\includegraphics[width=0.9\textwidth]{hauptmessung/etc100spec} 
% Bilddatei aus dem Unterverzeichnis bilder holen, skalieren auf 0.8*Satzspiegel
\caption {ETC Source Four LED Series 2 Lustr alle LED auf 100: Spektrum} 
\end{figure}

\begin{figure}[htp]     % h=here, t=top, b=bottom, p=page
\centering
\includegraphics[width=0.5\textwidth]{hauptmessung/etc100cct} 
% Bilddatei aus dem Unterverzeichnis bilder holen, skalieren auf 0.8*Satzspiegel
\caption {ETC Source Four LED Series 2 Lustr alle LED auf 100: Beleuchtungsstärke, CCT, $\Delta$UV und Farbortkoordinaten} 
\end{figure}

\begin{figure}[htp]     % h=here, t=top, b=bottom, p=page
\centering
\includegraphics[width=0.9\textwidth]{hauptmessung/etc100xy} 
% Bilddatei aus dem Unterverzeichnis bilder holen, skalieren auf 0.8*Satzspiegel
\caption {ETC Source Four LED Series 2 Lustr alle LED auf 100: XYZ-Farbraum Detailansicht} 
\end{figure}

\begin{figure}[htp]     % h=here, t=top, b=bottom, p=page
\centering
\includegraphics[width=0.9\textwidth]{hauptmessung/etc100cri} 
% Bilddatei aus dem Unterverzeichnis bilder holen, skalieren auf 0.8*Satzspiegel
\caption {ETC Source Four LED Series 2 Lustr alle LED auf 100: CRI} 
\end{figure}

\begin{figure}[htp]     % h=here, t=top, b=bottom, p=page
\centering
\includegraphics[width=0.9\textwidth]{hauptmessung/etc100cqs} 
% Bilddatei aus dem Unterverzeichnis bilder holen, skalieren auf 0.8*Satzspiegel
\caption {ETC Source Four LED Series 2 Lustr alle LED auf 100: CQS} 
\end{figure}

\begin{figure}[htp]     % h=here, t=top, b=bottom, p=page
\centering
\includegraphics[width=0.9\textwidth]{hauptmessung/etc100tlci} 
% Bilddatei aus dem Unterverzeichnis bilder holen, skalieren auf 0.8*Satzspiegel
\caption {ETC Source Four LED Series 2 Lustr alle LED auf 100: TLCI} 
\end{figure}

\begin{figure}[htp]     % h=here, t=top, b=bottom, p=page
\centering
\includegraphics[width=0.9\textwidth]{hauptmessung/etc100tm} 
% Bilddatei aus dem Unterverzeichnis bilder holen, skalieren auf 0.8*Satzspiegel
\caption {ETC Source Four LED Series 2 Lustr alle LED auf 100: TM-30} 
\end{figure}

\subsubsection{ETC Source Four LED Series 2 Lustr ohne Rot}

\begin{figure}[htp]     % h=here, t=top, b=bottom, p=page
\centering
\includegraphics[width=0.9\textwidth]{hauptmessung/etcbestspec} 
% Bilddatei aus dem Unterverzeichnis bilder holen, skalieren auf 0.8*Satzspiegel
\caption {ETC Source Four LED Series 2 Lustr ohne Rot: Spektrum} 
\end{figure}

\begin{figure}[htp]     % h=here, t=top, b=bottom, p=page
\centering
\includegraphics[width=0.5\textwidth]{hauptmessung/etcbestcct} 
% Bilddatei aus dem Unterverzeichnis bilder holen, skalieren auf 0.8*Satzspiegel
\caption {ETC Source Four LED Series 2 Lustr ohne Rot: Beleuchtungsstärke, CCT, $\Delta$UV und Farbortkoordinaten} 
\end{figure}

\begin{figure}[htp]     % h=here, t=top, b=bottom, p=page
\centering
\includegraphics[width=0.9\textwidth]{hauptmessung/etcbestxy} 
% Bilddatei aus dem Unterverzeichnis bilder holen, skalieren auf 0.8*Satzspiegel
\caption {ETC Source Four LED Series 2 Lustr ohne Rot: XYZ-Farbraum Detailansicht} 
\end{figure}

\begin{figure}[htp]     % h=here, t=top, b=bottom, p=page
\centering
\includegraphics[width=0.9\textwidth]{hauptmessung/etcbestcri} 
% Bilddatei aus dem Unterverzeichnis bilder holen, skalieren auf 0.8*Satzspiegel
\caption {ETC Source Four LED Series 2 Lustr ohne Rot: CRI} 
\end{figure}

\begin{figure}[htp]     % h=here, t=top, b=bottom, p=page
\centering
\includegraphics[width=0.9\textwidth]{hauptmessung/etcbestcqs} 
% Bilddatei aus dem Unterverzeichnis bilder holen, skalieren auf 0.8*Satzspiegel
\caption {ETC Source Four LED Series 2 Lustr ohne Rot: CQS} 
\end{figure}

\begin{figure}[htp]     % h=here, t=top, b=bottom, p=page
\centering
\includegraphics[width=0.9\textwidth]{hauptmessung/etcbesttlci} 
% Bilddatei aus dem Unterverzeichnis bilder holen, skalieren auf 0.8*Satzspiegel
\caption {ETC Source Four LED Series 2 Lustr ohne Rot: TLCI} 
\end{figure}

\begin{figure}[htp]     % h=here, t=top, b=bottom, p=page
\centering
\includegraphics[width=0.9\textwidth]{hauptmessung/etcbesttm} 
% Bilddatei aus dem Unterverzeichnis bilder holen, skalieren auf 0.8*Satzspiegel
\caption {ETC Source Four LED Series 2 Lustr ohne Rot: TM-30} 
\end{figure}

\subsubsection{ETC Source Four LED Series 2 Lustr mit 027 Rot}

\begin{figure}[htp]     % h=here, t=top, b=bottom, p=page
\centering
\includegraphics[width=0.9\textwidth]{hauptmessung/etc027spec} 
% Bilddatei aus dem Unterverzeichnis bilder holen, skalieren auf 0.8*Satzspiegel
\caption {ETC Source Four LED Series 2 Lustr mit 027 Rot: Spektrum} 
\end{figure}

\begin{figure}[htp]     % h=here, t=top, b=bottom, p=page
\centering
\includegraphics[width=0.5\textwidth]{hauptmessung/etc027cct} 
% Bilddatei aus dem Unterverzeichnis bilder holen, skalieren auf 0.8*Satzspiegel
\caption {ETC Source Four LED Series 2 Lustr mit 027 Rot: Beleuchtungsstärke, CCT, $\Delta$UV und Farbortkoordinaten} 
\end{figure}

\begin{figure}[htp]     % h=here, t=top, b=bottom, p=page
\centering
\includegraphics[width=0.9\textwidth]{hauptmessung/etc027xy} 
% Bilddatei aus dem Unterverzeichnis bilder holen, skalieren auf 0.8*Satzspiegel
\caption {ETC Source Four LED Series 2 Lustr mit 027 Rot: XYZ-Farbraum Detailansicht} 
\end{figure}

\begin{figure}[htp]     % h=here, t=top, b=bottom, p=page
\centering
\includegraphics[width=0.9\textwidth]{hauptmessung/etc027cri} 
% Bilddatei aus dem Unterverzeichnis bilder holen, skalieren auf 0.8*Satzspiegel
\caption {ETC Source Four LED Series 2 Lustr mit 027 Rot: CRI} 
\end{figure}

\begin{figure}[htp]     % h=here, t=top, b=bottom, p=page
\centering
\includegraphics[width=0.9\textwidth]{hauptmessung/etc027cqs} 
% Bilddatei aus dem Unterverzeichnis bilder holen, skalieren auf 0.8*Satzspiegel
\caption {ETC Source Four LED Series 2 Lustr mit 027 Rot: CQS} 
\end{figure}

\begin{figure}[htp]     % h=here, t=top, b=bottom, p=page
\centering
\includegraphics[width=0.9\textwidth]{hauptmessung/etc027tlci} 
% Bilddatei aus dem Unterverzeichnis bilder holen, skalieren auf 0.8*Satzspiegel
\caption {ETC Source Four LED Series 2 Lustr mit 027 Rot: TLCI} 
\end{figure}

\begin{figure}[htp]     % h=here, t=top, b=bottom, p=page
\centering
\includegraphics[width=0.9\textwidth]{hauptmessung/etc027tm} 
% Bilddatei aus dem Unterverzeichnis bilder holen, skalieren auf 0.8*Satzspiegel
\caption {ETC Source Four LED Series 2 Lustr mit 027 Rot: TM-30} 
\end{figure}

\subsubsection{ETC Source Four LED Series 2 Lustr mit 787 Rot}

\begin{figure}[htp]     % h=here, t=top, b=bottom, p=page
\centering
\includegraphics[width=0.9\textwidth]{hauptmessung/etc787spec} 
% Bilddatei aus dem Unterverzeichnis bilder holen, skalieren auf 0.8*Satzspiegel
\caption {ETC Source Four LED Series 2 Lustr mit 787 Rot: Spektrum} 
\end{figure}

\begin{figure}[htp]     % h=here, t=top, b=bottom, p=page
\centering
\includegraphics[width=0.5\textwidth]{hauptmessung/etc787cct} 
% Bilddatei aus dem Unterverzeichnis bilder holen, skalieren auf 0.8*Satzspiegel
\caption {ETC Source Four LED Series 2 Lustr mit 787 Rot: Beleuchtungsstärke, CCT, $\Delta$UV und Farbortkoordinaten} 
\end{figure}

\begin{figure}[htp]     % h=here, t=top, b=bottom, p=page
\centering
\includegraphics[width=0.9\textwidth]{hauptmessung/etc787xy} 
% Bilddatei aus dem Unterverzeichnis bilder holen, skalieren auf 0.8*Satzspiegel
\caption {ETC Source Four LED Series 2 Lustr mit 787 Rot: XYZ-Farbraum Detailansicht} 
\end{figure}

\begin{figure}[htp]     % h=here, t=top, b=bottom, p=page
\centering
\includegraphics[width=0.9\textwidth]{hauptmessung/etc787cri} 
% Bilddatei aus dem Unterverzeichnis bilder holen, skalieren auf 0.8*Satzspiegel
\caption {ETC Source Four LED Series 2 Lustr mit 787 Rot: CRI} 
\end{figure}

\begin{figure}[htp]     % h=here, t=top, b=bottom, p=page
\centering
\includegraphics[width=0.9\textwidth]{hauptmessung/etc787cqs} 
% Bilddatei aus dem Unterverzeichnis bilder holen, skalieren auf 0.8*Satzspiegel
\caption {ETC Source Four LED Series 2 Lustr mit 787 Rot: CQS} 
\end{figure}

\begin{figure}[htp]     % h=here, t=top, b=bottom, p=page
\centering
\includegraphics[width=0.9\textwidth]{hauptmessung/etc787tlci} 
% Bilddatei aus dem Unterverzeichnis bilder holen, skalieren auf 0.8*Satzspiegel
\caption {ETC Source Four LED Series 2 Lustr mit 787 Rot: TLCI} 
\end{figure}

\begin{figure}[htp]     % h=here, t=top, b=bottom, p=page
\centering
\includegraphics[width=0.9\textwidth]{hauptmessung/etc787tm} 
% Bilddatei aus dem Unterverzeichnis bilder holen, skalieren auf 0.8*Satzspiegel
\caption {ETC Source Four LED Series 2 Lustr mit 787 Rot: TM-30} 
\end{figure}


\subsection{Ayrton Ghibli}

\subsubsection{Ayrton Ghibli - alle LED auf 100}

\begin{figure}[htp]     % h=here, t=top, b=bottom, p=page
\centering
\includegraphics[width=0.9\textwidth]{hauptmessung/ghibli100spec} 
% Bilddatei aus dem Unterverzeichnis bilder holen, skalieren auf 0.8*Satzspiegel
\caption {Ayrton Ghibli alle LED auf 100: Spektrum} 
\end{figure}

\begin{figure}[htp]     % h=here, t=top, b=bottom, p=page
\centering
\includegraphics[width=0.5\textwidth]{hauptmessung/ghibli100cct} 
% Bilddatei aus dem Unterverzeichnis bilder holen, skalieren auf 0.8*Satzspiegel
\caption {Ayrton Ghibli alle LED auf 100: Beleuchtungsstärke, CCT, $\Delta$UV und Farbortkoordinaten} 
\end{figure}

\begin{figure}[htp]     % h=here, t=top, b=bottom, p=page
\centering
\includegraphics[width=0.9\textwidth]{hauptmessung/ghibli100xy} 
% Bilddatei aus dem Unterverzeichnis bilder holen, skalieren auf 0.8*Satzspiegel
\caption {Ayrton Ghibli alle LED auf 100: XYZ-Farbraum Detailansicht} 
\end{figure}

\begin{figure}[htp]     % h=here, t=top, b=bottom, p=page
\centering
\includegraphics[width=0.9\textwidth]{hauptmessung/ghibli100cri} 
% Bilddatei aus dem Unterverzeichnis bilder holen, skalieren auf 0.8*Satzspiegel
\caption {Ayrton Ghibli alle LED auf 100: CRI} 
\end{figure}

\begin{figure}[htp]     % h=here, t=top, b=bottom, p=page
\centering
\includegraphics[width=0.9\textwidth]{hauptmessung/ghibli100cqs} 
% Bilddatei aus dem Unterverzeichnis bilder holen, skalieren auf 0.8*Satzspiegel
\caption {Ayrton Ghibli alle LED auf 100: CQS} 
\end{figure}

\begin{figure}[htp]     % h=here, t=top, b=bottom, p=page
\centering
\includegraphics[width=0.9\textwidth]{hauptmessung/ghibli100tlci} 
% Bilddatei aus dem Unterverzeichnis bilder holen, skalieren auf 0.8*Satzspiegel
\caption {Ayrton Ghibli alle LED auf 100: TLCI} 
\end{figure}

\begin{figure}[htp]     % h=here, t=top, b=bottom, p=page
\centering
\includegraphics[width=0.9\textwidth]{hauptmessung/ghibli100tm} 
% Bilddatei aus dem Unterverzeichnis bilder holen, skalieren auf 0.8*Satzspiegel
\caption {Ayrton Ghibli alle LED auf 100: TM-30} 
\end{figure}

\subsubsection{Ayrton Ghibli ohne Rot}

\begin{figure}[htp]     % h=here, t=top, b=bottom, p=page
\centering
\includegraphics[width=0.9\textwidth]{hauptmessung/ghiblibestspec} 
% Bilddatei aus dem Unterverzeichnis bilder holen, skalieren auf 0.8*Satzspiegel
\caption {Ayrton Ghibli ohne Rot: Spektrum} 
\end{figure}

\begin{figure}[htp]     % h=here, t=top, b=bottom, p=page
\centering
\includegraphics[width=0.5\textwidth]{hauptmessung/ghiblibestcct} 
% Bilddatei aus dem Unterverzeichnis bilder holen, skalieren auf 0.8*Satzspiegel
\caption {Ayrton Ghibli ohne Rot: Beleuchtungsstärke, CCT, $\Delta$UV und Farbortkoordinaten} 
\end{figure}

\begin{figure}[htp]     % h=here, t=top, b=bottom, p=page
\centering
\includegraphics[width=0.9\textwidth]{hauptmessung/ghiblibestxy} 
% Bilddatei aus dem Unterverzeichnis bilder holen, skalieren auf 0.8*Satzspiegel
\caption {Ayrton Ghibli ohne Rot: XYZ-Farbraum Detailansicht} 
\end{figure}

\begin{figure}[htp]     % h=here, t=top, b=bottom, p=page
\centering
\includegraphics[width=0.9\textwidth]{hauptmessung/ghiblibestcri} 
% Bilddatei aus dem Unterverzeichnis bilder holen, skalieren auf 0.8*Satzspiegel
\caption {Ayrton Ghibli ohne Rot: CRI} 
\end{figure}

\begin{figure}[htp]     % h=here, t=top, b=bottom, p=page
\centering
\includegraphics[width=0.9\textwidth]{hauptmessung/ghiblibestcqs} 
% Bilddatei aus dem Unterverzeichnis bilder holen, skalieren auf 0.8*Satzspiegel
\caption {Ayrton Ghibli ohne Rot: CQS} 
\end{figure}

\begin{figure}[htp]     % h=here, t=top, b=bottom, p=page
\centering
\includegraphics[width=0.9\textwidth]{hauptmessung/ghiblibesttlci} 
% Bilddatei aus dem Unterverzeichnis bilder holen, skalieren auf 0.8*Satzspiegel
\caption {Ayrton Ghibli ohne Rot: TLCI} 
\end{figure}

\begin{figure}[htp]     % h=here, t=top, b=bottom, p=page
\centering
\includegraphics[width=0.9\textwidth]{hauptmessung/ghiblibesttm} 
% Bilddatei aus dem Unterverzeichnis bilder holen, skalieren auf 0.8*Satzspiegel
\caption {Ayrton Ghibli ohne Rot: TM-30} 
\end{figure}

\subsubsection{Ayrton Ghibli mit 027 Rot}

\begin{figure}[htp]     % h=here, t=top, b=bottom, p=page
\centering
\includegraphics[width=0.9\textwidth]{hauptmessung/ghibli027spec} 
% Bilddatei aus dem Unterverzeichnis bilder holen, skalieren auf 0.8*Satzspiegel
\caption {Ayrton Ghibli mit 027 Rot: Spektrum} 
\end{figure}

\begin{figure}[htp]     % h=here, t=top, b=bottom, p=page
\centering
\includegraphics[width=0.5\textwidth]{hauptmessung/ghibli027cct} 
% Bilddatei aus dem Unterverzeichnis bilder holen, skalieren auf 0.8*Satzspiegel
\caption {Ayrton Ghibli mit 027 Rot: Beleuchtungsstärke, CCT, $\Delta$UV und Farbortkoordinaten} 
\end{figure}

\begin{figure}[htp]     % h=here, t=top, b=bottom, p=page
\centering
\includegraphics[width=0.9\textwidth]{hauptmessung/ghibli027xy} 
% Bilddatei aus dem Unterverzeichnis bilder holen, skalieren auf 0.8*Satzspiegel
\caption {Ayrton Ghibli mit 027 Rot: XYZ-Farbraum Detailansicht} 
\end{figure}

\begin{figure}[htp]     % h=here, t=top, b=bottom, p=page
\centering
\includegraphics[width=0.9\textwidth]{hauptmessung/ghibli027cri} 
% Bilddatei aus dem Unterverzeichnis bilder holen, skalieren auf 0.8*Satzspiegel
\caption {Ayrton Ghibli mit 027 Rot: CRI} 
\end{figure}

\begin{figure}[htp]     % h=here, t=top, b=bottom, p=page
\centering
\includegraphics[width=0.9\textwidth]{hauptmessung/ghibli027cqs} 
% Bilddatei aus dem Unterverzeichnis bilder holen, skalieren auf 0.8*Satzspiegel
\caption {Ayrton Ghibli mit 027 Rot: CQS} 
\end{figure}

\begin{figure}[htp]     % h=here, t=top, b=bottom, p=page
\centering
\includegraphics[width=0.9\textwidth]{hauptmessung/ghibli027tlci} 
% Bilddatei aus dem Unterverzeichnis bilder holen, skalieren auf 0.8*Satzspiegel
\caption {Ayrton Ghibli mit 027 Rot: TLCI} 
\end{figure}

\begin{figure}[htp]     % h=here, t=top, b=bottom, p=page
\centering
\includegraphics[width=0.9\textwidth]{hauptmessung/ghibli027tm} 
% Bilddatei aus dem Unterverzeichnis bilder holen, skalieren auf 0.8*Satzspiegel
\caption {Ayrton Ghibli mit 027 Rot: TM-30} 
\end{figure}

\subsubsection{Ayrton Ghibli mit 787 Rot}

\begin{figure}[htp]     % h=here, t=top, b=bottom, p=page
\centering
\includegraphics[width=0.9\textwidth]{hauptmessung/ghibli787spec} 
% Bilddatei aus dem Unterverzeichnis bilder holen, skalieren auf 0.8*Satzspiegel
\caption {Ayrton Ghibli mit 787 Rot: Spektrum} 
\end{figure}

\begin{figure}[htp]     % h=here, t=top, b=bottom, p=page
\centering
\includegraphics[width=0.5\textwidth]{hauptmessung/ghibli787cct} 
% Bilddatei aus dem Unterverzeichnis bilder holen, skalieren auf 0.8*Satzspiegel
\caption {Ayrton Ghibli mit 787 Rot: Beleuchtungsstärke, CCT, $\Delta$UV und Farbortkoordinaten} 
\end{figure}

\begin{figure}[htp]     % h=here, t=top, b=bottom, p=page
\centering
\includegraphics[width=0.9\textwidth]{hauptmessung/ghibli787xy} 
% Bilddatei aus dem Unterverzeichnis bilder holen, skalieren auf 0.8*Satzspiegel
\caption {Ayrton Ghibli mit 787 Rot: XYZ-Farbraum Detailansicht} 
\end{figure}

\begin{figure}[htp]     % h=here, t=top, b=bottom, p=page
\centering
\includegraphics[width=0.9\textwidth]{hauptmessung/ghibli787cri} 
% Bilddatei aus dem Unterverzeichnis bilder holen, skalieren auf 0.8*Satzspiegel
\caption {Ayrton Ghibli mit 787 Rot: CRI} 
\end{figure}

\begin{figure}[htp]     % h=here, t=top, b=bottom, p=page
\centering
\includegraphics[width=0.9\textwidth]{hauptmessung/ghibli787cqs} 
% Bilddatei aus dem Unterverzeichnis bilder holen, skalieren auf 0.8*Satzspiegel
\caption {Ayrton Ghibli mit 787 Rot: CQS} 
\end{figure}

\begin{figure}[htp]     % h=here, t=top, b=bottom, p=page
\centering
\includegraphics[width=0.9\textwidth]{hauptmessung/ghibli787tlci} 
% Bilddatei aus dem Unterverzeichnis bilder holen, skalieren auf 0.8*Satzspiegel
\caption {Ayrton Ghibli mit 787 Rot: TLCI} 
\end{figure}

\begin{figure}[htp]     % h=here, t=top, b=bottom, p=page
\centering
\includegraphics[width=0.9\textwidth]{hauptmessung/ghibli787tm} 
% Bilddatei aus dem Unterverzeichnis bilder holen, skalieren auf 0.8*Satzspiegel
\caption {Ayrton Ghibli mit 787 Rot: TM-30} 
\end{figure}


\subsection{GLP Impression X4 L}

\subsubsection{GLP Impression X4 L - alle LED auf 100}

\begin{figure}[htp]     % h=here, t=top, b=bottom, p=page
\centering
\includegraphics[width=0.9\textwidth]{hauptmessung/glp100spec} 
% Bilddatei aus dem Unterverzeichnis bilder holen, skalieren auf 0.8*Satzspiegel
\caption {GLP Impression X4 L alle LED auf 100: Spektrum} 
\end{figure}

\begin{figure}[htp]     % h=here, t=top, b=bottom, p=page
\centering
\includegraphics[width=0.5\textwidth]{hauptmessung/glp100cct} 
% Bilddatei aus dem Unterverzeichnis bilder holen, skalieren auf 0.8*Satzspiegel
\caption {GLP Impression X4 L alle LED auf 100: Beleuchtungsstärke, CCT, $\Delta$UV und Farbortkoordinaten} 
\end{figure}

\begin{figure}[htp]     % h=here, t=top, b=bottom, p=page
\centering
\includegraphics[width=0.9\textwidth]{hauptmessung/glp100xy} 
% Bilddatei aus dem Unterverzeichnis bilder holen, skalieren auf 0.8*Satzspiegel
\caption {GLP Impression X4 L alle LED auf 100: XYZ-Farbraum Detailansicht} 
\end{figure}

\begin{figure}[htp]     % h=here, t=top, b=bottom, p=page
\centering
\includegraphics[width=0.9\textwidth]{hauptmessung/glp100cri} 
% Bilddatei aus dem Unterverzeichnis bilder holen, skalieren auf 0.8*Satzspiegel
\caption {GLP Impression X4 L alle LED auf 100: CRI} 
\end{figure}

\begin{figure}[htp]     % h=here, t=top, b=bottom, p=page
\centering
\includegraphics[width=0.9\textwidth]{hauptmessung/glp100cqs} 
% Bilddatei aus dem Unterverzeichnis bilder holen, skalieren auf 0.8*Satzspiegel
\caption {GLP Impression X4 L alle LED auf 100: CQS} 
\end{figure}

\begin{figure}[htp]     % h=here, t=top, b=bottom, p=page
\centering
\includegraphics[width=0.9\textwidth]{hauptmessung/glp100tlci} 
% Bilddatei aus dem Unterverzeichnis bilder holen, skalieren auf 0.8*Satzspiegel
\caption {GLP Impression X4 L alle LED auf 100: TLCI} 
\end{figure}

\begin{figure}[htp]     % h=here, t=top, b=bottom, p=page
\centering
\includegraphics[width=0.9\textwidth]{hauptmessung/glp100tm} 
% Bilddatei aus dem Unterverzeichnis bilder holen, skalieren auf 0.8*Satzspiegel
\caption {GLP Impression X4 L alle LED auf 100: TM-30} 
\end{figure}

\subsubsection{GLP Impression X4 L ohne Rot}

\begin{figure}[htp]     % h=here, t=top, b=bottom, p=page
\centering
\includegraphics[width=0.9\textwidth]{hauptmessung/glpbestspec} 
% Bilddatei aus dem Unterverzeichnis bilder holen, skalieren auf 0.8*Satzspiegel
\caption {GLP Impression X4 L ohne Rot: Spektrum} 
\end{figure}

\begin{figure}[htp]     % h=here, t=top, b=bottom, p=page
\centering
\includegraphics[width=0.5\textwidth]{hauptmessung/glpbestcct} 
% Bilddatei aus dem Unterverzeichnis bilder holen, skalieren auf 0.8*Satzspiegel
\caption {GLP Impression X4 L ohne Rot: Beleuchtungsstärke, CCT, $\Delta$UV und Farbortkoordinaten} 
\end{figure}

\begin{figure}[htp]     % h=here, t=top, b=bottom, p=page
\centering
\includegraphics[width=0.9\textwidth]{hauptmessung/glpbestxy} 
% Bilddatei aus dem Unterverzeichnis bilder holen, skalieren auf 0.8*Satzspiegel
\caption {GLP Impression X4 L ohne Rot: XYZ-Farbraum Detailansicht} 
\end{figure}

\begin{figure}[htp]     % h=here, t=top, b=bottom, p=page
\centering
\includegraphics[width=0.9\textwidth]{hauptmessung/glpbestcri} 
% Bilddatei aus dem Unterverzeichnis bilder holen, skalieren auf 0.8*Satzspiegel
\caption {GLP Impression X4 L ohne Rot: CRI} 
\end{figure}

\begin{figure}[htp]     % h=here, t=top, b=bottom, p=page
\centering
\includegraphics[width=0.9\textwidth]{hauptmessung/glpbestcqs} 
% Bilddatei aus dem Unterverzeichnis bilder holen, skalieren auf 0.8*Satzspiegel
\caption {GLP Impression X4 L ohne Rot: CQS} 
\end{figure}

\begin{figure}[htp]     % h=here, t=top, b=bottom, p=page
\centering
\includegraphics[width=0.9\textwidth]{hauptmessung/glpbesttlci} 
% Bilddatei aus dem Unterverzeichnis bilder holen, skalieren auf 0.8*Satzspiegel
\caption {GLP Impression X4 L ohne Rot: TLCI} 
\end{figure}

\begin{figure}[htp]     % h=here, t=top, b=bottom, p=page
\centering
\includegraphics[width=0.9\textwidth]{hauptmessung/glpbesttm} 
% Bilddatei aus dem Unterverzeichnis bilder holen, skalieren auf 0.8*Satzspiegel
\caption {GLP Impression X4 L ohne Rot: TM-30} 
\end{figure}

\subsubsection{GLP Impression X4 L mit 027 Rot}

\begin{figure}[htp]     % h=here, t=top, b=bottom, p=page
\centering
\includegraphics[width=0.9\textwidth]{hauptmessung/glp027spec} 
% Bilddatei aus dem Unterverzeichnis bilder holen, skalieren auf 0.8*Satzspiegel
\caption {GLP Impression X4 L mit 027 Rot: Spektrum} 
\end{figure}

\begin{figure}[htp]     % h=here, t=top, b=bottom, p=page
\centering
\includegraphics[width=0.5\textwidth]{hauptmessung/glp027cct} 
% Bilddatei aus dem Unterverzeichnis bilder holen, skalieren auf 0.8*Satzspiegel
\caption {GLP Impression X4 L mit 027 Rot: Beleuchtungsstärke, CCT, $\Delta$UV und Farbortkoordinaten} 
\end{figure}

\begin{figure}[htp]     % h=here, t=top, b=bottom, p=page
\centering
\includegraphics[width=0.9\textwidth]{hauptmessung/glp027xy} 
% Bilddatei aus dem Unterverzeichnis bilder holen, skalieren auf 0.8*Satzspiegel
\caption {GLP Impression X4 L mit 027 Rot: XYZ-Farbraum Detailansicht} 
\end{figure}

\begin{figure}[htp]     % h=here, t=top, b=bottom, p=page
\centering
\includegraphics[width=0.9\textwidth]{hauptmessung/glp027cri} 
% Bilddatei aus dem Unterverzeichnis bilder holen, skalieren auf 0.8*Satzspiegel
\caption {GLP Impression X4 L mit 027 Rot: CRI} 
\end{figure}

\begin{figure}[htp]     % h=here, t=top, b=bottom, p=page
\centering
\includegraphics[width=0.9\textwidth]{hauptmessung/glp027cqs} 
% Bilddatei aus dem Unterverzeichnis bilder holen, skalieren auf 0.8*Satzspiegel
\caption {GLP Impression X4 L mit 027 Rot: CQS} 
\end{figure}

\begin{figure}[htp]     % h=here, t=top, b=bottom, p=page
\centering
\includegraphics[width=0.9\textwidth]{hauptmessung/glp027tlci} 
% Bilddatei aus dem Unterverzeichnis bilder holen, skalieren auf 0.8*Satzspiegel
\caption {GLP Impression X4 L mit 027 Rot: TLCI} 
\end{figure}

\begin{figure}[htp]     % h=here, t=top, b=bottom, p=page
\centering
\includegraphics[width=0.9\textwidth]{hauptmessung/glp027tm} 
% Bilddatei aus dem Unterverzeichnis bilder holen, skalieren auf 0.8*Satzspiegel
\caption {GLP Impression X4 L mit 027 Rot: TM-30} 
\end{figure}

\subsubsection{GLP Impression X4 L mit 787 Rot}

\begin{figure}[htp]     % h=here, t=top, b=bottom, p=page
\centering
\includegraphics[width=0.9\textwidth]{hauptmessung/glp787spec} 
% Bilddatei aus dem Unterverzeichnis bilder holen, skalieren auf 0.8*Satzspiegel
\caption {GLP Impression X4 L mit 787 Rot: Spektrum} 
\end{figure}

\begin{figure}[htp]     % h=here, t=top, b=bottom, p=page
\centering
\includegraphics[width=0.5\textwidth]{hauptmessung/glp787cct} 
% Bilddatei aus dem Unterverzeichnis bilder holen, skalieren auf 0.8*Satzspiegel
\caption {GLP Impression X4 L mit 787 Rot: Beleuchtungsstärke, CCT, $\Delta$UV und Farbortkoordinaten} 
\end{figure}

\begin{figure}[htp]     % h=here, t=top, b=bottom, p=page
\centering
\includegraphics[width=0.9\textwidth]{hauptmessung/glp787xy} 
% Bilddatei aus dem Unterverzeichnis bilder holen, skalieren auf 0.8*Satzspiegel
\caption {GLP Impression X4 L mit 787 Rot: XYZ-Farbraum Detailansicht} 
\end{figure}

\begin{figure}[htp]     % h=here, t=top, b=bottom, p=page
\centering
\includegraphics[width=0.9\textwidth]{hauptmessung/glp787cri} 
% Bilddatei aus dem Unterverzeichnis bilder holen, skalieren auf 0.8*Satzspiegel
\caption {GLP Impression X4 L mit 787 Rot: CRI} 
\end{figure}

\begin{figure}[htp]     % h=here, t=top, b=bottom, p=page
\centering
\includegraphics[width=0.9\textwidth]{hauptmessung/glp787cqs} 
% Bilddatei aus dem Unterverzeichnis bilder holen, skalieren auf 0.8*Satzspiegel
\caption {GLP Impression X4 L mit 787 Rot: CQS} 
\end{figure}

\begin{figure}[htp]     % h=here, t=top, b=bottom, p=page
\centering
\includegraphics[width=0.9\textwidth]{hauptmessung/glp787tlci} 
% Bilddatei aus dem Unterverzeichnis bilder holen, skalieren auf 0.8*Satzspiegel
\caption {GLP Impression X4 L mit 787 Rot: TLCI} 
\end{figure}

\begin{figure}[htp]     % h=here, t=top, b=bottom, p=page
\centering
\includegraphics[width=0.9\textwidth]{hauptmessung/glp787tm} 
% Bilddatei aus dem Unterverzeichnis bilder holen, skalieren auf 0.8*Satzspiegel
\caption {GLP Impression X4 L mit 787 Rot: TM-30} 
\end{figure}

\subsection{Clay Paky K-Eye K20}

\subsubsection{Clay Paky K-Eye K20 - alle LED auf 100}

\begin{figure}[htp]     % h=here, t=top, b=bottom, p=page
\centering
\includegraphics[width=0.9\textwidth]{hauptmessung/keye100spec} 
% Bilddatei aus dem Unterverzeichnis bilder holen, skalieren auf 0.8*Satzspiegel
\caption {Clay Paky K-Eye K20 alle LED auf 100: Spektrum} 
\end{figure}

\begin{figure}[htp]     % h=here, t=top, b=bottom, p=page
\centering
\includegraphics[width=0.5\textwidth]{hauptmessung/keye100cct} 
% Bilddatei aus dem Unterverzeichnis bilder holen, skalieren auf 0.8*Satzspiegel
\caption {Clay Paky K-Eye K20 alle LED auf 100: Beleuchtungsstärke, CCT, $\Delta$UV und Farbortkoordinaten} 
\end{figure}

\begin{figure}[htp]     % h=here, t=top, b=bottom, p=page
\centering
\includegraphics[width=0.9\textwidth]{hauptmessung/keye100xy} 
% Bilddatei aus dem Unterverzeichnis bilder holen, skalieren auf 0.8*Satzspiegel
\caption {Clay Paky K-Eye K20 alle LED auf 100: XYZ-Farbraum Detailansicht} 
\end{figure}

\begin{figure}[htp]     % h=here, t=top, b=bottom, p=page
\centering
\includegraphics[width=0.9\textwidth]{hauptmessung/keye100cri} 
% Bilddatei aus dem Unterverzeichnis bilder holen, skalieren auf 0.8*Satzspiegel
\caption {Clay Paky K-Eye K20 alle LED auf 100: CRI} 
\end{figure}

\begin{figure}[htp]     % h=here, t=top, b=bottom, p=page
\centering
\includegraphics[width=0.9\textwidth]{hauptmessung/keye100cqs} 
% Bilddatei aus dem Unterverzeichnis bilder holen, skalieren auf 0.8*Satzspiegel
\caption {Clay Paky K-Eye K20 alle LED auf 100: CQS} 
\end{figure}

\begin{figure}[htp]     % h=here, t=top, b=bottom, p=page
\centering
\includegraphics[width=0.9\textwidth]{hauptmessung/keye100tlci} 
% Bilddatei aus dem Unterverzeichnis bilder holen, skalieren auf 0.8*Satzspiegel
\caption {Clay Paky K-Eye K20 alle LED auf 100: TLCI} 
\end{figure}

\begin{figure}[htp]     % h=here, t=top, b=bottom, p=page
\centering
\includegraphics[width=0.9\textwidth]{hauptmessung/keye100tm} 
% Bilddatei aus dem Unterverzeichnis bilder holen, skalieren auf 0.8*Satzspiegel
\caption {Clay Paky K-Eye K20 alle LED auf 100: TM-30} 
\end{figure}

\subsubsection{Clay Paky K-Eye K20 ohne Rot}

\begin{figure}[htp]     % h=here, t=top, b=bottom, p=page
\centering
\includegraphics[width=0.9\textwidth]{hauptmessung/keyebestspec} 
% Bilddatei aus dem Unterverzeichnis bilder holen, skalieren auf 0.8*Satzspiegel
\caption {Clay Paky K-Eye K20 ohne Rot: Spektrum} 
\end{figure}

\begin{figure}[htp]     % h=here, t=top, b=bottom, p=page
\centering
\includegraphics[width=0.5\textwidth]{hauptmessung/keyebestcct} 
% Bilddatei aus dem Unterverzeichnis bilder holen, skalieren auf 0.8*Satzspiegel
\caption {Clay Paky K-Eye K20 ohne Rot: Beleuchtungsstärke, CCT, $\Delta$UV und Farbortkoordinaten} 
\end{figure}

\begin{figure}[htp]     % h=here, t=top, b=bottom, p=page
\centering
\includegraphics[width=0.9\textwidth]{hauptmessung/keyebestxy} 
% Bilddatei aus dem Unterverzeichnis bilder holen, skalieren auf 0.8*Satzspiegel
\caption {Clay Paky K-Eye K20 ohne Rot: XYZ-Farbraum Detailansicht} 
\end{figure}

\begin{figure}[htp]     % h=here, t=top, b=bottom, p=page
\centering
\includegraphics[width=0.9\textwidth]{hauptmessung/keyebestcri} 
% Bilddatei aus dem Unterverzeichnis bilder holen, skalieren auf 0.8*Satzspiegel
\caption {Clay Paky K-Eye K20 ohne Rot: CRI} 
\end{figure}

\begin{figure}[htp]     % h=here, t=top, b=bottom, p=page
\centering
\includegraphics[width=0.9\textwidth]{hauptmessung/keyebestcqs} 
% Bilddatei aus dem Unterverzeichnis bilder holen, skalieren auf 0.8*Satzspiegel
\caption {Clay Paky K-Eye K20 ohne Rot: CQS} 
\end{figure}

\begin{figure}[htp]     % h=here, t=top, b=bottom, p=page
\centering
\includegraphics[width=0.9\textwidth]{hauptmessung/keyebesttlci} 
% Bilddatei aus dem Unterverzeichnis bilder holen, skalieren auf 0.8*Satzspiegel
\caption {Clay Paky K-Eye K20 ohne Rot: TLCI} 
\end{figure}

\begin{figure}[htp]     % h=here, t=top, b=bottom, p=page
\centering
\includegraphics[width=0.9\textwidth]{hauptmessung/keyebesttm} 
% Bilddatei aus dem Unterverzeichnis bilder holen, skalieren auf 0.8*Satzspiegel
\caption {Clay Paky K-Eye K20 ohne Rot: TM-30} 
\end{figure}

\subsubsection{Clay Paky K-Eye K20 mit 027 Rot}

\begin{figure}[htp]     % h=here, t=top, b=bottom, p=page
\centering
\includegraphics[width=0.9\textwidth]{hauptmessung/keye027spec} 
% Bilddatei aus dem Unterverzeichnis bilder holen, skalieren auf 0.8*Satzspiegel
\caption {Clay Paky K-Eye K20 mit 027 Rot: Spektrum} 
\end{figure}

\begin{figure}[htp]     % h=here, t=top, b=bottom, p=page
\centering
\includegraphics[width=0.5\textwidth]{hauptmessung/keye027cct} 
% Bilddatei aus dem Unterverzeichnis bilder holen, skalieren auf 0.8*Satzspiegel
\caption {Clay Paky K-Eye K20 mit 027 Rot: Beleuchtungsstärke, CCT, $\Delta$UV und Farbortkoordinaten} 
\end{figure}

\begin{figure}[htp]     % h=here, t=top, b=bottom, p=page
\centering
\includegraphics[width=0.9\textwidth]{hauptmessung/keye027xy} 
% Bilddatei aus dem Unterverzeichnis bilder holen, skalieren auf 0.8*Satzspiegel
\caption {Clay Paky K-Eye K20 mit 027 Rot: XYZ-Farbraum Detailansicht} 
\end{figure}

\begin{figure}[htp]     % h=here, t=top, b=bottom, p=page
\centering
\includegraphics[width=0.9\textwidth]{hauptmessung/keye027cri} 
% Bilddatei aus dem Unterverzeichnis bilder holen, skalieren auf 0.8*Satzspiegel
\caption {Clay Paky K-Eye K20 mit 027 Rot: CRI} 
\end{figure}

\begin{figure}[htp]     % h=here, t=top, b=bottom, p=page
\centering
\includegraphics[width=0.9\textwidth]{hauptmessung/keye027cqs} 
% Bilddatei aus dem Unterverzeichnis bilder holen, skalieren auf 0.8*Satzspiegel
\caption {Clay Paky K-Eye K20 mit 027 Rot: CQS} 
\end{figure}

\begin{figure}[htp]     % h=here, t=top, b=bottom, p=page
\centering
\includegraphics[width=0.9\textwidth]{hauptmessung/keye027tlci} 
% Bilddatei aus dem Unterverzeichnis bilder holen, skalieren auf 0.8*Satzspiegel
\caption {Clay Paky K-Eye K20 mit 027 Rot: TLCI} 
\end{figure}

\begin{figure}[htp]     % h=here, t=top, b=bottom, p=page
\centering
\includegraphics[width=0.9\textwidth]{hauptmessung/keye027tm} 
% Bilddatei aus dem Unterverzeichnis bilder holen, skalieren auf 0.8*Satzspiegel
\caption {Clay Paky K-Eye K20 mit 027 Rot: TM-30} 
\end{figure}

\subsubsection{Clay Paky K-Eye K20 mit 787 Rot}

\begin{figure}[htp]     % h=here, t=top, b=bottom, p=page
\centering
\includegraphics[width=0.9\textwidth]{hauptmessung/keye787spec} 
% Bilddatei aus dem Unterverzeichnis bilder holen, skalieren auf 0.8*Satzspiegel
\caption {Clay Paky K-Eye K20 mit 787 Rot: Spektrum} 
\end{figure}

\begin{figure}[htp]     % h=here, t=top, b=bottom, p=page
\centering
\includegraphics[width=0.5\textwidth]{hauptmessung/keye787cct} 
% Bilddatei aus dem Unterverzeichnis bilder holen, skalieren auf 0.8*Satzspiegel
\caption {Clay Paky K-Eye K20 mit 787 Rot: Beleuchtungsstärke, CCT, $\Delta$UV und Farbortkoordinaten} 
\end{figure}

\begin{figure}[htp]     % h=here, t=top, b=bottom, p=page
\centering
\includegraphics[width=0.9\textwidth]{hauptmessung/keye787xy} 
% Bilddatei aus dem Unterverzeichnis bilder holen, skalieren auf 0.8*Satzspiegel
\caption {Clay Paky K-Eye K20 mit 787 Rot: XYZ-Farbraum Detailansicht} 
\end{figure}

\begin{figure}[htp]     % h=here, t=top, b=bottom, p=page
\centering
\includegraphics[width=0.9\textwidth]{hauptmessung/keye787cri} 
% Bilddatei aus dem Unterverzeichnis bilder holen, skalieren auf 0.8*Satzspiegel
\caption {Clay Paky K-Eye K20 mit 787 Rot: CRI} 
\end{figure}

\begin{figure}[htp]     % h=here, t=top, b=bottom, p=page
\centering
\includegraphics[width=0.9\textwidth]{hauptmessung/keye787cqs} 
% Bilddatei aus dem Unterverzeichnis bilder holen, skalieren auf 0.8*Satzspiegel
\caption {Clay Paky K-Eye K20 mit 787 Rot: CQS} 
\end{figure}

\begin{figure}[htp]     % h=here, t=top, b=bottom, p=page
\centering
\includegraphics[width=0.9\textwidth]{hauptmessung/keye787tlci} 
% Bilddatei aus dem Unterverzeichnis bilder holen, skalieren auf 0.8*Satzspiegel
\caption {Clay Paky K-Eye K20 mit 787 Rot: TLCI} 
\end{figure}

\begin{figure}[htp]     % h=here, t=top, b=bottom, p=page
\centering
\includegraphics[width=0.9\textwidth]{hauptmessung/keye787tm} 
% Bilddatei aus dem Unterverzeichnis bilder holen, skalieren auf 0.8*Satzspiegel
\caption {Clay Paky K-Eye K20 mit 787 Rot: TM-30} 
\end{figure}


\section{UPRTek-Messungen}

\subsection{Arri D5}

\begin{figure}[htp]     % h=here, t=top, b=bottom, p=page
\centering
\includegraphics[width=0.5\textwidth]{upr/uprarri} 
% Bilddatei aus dem Unterverzeichnis bilder holen, skalieren auf 0.8*Satzspiegel
\caption {Arri D5: UPRTek-Messung} 
\end{figure}


\subsection{Robe DL7F Wash}

\begin{figure}[htp]     % h=here, t=top, b=bottom, p=page
\centering
\includegraphics[width=0.5\textwidth]{upr/uprdl7f100} 
% Bilddatei aus dem Unterverzeichnis bilder holen, skalieren auf 0.8*Satzspiegel
\caption {Robe DL7F Wash alle LED auf 100: UPRTek-Messung} 
\end{figure}

\begin{figure}[htp]     % h=here, t=top, b=bottom, p=page
\centering
\includegraphics[width=0.5\textwidth]{upr/uprdl7fbest} 
% Bilddatei aus dem Unterverzeichnis bilder holen, skalieren auf 0.8*Satzspiegel
\caption {Robe DL7F Wash ohne Rot: UPRTek-Messung} 
\end{figure}

\begin{figure}[htp]     % h=here, t=top, b=bottom, p=page
\centering
\includegraphics[width=0.5\textwidth]{upr/uprdl7f027} 
% Bilddatei aus dem Unterverzeichnis bilder holen, skalieren auf 0.8*Satzspiegel
\caption {Robe DL7F Wash mit 027 Rot: UPRTek-Messung} 
\end{figure}

\begin{figure}[htp]     % h=here, t=top, b=bottom, p=page
\centering
\includegraphics[width=0.5\textwidth]{upr/uprdl7f787} 
% Bilddatei aus dem Unterverzeichnis bilder holen, skalieren auf 0.8*Satzspiegel
\caption {Robe DL7F Wash mit 787 Rot: UPRTek-Messung} 
\end{figure}


\subsection{Martin MAC Encore Wash CLD}

\begin{figure}[htp]     % h=here, t=top, b=bottom, p=page
\centering
\includegraphics[width=0.5\textwidth]{upr/uprencore100} 
% Bilddatei aus dem Unterverzeichnis bilder holen, skalieren auf 0.8*Satzspiegel
\caption {Martin MAC Encore Wash CLD alle LED auf 100: UPRTek-Messung} 
\end{figure}

\begin{figure}[htp]     % h=here, t=top, b=bottom, p=page
\centering
\includegraphics[width=0.5\textwidth]{upr/uprencorebest} 
% Bilddatei aus dem Unterverzeichnis bilder holen, skalieren auf 0.8*Satzspiegel
\caption {Martin MAC Encore Wash CLD ohne Rot: UPRTek-Messung} 
\end{figure}

\begin{figure}[htp]     % h=here, t=top, b=bottom, p=page
\centering
\includegraphics[width=0.5\textwidth]{upr/uprencore027} 
% Bilddatei aus dem Unterverzeichnis bilder holen, skalieren auf 0.8*Satzspiegel
\caption {Martin MAC Encore Wash CLD mit 027 Rot: UPRTek-Messung} 
\end{figure}

\begin{figure}[htp]     % h=here, t=top, b=bottom, p=page
\centering
\includegraphics[width=0.5\textwidth]{upr/uprencore787} 
% Bilddatei aus dem Unterverzeichnis bilder holen, skalieren auf 0.8*Satzspiegel
\caption {Martin MAC Encore Wash CLD mit 787 Rot: UPRTek-Messung} 
\end{figure}


\subsection{ETC Source Four LED Series 2 Lustr}

\begin{figure}[htp]     % h=here, t=top, b=bottom, p=page
\centering
\includegraphics[width=0.5\textwidth]{upr/upretc100} 
% Bilddatei aus dem Unterverzeichnis bilder holen, skalieren auf 0.8*Satzspiegel
\caption {ETC Source Four LED Series 2 Lustr alle LED auf 100: UPRTek-Messung} 
\end{figure}

\begin{figure}[htp]     % h=here, t=top, b=bottom, p=page
\centering
\includegraphics[width=0.5\textwidth]{upr/upretcbest} 
% Bilddatei aus dem Unterverzeichnis bilder holen, skalieren auf 0.8*Satzspiegel
\caption {ETC Source Four LED Series 2 Lustr ohne Rot: UPRTek-Messung} 
\end{figure}

\begin{figure}[htp]     % h=here, t=top, b=bottom, p=page
\centering
\includegraphics[width=0.5\textwidth]{upr/upretc027} 
% Bilddatei aus dem Unterverzeichnis bilder holen, skalieren auf 0.8*Satzspiegel
\caption {ETC Source Four LED Series 2 Lustr mit 027 Rot: UPRTek-Messung} 
\end{figure}

\begin{figure}[htp]     % h=here, t=top, b=bottom, p=page
\centering
\includegraphics[width=0.5\textwidth]{upr/upretc787} 
% Bilddatei aus dem Unterverzeichnis bilder holen, skalieren auf 0.8*Satzspiegel
\caption {ETC Source Four LED Series 2 Lustr mit 787 Rot: UPRTek-Messung} 
\end{figure}



\subsection{Ayrton Ghibli}

\begin{figure}[htp]     % h=here, t=top, b=bottom, p=page
\centering
\includegraphics[width=0.5\textwidth]{upr/uprghiblibest} 
% Bilddatei aus dem Unterverzeichnis bilder holen, skalieren auf 0.8*Satzspiegel
\caption {Ayrton Ghibli ohne Rot: UPRTek-Messung} 
\end{figure}

\begin{figure}[htp]     % h=here, t=top, b=bottom, p=page
\centering
\includegraphics[width=0.5\textwidth]{upr/uprghibli027} 
% Bilddatei aus dem Unterverzeichnis bilder holen, skalieren auf 0.8*Satzspiegel
\caption {Ayrton Ghibli mit 027 Rot: UPRTek-Messung} 
\end{figure}

\begin{figure}[htp]     % h=here, t=top, b=bottom, p=page
\centering
\includegraphics[width=0.5\textwidth]{upr/uprghibli787} 
% Bilddatei aus dem Unterverzeichnis bilder holen, skalieren auf 0.8*Satzspiegel
\caption {Ayrton Ghibli mit 787 Rot: UPRTek-Messung} 
\end{figure}



\subsection{GLP Impression X4 L}

\begin{figure}[htp]     % h=here, t=top, b=bottom, p=page
\centering
\includegraphics[width=0.5\textwidth]{upr/uprglp100} 
% Bilddatei aus dem Unterverzeichnis bilder holen, skalieren auf 0.8*Satzspiegel
\caption {GLP Impression X4 L alle LED auf 100: UPRTek-Messung} 
\end{figure}

\begin{figure}[htp]     % h=here, t=top, b=bottom, p=page
\centering
\includegraphics[width=0.5\textwidth]{upr/uprglpbest} 
% Bilddatei aus dem Unterverzeichnis bilder holen, skalieren auf 0.8*Satzspiegel
\caption {GLP Impression X4 L ohne Rot: UPRTek-Messung} 
\end{figure}

\begin{figure}[htp]     % h=here, t=top, b=bottom, p=page
\centering
\includegraphics[width=0.5\textwidth]{upr/uprglp027} 
% Bilddatei aus dem Unterverzeichnis bilder holen, skalieren auf 0.8*Satzspiegel
\caption {GLP Impression X4 L mit 027 Rot: UPRTek-Messung} 
\end{figure}

\begin{figure}[htp]     % h=here, t=top, b=bottom, p=page
\centering
\includegraphics[width=0.5\textwidth]{upr/uprglp787} 
% Bilddatei aus dem Unterverzeichnis bilder holen, skalieren auf 0.8*Satzspiegel
\caption {GLP Impression X4 L mit 787 Rot: UPRTek-Messung} 
\end{figure}



\subsection{Clay Paky K-Eye K20}

\begin{figure}[htp]     % h=here, t=top, b=bottom, p=page
\centering
\includegraphics[width=0.5\textwidth]{upr/uprkeye100} 
% Bilddatei aus dem Unterverzeichnis bilder holen, skalieren auf 0.8*Satzspiegel
\caption {Clay Paky K-Eye K20 alle LED auf 100: UPRTek-Messung} 
\end{figure}

\begin{figure}[htp]     % h=here, t=top, b=bottom, p=page
\centering
\includegraphics[width=0.5\textwidth]{upr/uprkeyebest} 
% Bilddatei aus dem Unterverzeichnis bilder holen, skalieren auf 0.8*Satzspiegel
\caption {Clay Paky K-Eye K20 ohne Rot: UPRTek-Messung} 
\end{figure}

\begin{figure}[htp]     % h=here, t=top, b=bottom, p=page
\centering
\includegraphics[width=0.5\textwidth]{upr/uprkeye027} 
% Bilddatei aus dem Unterverzeichnis bilder holen, skalieren auf 0.8*Satzspiegel
\caption {Clay Paky K-Eye K20 mit 027 Rot: UPRTek-Messung} 
\end{figure}

\begin{figure}[htp]     % h=here, t=top, b=bottom, p=page
\centering
\includegraphics[width=0.5\textwidth]{upr/uprkeye787} 
% Bilddatei aus dem Unterverzeichnis bilder holen, skalieren auf 0.8*Satzspiegel
\caption {Clay Paky K-Eye K20 mit 787 Rot: UPRTek-Messung} 
\end{figure}




\section{Messungen Videos}

\subsection{ Robe DL7F Wash}

\begin{figure}[htp]     % h=here, t=top, b=bottom, p=page
\centering
\includegraphics[width=0.9\textwidth]{K Werte/dl7fkbestspec} 
% Bilddatei aus dem Unterverzeichnis bilder holen, skalieren auf 0.8*Satzspiegel
\caption { Robe DL7F Wash ohne Rot: Spektrum} 
\end{figure}

\begin{figure}[htp]     % h=here, t=top, b=bottom, p=page
\centering
\includegraphics[width=0.9\textwidth]{K Werte/dl7fk027spec} 
% Bilddatei aus dem Unterverzeichnis bilder holen, skalieren auf 0.8*Satzspiegel
\caption { Robe DL7F Wash mit 027 Rot: Spektrum} 
\end{figure}

\begin{figure}[htp]     % h=here, t=top, b=bottom, p=page
\centering
\includegraphics[width=0.9\textwidth]{K Werte/dl7fk787spec} 
% Bilddatei aus dem Unterverzeichnis bilder holen, skalieren auf 0.8*Satzspiegel
\caption { Robe DL7F Wash mit 787 Rot: Spektrum} 
\end{figure}


\subsection{Martin MAC Encore Wash CLD}

\begin{figure}[htp]     % h=here, t=top, b=bottom, p=page
\centering
\includegraphics[width=0.9\textwidth]{K Werte/encorekbestspec} 
% Bilddatei aus dem Unterverzeichnis bilder holen, skalieren auf 0.8*Satzspiegel
\caption {Martin MAC Encore Wash CLD ohne Rot: Spektrum} 
\end{figure}

\begin{figure}[htp]     % h=here, t=top, b=bottom, p=page
\centering
\includegraphics[width=0.9\textwidth]{K Werte/encorek027spec} 
% Bilddatei aus dem Unterverzeichnis bilder holen, skalieren auf 0.8*Satzspiegel
\caption {Martin MAC Encore Wash CLD mit 027 Rot: Spektrum} 
\end{figure}

\begin{figure}[htp]     % h=here, t=top, b=bottom, p=page
\centering
\includegraphics[width=0.9\textwidth]{K Werte/encorek787spec} 
% Bilddatei aus dem Unterverzeichnis bilder holen, skalieren auf 0.8*Satzspiegel
\caption {Martin MAC Encore Wash CLD mit 787 Rot: Spektrum} 
\end{figure}

\subsection{ETC Source Four LED Series 2 Lustr}

\begin{figure}[htp]     % h=here, t=top, b=bottom, p=page
\centering
\includegraphics[width=0.9\textwidth]{K Werte/etckbestspec} 
% Bilddatei aus dem Unterverzeichnis bilder holen, skalieren auf 0.8*Satzspiegel
\caption {ETC Source Four LED Series 2 Lustr ohne Rot: Spektrum} 
\end{figure}

\begin{figure}[htp]     % h=here, t=top, b=bottom, p=page
\centering
\includegraphics[width=0.9\textwidth]{K Werte/etck027spec} 
% Bilddatei aus dem Unterverzeichnis bilder holen, skalieren auf 0.8*Satzspiegel
\caption {ETC Source Four LED Series 2 Lustr mit 027 Rot: Spektrum} 
\end{figure}

\begin{figure}[htp]     % h=here, t=top, b=bottom, p=page
\centering
\includegraphics[width=0.9\textwidth]{K Werte/etck787spec} 
% Bilddatei aus dem Unterverzeichnis bilder holen, skalieren auf 0.8*Satzspiegel
\caption {ETC Source Four LED Series 2 Lustr mit 787 Rot: Spektrum} 
\end{figure}


\subsection{Ayrton Ghibli}

\begin{figure}[htp]     % h=here, t=top, b=bottom, p=page
\centering
\includegraphics[width=0.9\textwidth]{K Werte/ghiblikbestspec} 
% Bilddatei aus dem Unterverzeichnis bilder holen, skalieren auf 0.8*Satzspiegel
\caption {Ayrton Ghibli ohne Rot: Spektrum} 
\end{figure}

\begin{figure}[htp]     % h=here, t=top, b=bottom, p=page
\centering
\includegraphics[width=0.9\textwidth]{K Werte/ghiblik027spec} 
% Bilddatei aus dem Unterverzeichnis bilder holen, skalieren auf 0.8*Satzspiegel
\caption {Ayrton Ghibli mit 027 Rot: Spektrum} 
\end{figure}

\begin{figure}[htp]     % h=here, t=top, b=bottom, p=page
\centering
\includegraphics[width=0.9\textwidth]{K Werte/ghiblik787spec} 
% Bilddatei aus dem Unterverzeichnis bilder holen, skalieren auf 0.8*Satzspiegel
\caption {Ayrton Ghibli mit 787 Rot: Spektrum} 
\end{figure}


\subsection{GLP Impression X4 L}

\begin{figure}[htp]     % h=here, t=top, b=bottom, p=page
\centering
\includegraphics[width=0.9\textwidth]{K Werte/glpkbestspec} 
% Bilddatei aus dem Unterverzeichnis bilder holen, skalieren auf 0.8*Satzspiegel
\caption {GLP Impression X4 L ohne Rot: Spektrum} 
\end{figure}

\begin{figure}[htp]     % h=here, t=top, b=bottom, p=page
\centering
\includegraphics[width=0.9\textwidth]{K Werte/glpk027spec} 
% Bilddatei aus dem Unterverzeichnis bilder holen, skalieren auf 0.8*Satzspiegel
\caption {GLP Impression X4 L mit 027 Rot: Spektrum} 
\end{figure}

\begin{figure}[htp]     % h=here, t=top, b=bottom, p=page
\centering
\includegraphics[width=0.9\textwidth]{K Werte/glpk787spec} 
% Bilddatei aus dem Unterverzeichnis bilder holen, skalieren auf 0.8*Satzspiegel
\caption {GLP Impression X4 L mit 787 Rot: Spektrum} 
\end{figure}


\subsection{Clay Paky K-Eye K20}

\begin{figure}[htp]     % h=here, t=top, b=bottom, p=page
\centering
\includegraphics[width=0.9\textwidth]{K Werte/keyekbestspec} 
% Bilddatei aus dem Unterverzeichnis bilder holen, skalieren auf 0.8*Satzspiegel
\caption {Clay Paky K-Eye K20 ohne Rot: Spektrum} 
\end{figure}

\begin{figure}[htp]     % h=here, t=top, b=bottom, p=page
\centering
\includegraphics[width=0.9\textwidth]{K Werte/keyek027spec} 
% Bilddatei aus dem Unterverzeichnis bilder holen, skalieren auf 0.8*Satzspiegel
\caption {Clay Paky K-Eye K20 mit 027 Rot: Spektrum} 
\end{figure}

\begin{figure}[htp]     % h=here, t=top, b=bottom, p=page
\centering
\includegraphics[width=0.9\textwidth]{K Werte/keyek787spec} 
% Bilddatei aus dem Unterverzeichnis bilder holen, skalieren auf 0.8*Satzspiegel
\caption {Clay Paky K-Eye K20 mit 787 Rot: Spektrum} 
\end{figure}


\section{Farbtafel}

\subsubsection{Arri D5}

\begin{figure}[htp]     % h=here, t=top, b=bottom, p=page
\centering
\includegraphics[width=0.9\textwidth]{Farbtafel/farbtafelarriD5basisWB} 
% Bilddatei aus dem Unterverzeichnis bilder holen, skalieren auf 0.8*Satzspiegel
\caption {Arri D5: Farbtafel} 
\end{figure}

\begin{figure}[htp]     % h=here, t=top, b=bottom, p=page
\centering
\includegraphics[width=0.9\textwidth]{FarbtafelWFM/farbtafelWFMarriD5basisWB} 
% Bilddatei aus dem Unterverzeichnis bilder holen, skalieren auf 0.8*Satzspiegel
\caption {Arri D5: Vektorskop Farbtafel} 
\end{figure}



\subsubsection{Martin MAC Encore Wash CLD}

\begin{figure}[htp]     % h=here, t=top, b=bottom, p=page
\centering
\includegraphics[width=0.9\textwidth]{Farbtafel/farbtafelencorebestbasisWB} 
% Bilddatei aus dem Unterverzeichnis bilder holen, skalieren auf 0.8*Satzspiegel
\caption {Martin MAC Encore Wash CLD ohne Rot (Basis WB): Farbtafel} 
\end{figure}

\begin{figure}[htp]     % h=here, t=top, b=bottom, p=page
\centering
\includegraphics[width=0.9\textwidth]{FarbtafelWFM/farbtafelWFMencorebestbasisWB} 
% Bilddatei aus dem Unterverzeichnis bilder holen, skalieren auf 0.8*Satzspiegel
\caption {Martin MAC Encore Wash CLD ohne Rot (Basis WB): WFM Farbtafel} 
\end{figure}

\begin{figure}[htp]     % h=here, t=top, b=bottom, p=page
\centering
\includegraphics[width=0.9\textwidth]{Farbtafel/farbtafelencorebestDMXWB} 
% Bilddatei aus dem Unterverzeichnis bilder holen, skalieren auf 0.8*Satzspiegel
\caption {Martin MAC Encore Wash CLD ohne Rot (DMX WB): Farbtafel} 
\end{figure}

\begin{figure}[htp]     % h=here, t=top, b=bottom, p=page
\centering
\includegraphics[width=0.9\textwidth]{FarbtafelWFM/farbtafelWFMencorebestDMXWB} 
% Bilddatei aus dem Unterverzeichnis bilder holen, skalieren auf 0.8*Satzspiegel
\caption {Martin MAC Encore Wash CLD ohne Rot (DMX WB): WFM Farbtafel} 
\end{figure}

\begin{figure}[htp]     % h=here, t=top, b=bottom, p=page
\centering
\includegraphics[width=0.9\textwidth]{Farbtafel/farbtafelencorebestVideoWB} 
% Bilddatei aus dem Unterverzeichnis bilder holen, skalieren auf 0.8*Satzspiegel
\caption {Martin MAC Encore Wash CLD ohne Rot (Video WB): Farbtafel} 
\end{figure}

\begin{figure}[htp]     % h=here, t=top, b=bottom, p=page
\centering
\includegraphics[width=0.9\textwidth]{FarbtafelWFM/farbtafelWFMencorebestVideoWB} 
% Bilddatei aus dem Unterverzeichnis bilder holen, skalieren auf 0.8*Satzspiegel
\caption {Martin MAC Encore Wash CLD ohne Rot (Video WB): WFM Farbtafel} 
\end{figure}




\begin{figure}[htp]     % h=here, t=top, b=bottom, p=page
\centering
\includegraphics[width=0.9\textwidth]{Farbtafel/farbtafelencore027basisWB} 
% Bilddatei aus dem Unterverzeichnis bilder holen, skalieren auf 0.8*Satzspiegel
\caption {Martin MAC Encore Wash CLD mit 027 Rot (Basis WB): Farbtafel} 
\end{figure}

\begin{figure}[htp]     % h=here, t=top, b=bottom, p=page
\centering
\includegraphics[width=0.9\textwidth]{FarbtafelWFM/farbtafelWFMencore027basisWB} 
% Bilddatei aus dem Unterverzeichnis bilder holen, skalieren auf 0.8*Satzspiegel
\caption {Martin MAC Encore Wash CLD mit 027 Rot (Basis WB): WFM Farbtafel} 
\end{figure}

\begin{figure}[htp]     % h=here, t=top, b=bottom, p=page
\centering
\includegraphics[width=0.9\textwidth]{Farbtafel/farbtafelencore027DMXWB} 
% Bilddatei aus dem Unterverzeichnis bilder holen, skalieren auf 0.8*Satzspiegel
\caption {Martin MAC Encore Wash CLD mit 027 Rot (DMX WB): Farbtafel} 
\end{figure}

\begin{figure}[htp]     % h=here, t=top, b=bottom, p=page
\centering
\includegraphics[width=0.9\textwidth]{FarbtafelWFM/farbtafelWFMencore027DMXWB} 
% Bilddatei aus dem Unterverzeichnis bilder holen, skalieren auf 0.8*Satzspiegel
\caption {Martin MAC Encore Wash CLD mit 027 Rot (DMX WB): WFM Farbtafel} 
\end{figure}

\begin{figure}[htp]     % h=here, t=top, b=bottom, p=page
\centering
\includegraphics[width=0.9\textwidth]{Farbtafel/farbtafelencore027VideoWB} 
% Bilddatei aus dem Unterverzeichnis bilder holen, skalieren auf 0.8*Satzspiegel
\caption {Martin MAC Encore Wash CLD mit 027 Rot (Video WB): Farbtafel} 
\end{figure}

\begin{figure}[htp]     % h=here, t=top, b=bottom, p=page
\centering
\includegraphics[width=0.9\textwidth]{FarbtafelWFM/farbtafelWFMencore027VideoWB} 
% Bilddatei aus dem Unterverzeichnis bilder holen, skalieren auf 0.8*Satzspiegel
\caption {Martin MAC Encore Wash CLD mit 027 Rot (Video WB): WFM Farbtafel} 
\end{figure}



\begin{figure}[htp]     % h=here, t=top, b=bottom, p=page
\centering
\includegraphics[width=0.9\textwidth]{Farbtafel/farbtafelencore787basisWB} 
% Bilddatei aus dem Unterverzeichnis bilder holen, skalieren auf 0.8*Satzspiegel
\caption {Martin MAC Encore Wash CLD mit 787 Rot (Basis WB): Farbtafel} 
\end{figure}

\begin{figure}[htp]     % h=here, t=top, b=bottom, p=page
\centering
\includegraphics[width=0.9\textwidth]{FarbtafelWFM/farbtafelWFMencore787basisWB} 
% Bilddatei aus dem Unterverzeichnis bilder holen, skalieren auf 0.8*Satzspiegel
\caption {Martin MAC Encore Wash CLD mit 787 Rot (Basis WB): WFM Farbtafel} 
\end{figure}

\begin{figure}[htp]     % h=here, t=top, b=bottom, p=page
\centering
\includegraphics[width=0.9\textwidth]{Farbtafel/farbtafelencore787DMXWB} 
% Bilddatei aus dem Unterverzeichnis bilder holen, skalieren auf 0.8*Satzspiegel
\caption {Martin MAC Encore Wash CLD mit 787 Rot (DMX WB): Farbtafel} 
\end{figure}

\begin{figure}[htp]     % h=here, t=top, b=bottom, p=page
\centering
\includegraphics[width=0.9\textwidth]{FarbtafelWFM/farbtafelWFMencore787DMXWB} 
% Bilddatei aus dem Unterverzeichnis bilder holen, skalieren auf 0.8*Satzspiegel
\caption {Martin MAC Encore Wash CLD mit 787 Rot (DMX WB): WFM Farbtafel} 
\end{figure}

\begin{figure}[htp]     % h=here, t=top, b=bottom, p=page
\centering
\includegraphics[width=0.9\textwidth]{Farbtafel/farbtafelencore787VideoWB} 
% Bilddatei aus dem Unterverzeichnis bilder holen, skalieren auf 0.8*Satzspiegel
\caption {Martin MAC Encore Wash CLD mit 787 Rot (Video WB): Farbtafel} 
\end{figure}

\begin{figure}[htp]     % h=here, t=top, b=bottom, p=page
\centering
\includegraphics[width=0.9\textwidth]{FarbtafelWFM/farbtafelWFMencore787VideoWB} 
% Bilddatei aus dem Unterverzeichnis bilder holen, skalieren auf 0.8*Satzspiegel
\caption {Martin MAC Encore Wash CLD mit 787 Rot (Video WB): WFM Farbtafel} 
\end{figure}



\subsubsection{Clay Paky K-Eye K20}

\begin{figure}[htp]     % h=here, t=top, b=bottom, p=page
\centering
\includegraphics[width=0.9\textwidth]{Farbtafel/farbtafelkeyebestbasisWB} 
% Bilddatei aus dem Unterverzeichnis bilder holen, skalieren auf 0.8*Satzspiegel
\caption {Clay Paky K-Eye K20 ohne Rot (Basis WB): Farbtafel} 
\end{figure}

\begin{figure}[htp]     % h=here, t=top, b=bottom, p=page
\centering
\includegraphics[width=0.9\textwidth]{FarbtafelWFM/farbtafelWFMkeyebestbasisWB} 
% Bilddatei aus dem Unterverzeichnis bilder holen, skalieren auf 0.8*Satzspiegel
\caption {Clay Paky K-Eye K20 ohne Rot (Basis WB): WFM Farbtafel} 
\end{figure}

\begin{figure}[htp]     % h=here, t=top, b=bottom, p=page
\centering
\includegraphics[width=0.9\textwidth]{Farbtafel/farbtafelkeyebestDMXWB} 
% Bilddatei aus dem Unterverzeichnis bilder holen, skalieren auf 0.8*Satzspiegel
\caption {Clay Paky K-Eye K20 ohne Rot (DMX WB): Farbtafel} 
\end{figure}

\begin{figure}[htp]     % h=here, t=top, b=bottom, p=page
\centering
\includegraphics[width=0.9\textwidth]{FarbtafelWFM/farbtafelWFMkeyebestDMXWB} 
% Bilddatei aus dem Unterverzeichnis bilder holen, skalieren auf 0.8*Satzspiegel
\caption {Clay Paky K-Eye K20 ohne Rot (DMX WB): WFM Farbtafel} 
\end{figure}

\begin{figure}[htp]     % h=here, t=top, b=bottom, p=page
\centering
\includegraphics[width=0.9\textwidth]{Farbtafel/farbtafelkeyebestVideoWB} 
% Bilddatei aus dem Unterverzeichnis bilder holen, skalieren auf 0.8*Satzspiegel
\caption {Clay Paky K-Eye K20 ohne Rot (Video WB): Farbtafel} 
\end{figure}

\begin{figure}[htp]     % h=here, t=top, b=bottom, p=page
\centering
\includegraphics[width=0.9\textwidth]{FarbtafelWFM/farbtafelWFMkeyebestVideoWB} 
% Bilddatei aus dem Unterverzeichnis bilder holen, skalieren auf 0.8*Satzspiegel
\caption {Clay Paky K-Eye K20 ohne Rot (Video WB): WFM Farbtafel} 
\end{figure}




\begin{figure}[htp]     % h=here, t=top, b=bottom, p=page
\centering
\includegraphics[width=0.9\textwidth]{Farbtafel/farbtafelkeye027basisWB} 
% Bilddatei aus dem Unterverzeichnis bilder holen, skalieren auf 0.8*Satzspiegel
\caption {Clay Paky K-Eye K20 mit 027 Rot (Basis WB): Farbtafel} 
\end{figure}

\begin{figure}[htp]     % h=here, t=top, b=bottom, p=page
\centering
\includegraphics[width=0.9\textwidth]{FarbtafelWFM/farbtafelWFMkeye027basisWB} 
% Bilddatei aus dem Unterverzeichnis bilder holen, skalieren auf 0.8*Satzspiegel
\caption {Clay Paky K-Eye K20 mit 027 Rot (Basis WB): WFM Farbtafel} 
\end{figure}

\begin{figure}[htp]     % h=here, t=top, b=bottom, p=page
\centering
\includegraphics[width=0.9\textwidth]{Farbtafel/farbtafelkeye027DMXWB} 
% Bilddatei aus dem Unterverzeichnis bilder holen, skalieren auf 0.8*Satzspiegel
\caption {Clay Paky K-Eye K20 mit 027 Rot (DMX WB): Farbtafel} 
\end{figure}

\begin{figure}[htp]     % h=here, t=top, b=bottom, p=page
\centering
\includegraphics[width=0.9\textwidth]{FarbtafelWFM/farbtafelWFMkeye027DMXWB} 
% Bilddatei aus dem Unterverzeichnis bilder holen, skalieren auf 0.8*Satzspiegel
\caption {Clay Paky K-Eye K20 mit 027 Rot (DMX WB): WFM Farbtafel} 
\end{figure}

\begin{figure}[htp]     % h=here, t=top, b=bottom, p=page
\centering
\includegraphics[width=0.9\textwidth]{Farbtafel/farbtafelkeye027VideoWB} 
% Bilddatei aus dem Unterverzeichnis bilder holen, skalieren auf 0.8*Satzspiegel
\caption {Clay Paky K-Eye K20 mit 027 Rot (Video WB): Farbtafel} 
\end{figure}

\begin{figure}[htp]     % h=here, t=top, b=bottom, p=page
\centering
\includegraphics[width=0.9\textwidth]{FarbtafelWFM/farbtafelWFMkeye027VideoWB} 
% Bilddatei aus dem Unterverzeichnis bilder holen, skalieren auf 0.8*Satzspiegel
\caption {Clay Paky K-Eye K20 mit 027 Rot (Video WB): WFM Farbtafel} 
\end{figure}



\begin{figure}[htp]     % h=here, t=top, b=bottom, p=page
\centering
\includegraphics[width=0.9\textwidth]{Farbtafel/farbtafelkeye787basisWB} 
% Bilddatei aus dem Unterverzeichnis bilder holen, skalieren auf 0.8*Satzspiegel
\caption {Clay Paky K-Eye K20 mit 787 Rot (Basis WB): Farbtafel} 
\end{figure}

\begin{figure}[htp]     % h=here, t=top, b=bottom, p=page
\centering
\includegraphics[width=0.9\textwidth]{FarbtafelWFM/farbtafelWFMkeye787basisWB} 
% Bilddatei aus dem Unterverzeichnis bilder holen, skalieren auf 0.8*Satzspiegel
\caption {Clay Paky K-Eye K20 mit 787 Rot (Basis WB): WFM Farbtafel} 
\end{figure}

\begin{figure}[htp]     % h=here, t=top, b=bottom, p=page
\centering
\includegraphics[width=0.9\textwidth]{Farbtafel/farbtafelkeye787DMXWB} 
% Bilddatei aus dem Unterverzeichnis bilder holen, skalieren auf 0.8*Satzspiegel
\caption {Clay Paky K-Eye K20 mit 787 Rot (DMX WB): Farbtafel} 
\end{figure}

\begin{figure}[htp]     % h=here, t=top, b=bottom, p=page
\centering
\includegraphics[width=0.9\textwidth]{FarbtafelWFM/farbtafelWFMkeye787DMXWB} 
% Bilddatei aus dem Unterverzeichnis bilder holen, skalieren auf 0.8*Satzspiegel
\caption {Clay Paky K-Eye K20 mit 787 Rot (DMX WB): WFM Farbtafel} 
\end{figure}

\begin{figure}[htp]     % h=here, t=top, b=bottom, p=page
\centering
\includegraphics[width=0.9\textwidth]{Farbtafel/farbtafelkeye787VideoWB} 
% Bilddatei aus dem Unterverzeichnis bilder holen, skalieren auf 0.8*Satzspiegel
\caption {Clay Paky K-Eye K20 mit 787 Rot (Video WB): Farbtafel} 
\end{figure}

\begin{figure}[htp]     % h=here, t=top, b=bottom, p=page
\centering
\includegraphics[width=0.9\textwidth]{FarbtafelWFM/farbtafelWFMkeye787VideoWB} 
% Bilddatei aus dem Unterverzeichnis bilder holen, skalieren auf 0.8*Satzspiegel
\caption {Clay Paky K-Eye K20 mit 787 Rot (Video WB): WFM Farbtafel} 
\end{figure}



\section{Videoscreenshots}

\subsection{Arri D5}

\begin{figure}[htp]     % h=here, t=top, b=bottom, p=page
\centering
\includegraphics[width=0.9\textwidth]{Videos/WFMarriD5basisWB} 
% Bilddatei aus dem Unterverzeichnis bilder holen, skalieren auf 0.8*Satzspiegel
\caption {Arri D5 (Basis WB): Video Screenshot} 
\end{figure}

\subsection{Robe DL7F Wash}

\begin{figure}[htp]     % h=here, t=top, b=bottom, p=page
\centering
\includegraphics[width=0.9\textwidth]{Videos/WFMdl7fbestbasisWB} 
% Bilddatei aus dem Unterverzeichnis bilder holen, skalieren auf 0.8*Satzspiegel
\caption {Robe DL7F Wash ohne Rot (Basis WB): Video Screenshot} 
\end{figure}

\begin{figure}[htp]     % h=here, t=top, b=bottom, p=page
\centering
\includegraphics[width=0.9\textwidth]{Videos/WFMdl7fbestDMXWB} 
% Bilddatei aus dem Unterverzeichnis bilder holen, skalieren auf 0.8*Satzspiegel
\caption {Robe DL7F Wash ohne Rot (DMX WB): Video Screenshot} 
\end{figure}

\begin{figure}[htp]     % h=here, t=top, b=bottom, p=page
\centering
\includegraphics[width=0.9\textwidth]{Videos/WFMdl7fbestVideoWB} 
% Bilddatei aus dem Unterverzeichnis bilder holen, skalieren auf 0.8*Satzspiegel
\caption {Robe DL7F Wash ohne Rot (Video WB): Video Screenshot} 
\end{figure}



\begin{figure}[htp]     % h=here, t=top, b=bottom, p=page
\centering
\includegraphics[width=0.9\textwidth]{Videos/WFMdl7f027basisWB} 
% Bilddatei aus dem Unterverzeichnis bilder holen, skalieren auf 0.8*Satzspiegel
\caption {Robe DL7F Wash 027 Rot (Basis WB): Video Screenshot} 
\end{figure}

\begin{figure}[htp]     % h=here, t=top, b=bottom, p=page
\centering
\includegraphics[width=0.9\textwidth]{Videos/WFMdl7f027DMXWB} 
% Bilddatei aus dem Unterverzeichnis bilder holen, skalieren auf 0.8*Satzspiegel
\caption {Robe DL7F Wash 027 Rot (DMX WB): Video Screenshot} 
\end{figure}

\begin{figure}[htp]     % h=here, t=top, b=bottom, p=page
\centering
\includegraphics[width=0.9\textwidth]{Videos/WFMdl7f027VideoWB} 
% Bilddatei aus dem Unterverzeichnis bilder holen, skalieren auf 0.8*Satzspiegel
\caption {Robe DL7F Wash 027 Rot (Video WB): Video Screenshot} 
\end{figure}



\begin{figure}[htp]     % h=here, t=top, b=bottom, p=page
\centering
\includegraphics[width=0.9\textwidth]{Videos/WFMdl7f787basisWB} 
% Bilddatei aus dem Unterverzeichnis bilder holen, skalieren auf 0.8*Satzspiegel
\caption {Robe DL7F Wash 787 Rot (Basis WB): Video Screenshot} 
\end{figure}

\begin{figure}[htp]     % h=here, t=top, b=bottom, p=page
\centering
\includegraphics[width=0.9\textwidth]{Videos/WFMdl7f787DMXWB} 
% Bilddatei aus dem Unterverzeichnis bilder holen, skalieren auf 0.8*Satzspiegel
\caption {Robe DL7F Wash 787 Rot (DMX WB): Video Screenshot} 
\end{figure}

\begin{figure}[htp]     % h=here, t=top, b=bottom, p=page
\centering
\includegraphics[width=0.9\textwidth]{Videos/WFMdl7f787VideoWB} 
% Bilddatei aus dem Unterverzeichnis bilder holen, skalieren auf 0.8*Satzspiegel
\caption {Robe DL7F Wash 787 Rot (Video WB): Video Screenshot} 
\end{figure}


\subsection{Martin MAC Encore Wash CLD}

\begin{figure}[htp]     % h=here, t=top, b=bottom, p=page
\centering
\includegraphics[width=0.9\textwidth]{Videos/WFMencorebestbasisWB} 
% Bilddatei aus dem Unterverzeichnis bilder holen, skalieren auf 0.8*Satzspiegel
\caption {Martin MAC Encore Wash CLD ohne Rot (Basis WB): Video Screenshot} 
\end{figure}

\begin{figure}[htp]     % h=here, t=top, b=bottom, p=page
\centering
\includegraphics[width=0.9\textwidth]{Videos/WFMencorebestDMXWB} 
% Bilddatei aus dem Unterverzeichnis bilder holen, skalieren auf 0.8*Satzspiegel
\caption {Martin MAC Encore Wash CLD ohne Rot (DMX WB): Video Screenshot} 
\end{figure}

\begin{figure}[htp]     % h=here, t=top, b=bottom, p=page
\centering
\includegraphics[width=0.9\textwidth]{Videos/WFMencorebestVideoWB} 
% Bilddatei aus dem Unterverzeichnis bilder holen, skalieren auf 0.8*Satzspiegel
\caption {Martin MAC Encore Wash CLD ohne Rot (Video WB): Video Screenshot} 
\end{figure}



\begin{figure}[htp]     % h=here, t=top, b=bottom, p=page
\centering
\includegraphics[width=0.9\textwidth]{Videos/WFMencore027basisWB} 
% Bilddatei aus dem Unterverzeichnis bilder holen, skalieren auf 0.8*Satzspiegel
\caption {Martin MAC Encore Wash CLD 027 Rot (Basis WB): Video Screenshot} 
\end{figure}

\begin{figure}[htp]     % h=here, t=top, b=bottom, p=page
\centering
\includegraphics[width=0.9\textwidth]{Videos/WFMencore027DMXWB} 
% Bilddatei aus dem Unterverzeichnis bilder holen, skalieren auf 0.8*Satzspiegel
\caption {Martin MAC Encore Wash CLD 027 Rot (DMX WB): Video Screenshot} 
\end{figure}

\begin{figure}[htp]     % h=here, t=top, b=bottom, p=page
\centering
\includegraphics[width=0.9\textwidth]{Videos/WFMencore027VideoWB} 
% Bilddatei aus dem Unterverzeichnis bilder holen, skalieren auf 0.8*Satzspiegel
\caption {Martin MAC Encore Wash CLD 027 Rot (Video WB): Video Screenshot} 
\end{figure}



\begin{figure}[htp]     % h=here, t=top, b=bottom, p=page
\centering
\includegraphics[width=0.9\textwidth]{Videos/WFMencore787basisWB} 
% Bilddatei aus dem Unterverzeichnis bilder holen, skalieren auf 0.8*Satzspiegel
\caption {Martin MAC Encore Wash CLD 787 Rot (Basis WB): Video Screenshot} 
\end{figure}

\begin{figure}[htp]     % h=here, t=top, b=bottom, p=page
\centering
\includegraphics[width=0.9\textwidth]{Videos/WFMencore787DMXWB} 
% Bilddatei aus dem Unterverzeichnis bilder holen, skalieren auf 0.8*Satzspiegel
\caption {Martin MAC Encore Wash CLD 787 Rot (DMX WB): Video Screenshot} 
\end{figure}

\begin{figure}[htp]     % h=here, t=top, b=bottom, p=page
\centering
\includegraphics[width=0.9\textwidth]{Videos/WFMencore787VideoWB} 
% Bilddatei aus dem Unterverzeichnis bilder holen, skalieren auf 0.8*Satzspiegel
\caption {Martin MAC Encore Wash CLD 787 Rot (Video WB): Video Screenshot} 
\end{figure}



\subsection{ETC Source Four LED Series 2 Lustr}

\begin{figure}[htp]     % h=here, t=top, b=bottom, p=page
\centering
\includegraphics[width=0.9\textwidth]{Videos/WFMetcbestbasisWB} 
% Bilddatei aus dem Unterverzeichnis bilder holen, skalieren auf 0.8*Satzspiegel
\caption {ETC Source Four LED Series 2 Lustr ohne Rot (Basis WB): Video Screenshot} 
\end{figure}

\begin{figure}[htp]     % h=here, t=top, b=bottom, p=page
\centering
\includegraphics[width=0.9\textwidth]{Videos/WFMetcbestDMXWB} 
% Bilddatei aus dem Unterverzeichnis bilder holen, skalieren auf 0.8*Satzspiegel
\caption {ETC Source Four LED Series 2 Lustr ohne Rot (DMX WB): Video Screenshot} 
\end{figure}

\begin{figure}[htp]     % h=here, t=top, b=bottom, p=page
\centering
\includegraphics[width=0.9\textwidth]{Videos/WFMetcbestVideoWB} 
% Bilddatei aus dem Unterverzeichnis bilder holen, skalieren auf 0.8*Satzspiegel
\caption {ETC Source Four LED Series 2 Lustr ohne Rot (Video WB): Video Screenshot} 
\end{figure}



\begin{figure}[htp]     % h=here, t=top, b=bottom, p=page
\centering
\includegraphics[width=0.9\textwidth]{Videos/WFMetc027basisWB} 
% Bilddatei aus dem Unterverzeichnis bilder holen, skalieren auf 0.8*Satzspiegel
\caption {ETC Source Four LED Series 2 Lustr 027 Rot (Basis WB): Video Screenshot} 
\end{figure}

\begin{figure}[htp]     % h=here, t=top, b=bottom, p=page
\centering
\includegraphics[width=0.9\textwidth]{Videos/WFMetc027DMXWB} 
% Bilddatei aus dem Unterverzeichnis bilder holen, skalieren auf 0.8*Satzspiegel
\caption {ETC Source Four LED Series 2 Lustr 027 Rot (DMX WB): Video Screenshot} 
\end{figure}

\begin{figure}[htp]     % h=here, t=top, b=bottom, p=page
\centering
\includegraphics[width=0.9\textwidth]{Videos/WFMetc027VideoWB} 
% Bilddatei aus dem Unterverzeichnis bilder holen, skalieren auf 0.8*Satzspiegel
\caption {ETC Source Four LED Series 2 Lustr 027 Rot (Video WB): Video Screenshot} 
\end{figure}



\begin{figure}[htp]     % h=here, t=top, b=bottom, p=page
\centering
\includegraphics[width=0.9\textwidth]{Videos/WFMetc787basisWB} 
% Bilddatei aus dem Unterverzeichnis bilder holen, skalieren auf 0.8*Satzspiegel
\caption {ETC Source Four LED Series 2 Lustr 787 Rot (Basis WB): Video Screenshot} 
\end{figure}

\begin{figure}[htp]     % h=here, t=top, b=bottom, p=page
\centering
\includegraphics[width=0.9\textwidth]{Videos/WFMetc787DMXWB} 
% Bilddatei aus dem Unterverzeichnis bilder holen, skalieren auf 0.8*Satzspiegel
\caption {ETC Source Four LED Series 2 Lustr 787 Rot (DMX WB): Video Screenshot} 
\end{figure}

\begin{figure}[htp]     % h=here, t=top, b=bottom, p=page
\centering
\includegraphics[width=0.9\textwidth]{Videos/WFMetc787VideoWB} 
% Bilddatei aus dem Unterverzeichnis bilder holen, skalieren auf 0.8*Satzspiegel
\caption {ETC Source Four LED Series 2 Lustr 787 Rot (Video WB): Video Screenshot} 
\end{figure}



\subsection{Ayrton Ghibli}

\begin{figure}[htp]     % h=here, t=top, b=bottom, p=page
\centering
\includegraphics[width=0.9\textwidth]{Videos/WFMghiblibestbasisWB} 
% Bilddatei aus dem Unterverzeichnis bilder holen, skalieren auf 0.8*Satzspiegel
\caption {Ayrton Ghibli ohne Rot (Basis WB): Video Screenshot} 
\end{figure}

\begin{figure}[htp]     % h=here, t=top, b=bottom, p=page
\centering
\includegraphics[width=0.9\textwidth]{Videos/WFMghiblibestDMXWB} 
% Bilddatei aus dem Unterverzeichnis bilder holen, skalieren auf 0.8*Satzspiegel
\caption {Ayrton Ghibli ohne Rot (DMX WB): Video Screenshot} 
\end{figure}

\begin{figure}[htp]     % h=here, t=top, b=bottom, p=page
\centering
\includegraphics[width=0.9\textwidth]{Videos/WFMghiblibestVideoWB} 
% Bilddatei aus dem Unterverzeichnis bilder holen, skalieren auf 0.8*Satzspiegel
\caption {Ayrton Ghibli ohne Rot (Video WB): Video Screenshot} 
\end{figure}



\begin{figure}[htp]     % h=here, t=top, b=bottom, p=page
\centering
\includegraphics[width=0.9\textwidth]{Videos/WFMghibli027basisWB} 
% Bilddatei aus dem Unterverzeichnis bilder holen, skalieren auf 0.8*Satzspiegel
\caption {Ayrton Ghibli 027 Rot (Basis WB): Video Screenshot} 
\end{figure}

\begin{figure}[htp]     % h=here, t=top, b=bottom, p=page
\centering
\includegraphics[width=0.9\textwidth]{Videos/WFMghibli027DMXWB} 
% Bilddatei aus dem Unterverzeichnis bilder holen, skalieren auf 0.8*Satzspiegel
\caption {Ayrton Ghibli 027 Rot (DMX WB): Video Screenshot} 
\end{figure}

\begin{figure}[htp]     % h=here, t=top, b=bottom, p=page
\centering
\includegraphics[width=0.9\textwidth]{Videos/WFMghibli027VideoWB} 
% Bilddatei aus dem Unterverzeichnis bilder holen, skalieren auf 0.8*Satzspiegel
\caption {Ayrton Ghibli 027 Rot (Video WB): Video Screenshot} 
\end{figure}



\begin{figure}[htp]     % h=here, t=top, b=bottom, p=page
\centering
\includegraphics[width=0.9\textwidth]{Videos/WFMghibli787basisWB} 
% Bilddatei aus dem Unterverzeichnis bilder holen, skalieren auf 0.8*Satzspiegel
\caption {Ayrton Ghibli 787 Rot (Basis WB): Video Screenshot} 
\end{figure}

\begin{figure}[htp]     % h=here, t=top, b=bottom, p=page
\centering
\includegraphics[width=0.9\textwidth]{Videos/WFMghibli787DMXWB} 
% Bilddatei aus dem Unterverzeichnis bilder holen, skalieren auf 0.8*Satzspiegel
\caption {Ayrton Ghibli 787 Rot (DMX WB): Video Screenshot} 
\end{figure}

\begin{figure}[htp]     % h=here, t=top, b=bottom, p=page
\centering
\includegraphics[width=0.9\textwidth]{Videos/WFMghibli787VideoWB} 
% Bilddatei aus dem Unterverzeichnis bilder holen, skalieren auf 0.8*Satzspiegel
\caption {Ayrton Ghibli 787 Rot (Video WB): Video Screenshot} 
\end{figure}



\subsection{GLP Impression X4 L}

\begin{figure}[htp]     % h=here, t=top, b=bottom, p=page
\centering
\includegraphics[width=0.9\textwidth]{Videos/WFMglpbestbasisWB} 
% Bilddatei aus dem Unterverzeichnis bilder holen, skalieren auf 0.8*Satzspiegel
\caption {GLP Impression X4 L ohne Rot (Basis WB): Video Screenshot} 
\end{figure}

\begin{figure}[htp]     % h=here, t=top, b=bottom, p=page
\centering
\includegraphics[width=0.9\textwidth]{Videos/WFMglpbestDMXWB} 
% Bilddatei aus dem Unterverzeichnis bilder holen, skalieren auf 0.8*Satzspiegel
\caption {GLP Impression X4 L ohne Rot (DMX WB): Video Screenshot} 
\end{figure}

\begin{figure}[htp]     % h=here, t=top, b=bottom, p=page
\centering
\includegraphics[width=0.9\textwidth]{Videos/WFMglpbestVideoWB} 
% Bilddatei aus dem Unterverzeichnis bilder holen, skalieren auf 0.8*Satzspiegel
\caption {GLP Impression X4 L ohne Rot (Video WB): Video Screenshot} 
\end{figure}



\begin{figure}[htp]     % h=here, t=top, b=bottom, p=page
\centering
\includegraphics[width=0.9\textwidth]{Videos/WFMglp027basisWB} 
% Bilddatei aus dem Unterverzeichnis bilder holen, skalieren auf 0.8*Satzspiegel
\caption {GLP Impression X4 L 027 Rot (Basis WB): Video Screenshot} 
\end{figure}

\begin{figure}[htp]     % h=here, t=top, b=bottom, p=page
\centering
\includegraphics[width=0.9\textwidth]{Videos/WFMglp027DMXWB} 
% Bilddatei aus dem Unterverzeichnis bilder holen, skalieren auf 0.8*Satzspiegel
\caption {GLP Impression X4 L 027 Rot (DMX WB): Video Screenshot} 
\end{figure}

\begin{figure}[htp]     % h=here, t=top, b=bottom, p=page
\centering
\includegraphics[width=0.9\textwidth]{Videos/WFMglp027VideoWB} 
% Bilddatei aus dem Unterverzeichnis bilder holen, skalieren auf 0.8*Satzspiegel
\caption {GLP Impression X4 L 027 Rot (Video WB): Video Screenshot} 
\end{figure}



\begin{figure}[htp]     % h=here, t=top, b=bottom, p=page
\centering
\includegraphics[width=0.9\textwidth]{Videos/WFMglp787basisWB} 
% Bilddatei aus dem Unterverzeichnis bilder holen, skalieren auf 0.8*Satzspiegel
\caption {GLP Impression X4 L 787 Rot (Basis WB): Video Screenshot} 
\end{figure}

\begin{figure}[htp]     % h=here, t=top, b=bottom, p=page
\centering
\includegraphics[width=0.9\textwidth]{Videos/WFMglp787DMXWB} 
% Bilddatei aus dem Unterverzeichnis bilder holen, skalieren auf 0.8*Satzspiegel
\caption {GLP Impression X4 L 787 Rot (DMX WB): Video Screenshot} 
\end{figure}

\begin{figure}[htp]     % h=here, t=top, b=bottom, p=page
\centering
\includegraphics[width=0.9\textwidth]{Videos/WFMglp787VideoWB} 
% Bilddatei aus dem Unterverzeichnis bilder holen, skalieren auf 0.8*Satzspiegel
\caption {GLP Impression X4 L 787 Rot (Video WB): Video Screenshot} 
\end{figure}



\subsection{Clay Paky K-Eye K20}

\begin{figure}[htp]     % h=here, t=top, b=bottom, p=page
\centering
\includegraphics[width=0.9\textwidth]{Videos/WFMkeyebestbasisWB} 
% Bilddatei aus dem Unterverzeichnis bilder holen, skalieren auf 0.8*Satzspiegel
\caption {Clay Paky K-Eye K20 ohne Rot (Basis WB): Video Screenshot} 
\end{figure}

\begin{figure}[htp]     % h=here, t=top, b=bottom, p=page
\centering
\includegraphics[width=0.9\textwidth]{Videos/WFMkeyebestDMXWB} 
% Bilddatei aus dem Unterverzeichnis bilder holen, skalieren auf 0.8*Satzspiegel
\caption {Clay Paky K-Eye K20 ohne Rot (DMX WB): Video Screenshot} 
\end{figure}

\begin{figure}[htp]     % h=here, t=top, b=bottom, p=page
\centering
\includegraphics[width=0.9\textwidth]{Videos/WFMkeyebestVideoWB} 
% Bilddatei aus dem Unterverzeichnis bilder holen, skalieren auf 0.8*Satzspiegel
\caption {Clay Paky K-Eye K20 ohne Rot (Video WB): Video Screenshot} 
\end{figure}



\begin{figure}[htp]     % h=here, t=top, b=bottom, p=page
\centering
\includegraphics[width=0.9\textwidth]{Videos/WFMkeye027basisWB} 
% Bilddatei aus dem Unterverzeichnis bilder holen, skalieren auf 0.8*Satzspiegel
\caption {Clay Paky K-Eye K20 027 Rot (Basis WB): Video Screenshot} 
\end{figure}

\begin{figure}[htp]     % h=here, t=top, b=bottom, p=page
\centering
\includegraphics[width=0.9\textwidth]{Videos/WFMkeye027DMXWB} 
% Bilddatei aus dem Unterverzeichnis bilder holen, skalieren auf 0.8*Satzspiegel
\caption {Clay Paky K-Eye K20 027 Rot (DMX WB): Video Screenshot} 
\end{figure}

\begin{figure}[htp]     % h=here, t=top, b=bottom, p=page
\centering
\includegraphics[width=0.9\textwidth]{Videos/WFMkeye027VideoWB} 
% Bilddatei aus dem Unterverzeichnis bilder holen, skalieren auf 0.8*Satzspiegel
\caption {Clay Paky K-Eye K20 027 Rot (Video WB): Video Screenshot} 
\end{figure}



\begin{figure}[htp]     % h=here, t=top, b=bottom, p=page
\centering
\includegraphics[width=0.9\textwidth]{Videos/WFMkeye787basisWB} 
% Bilddatei aus dem Unterverzeichnis bilder holen, skalieren auf 0.8*Satzspiegel
\caption {Clay Paky K-Eye K20 787 Rot (Basis WB): Video Screenshot} 
\end{figure}

\begin{figure}[htp]     % h=here, t=top, b=bottom, p=page
\centering
\includegraphics[width=0.9\textwidth]{Videos/WFMkeye787DMXWB} 
% Bilddatei aus dem Unterverzeichnis bilder holen, skalieren auf 0.8*Satzspiegel
\caption {Clay Paky K-Eye K20 787 Rot (DMX WB): Video Screenshot} 
\end{figure}

\begin{figure}[htp]     % h=here, t=top, b=bottom, p=page
\centering
\includegraphics[width=0.9\textwidth]{Videos/WFMkeye787VideoWB} 
% Bilddatei aus dem Unterverzeichnis bilder holen, skalieren auf 0.8*Satzspiegel
\caption {Clay Paky K-Eye K20 787 Rot (Video WB): Video Screenshot} 
\end{figure}


\section{Video WFM Grautreppe}

\subsection{Arri D5}

\begin{figure}[htp]     % h=here, t=top, b=bottom, p=page
\centering
\includegraphics[width=0.9\textwidth]{GrautreppeWFM/WFMarriD5basisWB} 
% Bilddatei aus dem Unterverzeichnis bilder holen, skalieren auf 0.8*Satzspiegel
\caption {Arri D5 (Basis WB): Video WFM Grautreppe} 
\end{figure}

\subsection{Robe DL7F Wash}

\begin{figure}[htp]     % h=here, t=top, b=bottom, p=page
\centering
\includegraphics[width=0.9\textwidth]{GrautreppeWFM/WFMdl7fbestbasisWB} 
% Bilddatei aus dem Unterverzeichnis bilder holen, skalieren auf 0.8*Satzspiegel
\caption {Robe DL7F Wash ohne Rot (Basis WB): Video WFM Grautreppe} 
\end{figure}

\begin{figure}[htp]     % h=here, t=top, b=bottom, p=page
\centering
\includegraphics[width=0.9\textwidth]{GrautreppeWFM/WFMdl7fbestDMXWB} 
% Bilddatei aus dem Unterverzeichnis bilder holen, skalieren auf 0.8*Satzspiegel
\caption {Robe DL7F Wash ohne Rot (DMX WB): Video WFM Grautreppe} 
\end{figure}

\begin{figure}[htp]     % h=here, t=top, b=bottom, p=page
\centering
\includegraphics[width=0.9\textwidth]{GrautreppeWFM/WFMdl7fbestVideoWB} 
% Bilddatei aus dem Unterverzeichnis bilder holen, skalieren auf 0.8*Satzspiegel
\caption {Robe DL7F Wash ohne Rot (Video WB): Video WFM Grautreppe} 
\end{figure}



\begin{figure}[htp]     % h=here, t=top, b=bottom, p=page
\centering
\includegraphics[width=0.9\textwidth]{GrautreppeWFM/WFMdl7f027basisWB} 
% Bilddatei aus dem Unterverzeichnis bilder holen, skalieren auf 0.8*Satzspiegel
\caption {Robe DL7F Wash 027 Rot (Basis WB): Video WFM Grautreppe} 
\end{figure}

\begin{figure}[htp]     % h=here, t=top, b=bottom, p=page
\centering
\includegraphics[width=0.9\textwidth]{GrautreppeWFM/WFMdl7f027DMXWB} 
% Bilddatei aus dem Unterverzeichnis bilder holen, skalieren auf 0.8*Satzspiegel
\caption {Robe DL7F Wash 027 Rot (DMX WB): Video WFM Grautreppe} 
\end{figure}

\begin{figure}[htp]     % h=here, t=top, b=bottom, p=page
\centering
\includegraphics[width=0.9\textwidth]{GrautreppeWFM/WFMdl7f027VideoWB} 
% Bilddatei aus dem Unterverzeichnis bilder holen, skalieren auf 0.8*Satzspiegel
\caption {Robe DL7F Wash 027 Rot (Video WB): Video WFM Grautreppe} 
\end{figure}



\begin{figure}[htp]     % h=here, t=top, b=bottom, p=page
\centering
\includegraphics[width=0.9\textwidth]{GrautreppeWFM/WFMdl7f787basisWB} 
% Bilddatei aus dem Unterverzeichnis bilder holen, skalieren auf 0.8*Satzspiegel
\caption {Robe DL7F Wash 787 Rot (Basis WB): Video WFM Grautreppe} 
\end{figure}

\begin{figure}[htp]     % h=here, t=top, b=bottom, p=page
\centering
\includegraphics[width=0.9\textwidth]{GrautreppeWFM/WFMdl7f787DMXWB} 
% Bilddatei aus dem Unterverzeichnis bilder holen, skalieren auf 0.8*Satzspiegel
\caption {Robe DL7F Wash 787 Rot (DMX WB): Video WFM Grautreppe} 
\end{figure}

\begin{figure}[htp]     % h=here, t=top, b=bottom, p=page
\centering
\includegraphics[width=0.9\textwidth]{GrautreppeWFM/WFMdl7f787VideoWB} 
% Bilddatei aus dem Unterverzeichnis bilder holen, skalieren auf 0.8*Satzspiegel
\caption {Robe DL7F Wash 787 Rot (Video WB): Video WFM Grautreppe} 
\end{figure}


\subsection{Martin MAC Encore Wash CLD}

\begin{figure}[htp]     % h=here, t=top, b=bottom, p=page
\centering
\includegraphics[width=0.9\textwidth]{GrautreppeWFM/WFMencorebestbasisWB} 
% Bilddatei aus dem Unterverzeichnis bilder holen, skalieren auf 0.8*Satzspiegel
\caption {Martin MAC Encore Wash CLD ohne Rot (Basis WB): Video WFM Grautreppe} 
\end{figure}

\begin{figure}[htp]     % h=here, t=top, b=bottom, p=page
\centering
\includegraphics[width=0.9\textwidth]{GrautreppeWFM/WFMencorebestDMXWB} 
% Bilddatei aus dem Unterverzeichnis bilder holen, skalieren auf 0.8*Satzspiegel
\caption {Martin MAC Encore Wash CLD ohne Rot (DMX WB): Video WFM Grautreppe} 
\end{figure}

\begin{figure}[htp]     % h=here, t=top, b=bottom, p=page
\centering
\includegraphics[width=0.9\textwidth]{GrautreppeWFM/WFMencorebestVideoWB} 
% Bilddatei aus dem Unterverzeichnis bilder holen, skalieren auf 0.8*Satzspiegel
\caption {Martin MAC Encore Wash CLD ohne Rot (Video WB): Video WFM Grautreppe} 
\end{figure}



\begin{figure}[htp]     % h=here, t=top, b=bottom, p=page
\centering
\includegraphics[width=0.9\textwidth]{GrautreppeWFM/WFMencore027basisWB} 
% Bilddatei aus dem Unterverzeichnis bilder holen, skalieren auf 0.8*Satzspiegel
\caption {Martin MAC Encore Wash CLD 027 Rot (Basis WB): Video WFM Grautreppe} 
\end{figure}

\begin{figure}[htp]     % h=here, t=top, b=bottom, p=page
\centering
\includegraphics[width=0.9\textwidth]{GrautreppeWFM/WFMencore027DMXWB} 
% Bilddatei aus dem Unterverzeichnis bilder holen, skalieren auf 0.8*Satzspiegel
\caption {Martin MAC Encore Wash CLD 027 Rot (DMX WB): Video WFM Grautreppe} 
\end{figure}

\begin{figure}[htp]     % h=here, t=top, b=bottom, p=page
\centering
\includegraphics[width=0.9\textwidth]{GrautreppeWFM/WFMencore027VideoWB} 
% Bilddatei aus dem Unterverzeichnis bilder holen, skalieren auf 0.8*Satzspiegel
\caption {Martin MAC Encore Wash CLD 027 Rot (Video WB): Video WFM Grautreppe} 
\end{figure}



\begin{figure}[htp]     % h=here, t=top, b=bottom, p=page
\centering
\includegraphics[width=0.9\textwidth]{GrautreppeWFM/WFMencore787basisWB} 
% Bilddatei aus dem Unterverzeichnis bilder holen, skalieren auf 0.8*Satzspiegel
\caption {Martin MAC Encore Wash CLD 787 Rot (Basis WB): Video WFM Grautreppe} 
\end{figure}

\begin{figure}[htp]     % h=here, t=top, b=bottom, p=page
\centering
\includegraphics[width=0.9\textwidth]{GrautreppeWFM/WFMencore787DMXWB} 
% Bilddatei aus dem Unterverzeichnis bilder holen, skalieren auf 0.8*Satzspiegel
\caption {Martin MAC Encore Wash CLD 787 Rot (DMX WB): Video WFM Grautreppe} 
\end{figure}

\begin{figure}[htp]     % h=here, t=top, b=bottom, p=page
\centering
\includegraphics[width=0.9\textwidth]{GrautreppeWFM/WFMencore787VideoWB} 
% Bilddatei aus dem Unterverzeichnis bilder holen, skalieren auf 0.8*Satzspiegel
\caption {Martin MAC Encore Wash CLD 787 Rot (Video WB): Video WFM Grautreppe} 
\end{figure}



\subsection{ETC Source Four LED Series 2 Lustr}

\begin{figure}[htp]     % h=here, t=top, b=bottom, p=page
\centering
\includegraphics[width=0.9\textwidth]{GrautreppeWFM/WFMetcbestbasisWB} 
% Bilddatei aus dem Unterverzeichnis bilder holen, skalieren auf 0.8*Satzspiegel
\caption {ETC Source Four LED Series 2 Lustr ohne Rot (Basis WB): Video WFM Grautreppe} 
\end{figure}

\begin{figure}[htp]     % h=here, t=top, b=bottom, p=page
\centering
\includegraphics[width=0.9\textwidth]{GrautreppeWFM/WFMetcbestDMXWB} 
% Bilddatei aus dem Unterverzeichnis bilder holen, skalieren auf 0.8*Satzspiegel
\caption {ETC Source Four LED Series 2 Lustr ohne Rot (DMX WB): Video WFM Grautreppe} 
\end{figure}

\begin{figure}[htp]     % h=here, t=top, b=bottom, p=page
\centering
\includegraphics[width=0.9\textwidth]{GrautreppeWFM/WFMetcbestVideoWB} 
% Bilddatei aus dem Unterverzeichnis bilder holen, skalieren auf 0.8*Satzspiegel
\caption {ETC Source Four LED Series 2 Lustr ohne Rot (Video WB): Video WFM Grautreppe} 
\end{figure}



\begin{figure}[htp]     % h=here, t=top, b=bottom, p=page
\centering
\includegraphics[width=0.9\textwidth]{GrautreppeWFM/WFMetc027basisWB} 
% Bilddatei aus dem Unterverzeichnis bilder holen, skalieren auf 0.8*Satzspiegel
\caption {ETC Source Four LED Series 2 Lustr 027 Rot (Basis WB): Video WFM Grautreppe} 
\end{figure}

\begin{figure}[htp]     % h=here, t=top, b=bottom, p=page
\centering
\includegraphics[width=0.9\textwidth]{GrautreppeWFM/WFMetc027DMXWB} 
% Bilddatei aus dem Unterverzeichnis bilder holen, skalieren auf 0.8*Satzspiegel
\caption {ETC Source Four LED Series 2 Lustr 027 Rot (DMX WB): Video WFM Grautreppe} 
\end{figure}

\begin{figure}[htp]     % h=here, t=top, b=bottom, p=page
\centering
\includegraphics[width=0.9\textwidth]{GrautreppeWFM/WFMetc027VideoWB} 
% Bilddatei aus dem Unterverzeichnis bilder holen, skalieren auf 0.8*Satzspiegel
\caption {ETC Source Four LED Series 2 Lustr 027 Rot (Video WB): Video WFM Grautreppe} 
\end{figure}



\begin{figure}[htp]     % h=here, t=top, b=bottom, p=page
\centering
\includegraphics[width=0.9\textwidth]{GrautreppeWFM/WFMetc787basisWB} 
% Bilddatei aus dem Unterverzeichnis bilder holen, skalieren auf 0.8*Satzspiegel
\caption {ETC Source Four LED Series 2 Lustr 787 Rot (Basis WB): Video WFM Grautreppe} 
\end{figure}

\begin{figure}[htp]     % h=here, t=top, b=bottom, p=page
\centering
\includegraphics[width=0.9\textwidth]{GrautreppeWFM/WFMetc787DMXWB} 
% Bilddatei aus dem Unterverzeichnis bilder holen, skalieren auf 0.8*Satzspiegel
\caption {ETC Source Four LED Series 2 Lustr 787 Rot (DMX WB): Video WFM Grautreppe} 
\end{figure}

\begin{figure}[htp]     % h=here, t=top, b=bottom, p=page
\centering
\includegraphics[width=0.9\textwidth]{GrautreppeWFM/WFMetc787VideoWB} 
% Bilddatei aus dem Unterverzeichnis bilder holen, skalieren auf 0.8*Satzspiegel
\caption {ETC Source Four LED Series 2 Lustr 787 Rot (Video WB): Video WFM Grautreppe} 
\end{figure}



\subsection{Ayrton Ghibli}

\begin{figure}[htp]     % h=here, t=top, b=bottom, p=page
\centering
\includegraphics[width=0.9\textwidth]{GrautreppeWFM/WFMghiblibestbasisWB} 
% Bilddatei aus dem Unterverzeichnis bilder holen, skalieren auf 0.8*Satzspiegel
\caption {Ayrton Ghibli ohne Rot (Basis WB): Video WFM Grautreppe} 
\end{figure}

\begin{figure}[htp]     % h=here, t=top, b=bottom, p=page
\centering
\includegraphics[width=0.9\textwidth]{GrautreppeWFM/WFMghiblibestDMXWB} 
% Bilddatei aus dem Unterverzeichnis bilder holen, skalieren auf 0.8*Satzspiegel
\caption {Ayrton Ghibli ohne Rot (DMX WB): Video WFM Grautreppe} 
\end{figure}

\begin{figure}[htp]     % h=here, t=top, b=bottom, p=page
\centering
\includegraphics[width=0.9\textwidth]{GrautreppeWFM/WFMghiblibestVideoWB} 
% Bilddatei aus dem Unterverzeichnis bilder holen, skalieren auf 0.8*Satzspiegel
\caption {Ayrton Ghibli ohne Rot (Video WB): Video WFM Grautreppe} 
\end{figure}



\begin{figure}[htp]     % h=here, t=top, b=bottom, p=page
\centering
\includegraphics[width=0.9\textwidth]{GrautreppeWFM/WFMghibli027basisWB} 
% Bilddatei aus dem Unterverzeichnis bilder holen, skalieren auf 0.8*Satzspiegel
\caption {Ayrton Ghibli 027 Rot (Basis WB): Video WFM Grautreppe} 
\end{figure}

\begin{figure}[htp]     % h=here, t=top, b=bottom, p=page
\centering
\includegraphics[width=0.9\textwidth]{GrautreppeWFM/WFMghibli027DMXWB} 
% Bilddatei aus dem Unterverzeichnis bilder holen, skalieren auf 0.8*Satzspiegel
\caption {Ayrton Ghibli 027 Rot (DMX WB): Video WFM Grautreppe} 
\end{figure}

\begin{figure}[htp]     % h=here, t=top, b=bottom, p=page
\centering
\includegraphics[width=0.9\textwidth]{GrautreppeWFM/WFMghibli027VideoWB} 
% Bilddatei aus dem Unterverzeichnis bilder holen, skalieren auf 0.8*Satzspiegel
\caption {Ayrton Ghibli 027 Rot (Video WB): Video WFM Grautreppe} 
\end{figure}



\begin{figure}[htp]     % h=here, t=top, b=bottom, p=page
\centering
\includegraphics[width=0.9\textwidth]{GrautreppeWFM/WFMghibli787basisWB} 
% Bilddatei aus dem Unterverzeichnis bilder holen, skalieren auf 0.8*Satzspiegel
\caption {Ayrton Ghibli 787 Rot (Basis WB): Video WFM Grautreppe} 
\end{figure}

\begin{figure}[htp]     % h=here, t=top, b=bottom, p=page
\centering
\includegraphics[width=0.9\textwidth]{GrautreppeWFM/WFMghibli787DMXWB} 
% Bilddatei aus dem Unterverzeichnis bilder holen, skalieren auf 0.8*Satzspiegel
\caption {Ayrton Ghibli 787 Rot (DMX WB): Video WFM Grautreppe} 
\end{figure}

\begin{figure}[htp]     % h=here, t=top, b=bottom, p=page
\centering
\includegraphics[width=0.9\textwidth]{GrautreppeWFM/WFMghibli787VideoWB} 
% Bilddatei aus dem Unterverzeichnis bilder holen, skalieren auf 0.8*Satzspiegel
\caption {Ayrton Ghibli 787 Rot (Video WB): Video WFM Grautreppe} 
\end{figure}



\subsection{GLP Impression X4 L}

\begin{figure}[htp]     % h=here, t=top, b=bottom, p=page
\centering
\includegraphics[width=0.9\textwidth]{GrautreppeWFM/WFMglpbestbasisWB} 
% Bilddatei aus dem Unterverzeichnis bilder holen, skalieren auf 0.8*Satzspiegel
\caption {GLP Impression X4 L ohne Rot (Basis WB): Video WFM Grautreppe} 
\end{figure}

\begin{figure}[htp]     % h=here, t=top, b=bottom, p=page
\centering
\includegraphics[width=0.9\textwidth]{GrautreppeWFM/WFMglpbestDMXWB} 
% Bilddatei aus dem Unterverzeichnis bilder holen, skalieren auf 0.8*Satzspiegel
\caption {GLP Impression X4 L ohne Rot (DMX WB): Video WFM Grautreppe} 
\end{figure}

\begin{figure}[htp]     % h=here, t=top, b=bottom, p=page
\centering
\includegraphics[width=0.9\textwidth]{GrautreppeWFM/WFMglpbestVideoWB} 
% Bilddatei aus dem Unterverzeichnis bilder holen, skalieren auf 0.8*Satzspiegel
\caption {GLP Impression X4 L ohne Rot (Video WB): Video WFM Grautreppe} 
\end{figure}



\begin{figure}[htp]     % h=here, t=top, b=bottom, p=page
\centering
\includegraphics[width=0.9\textwidth]{GrautreppeWFM/WFMglp027basisWB} 
% Bilddatei aus dem Unterverzeichnis bilder holen, skalieren auf 0.8*Satzspiegel
\caption {GLP Impression X4 L 027 Rot (Basis WB): Video WFM Grautreppe} 
\end{figure}

\begin{figure}[htp]     % h=here, t=top, b=bottom, p=page
\centering
\includegraphics[width=0.9\textwidth]{GrautreppeWFM/WFMglp027DMXWB} 
% Bilddatei aus dem Unterverzeichnis bilder holen, skalieren auf 0.8*Satzspiegel
\caption {GLP Impression X4 L 027 Rot (DMX WB): Video WFM Grautreppe} 
\end{figure}

\begin{figure}[htp]     % h=here, t=top, b=bottom, p=page
\centering
\includegraphics[width=0.9\textwidth]{GrautreppeWFM/WFMglp027VideoWB} 
% Bilddatei aus dem Unterverzeichnis bilder holen, skalieren auf 0.8*Satzspiegel
\caption {GLP Impression X4 L 027 Rot (Video WB): Video WFM Grautreppe} 
\end{figure}



\begin{figure}[htp]     % h=here, t=top, b=bottom, p=page
\centering
\includegraphics[width=0.9\textwidth]{GrautreppeWFM/WFMglp787basisWB} 
% Bilddatei aus dem Unterverzeichnis bilder holen, skalieren auf 0.8*Satzspiegel
\caption {GLP Impression X4 L 787 Rot (Basis WB): Video WFM Grautreppe} 
\end{figure}

\begin{figure}[htp]     % h=here, t=top, b=bottom, p=page
\centering
\includegraphics[width=0.9\textwidth]{GrautreppeWFM/WFMglp787DMXWB} 
% Bilddatei aus dem Unterverzeichnis bilder holen, skalieren auf 0.8*Satzspiegel
\caption {GLP Impression X4 L 787 Rot (DMX WB): Video WFM Grautreppe} 
\end{figure}

\begin{figure}[htp]     % h=here, t=top, b=bottom, p=page
\centering
\includegraphics[width=0.9\textwidth]{GrautreppeWFM/WFMglp787VideoWB} 
% Bilddatei aus dem Unterverzeichnis bilder holen, skalieren auf 0.8*Satzspiegel
\caption {GLP Impression X4 L 787 Rot (Video WB): Video WFM Grautreppe} 
\end{figure}



\subsection{Clay Paky K-Eye K20}

\begin{figure}[htp]     % h=here, t=top, b=bottom, p=page
\centering
\includegraphics[width=0.9\textwidth]{GrautreppeWFM/WFMkeyebestbasisWB} 
% Bilddatei aus dem Unterverzeichnis bilder holen, skalieren auf 0.8*Satzspiegel
\caption {Clay Paky K-Eye K20 ohne Rot (Basis WB): Video WFM Grautreppe} 
\end{figure}

\begin{figure}[htp]     % h=here, t=top, b=bottom, p=page
\centering
\includegraphics[width=0.9\textwidth]{GrautreppeWFM/WFMkeyebestDMXWB} 
% Bilddatei aus dem Unterverzeichnis bilder holen, skalieren auf 0.8*Satzspiegel
\caption {Clay Paky K-Eye K20 ohne Rot (DMX WB): Video WFM Grautreppe} 
\end{figure}

\begin{figure}[htp]     % h=here, t=top, b=bottom, p=page
\centering
\includegraphics[width=0.9\textwidth]{GrautreppeWFM/WFMkeyebestVideoWB} 
% Bilddatei aus dem Unterverzeichnis bilder holen, skalieren auf 0.8*Satzspiegel
\caption {Clay Paky K-Eye K20 ohne Rot (Video WB): Video WFM Grautreppe} 
\end{figure}



\begin{figure}[htp]     % h=here, t=top, b=bottom, p=page
\centering
\includegraphics[width=0.9\textwidth]{GrautreppeWFM/WFMkeye027basisWB} 
% Bilddatei aus dem Unterverzeichnis bilder holen, skalieren auf 0.8*Satzspiegel
\caption {Clay Paky K-Eye K20 027 Rot (Basis WB): Video WFM Grautreppe} 
\end{figure}

\begin{figure}[htp]     % h=here, t=top, b=bottom, p=page
\centering
\includegraphics[width=0.9\textwidth]{GrautreppeWFM/WFMkeye027DMXWB} 
% Bilddatei aus dem Unterverzeichnis bilder holen, skalieren auf 0.8*Satzspiegel
\caption {Clay Paky K-Eye K20 027 Rot (DMX WB): Video WFM Grautreppe} 
\end{figure}

\begin{figure}[htp]     % h=here, t=top, b=bottom, p=page
\centering
\includegraphics[width=0.9\textwidth]{GrautreppeWFM/WFMkeye027VideoWB} 
% Bilddatei aus dem Unterverzeichnis bilder holen, skalieren auf 0.8*Satzspiegel
\caption {Clay Paky K-Eye K20 027 Rot (Video WB): Video WFM Grautreppe} 
\end{figure}



\begin{figure}[htp]     % h=here, t=top, b=bottom, p=page
\centering
\includegraphics[width=0.9\textwidth]{GrautreppeWFM/WFMkeye787basisWB} 
% Bilddatei aus dem Unterverzeichnis bilder holen, skalieren auf 0.8*Satzspiegel
\caption {Clay Paky K-Eye K20 787 Rot (Basis WB): Video WFM Grautreppe} 
\end{figure}

\begin{figure}[htp]     % h=here, t=top, b=bottom, p=page
\centering
\includegraphics[width=0.9\textwidth]{GrautreppeWFM/WFMkeye787DMXWB} 
% Bilddatei aus dem Unterverzeichnis bilder holen, skalieren auf 0.8*Satzspiegel
\caption {Clay Paky K-Eye K20 787 Rot (DMX WB): Video WFM Grautreppe} 
\end{figure}

\begin{figure}[htp]     % h=here, t=top, b=bottom, p=page
\centering
\includegraphics[width=0.9\textwidth]{GrautreppeWFM/WFMkeye787VideoWB} 
% Bilddatei aus dem Unterverzeichnis bilder holen, skalieren auf 0.8*Satzspiegel
\caption {Clay Paky K-Eye K20 787 Rot (Video WB): Video WFM Grautreppe} 
\end{figure}



\section{Einverständniserklärungen}

\begin{figure}[htp]     % h=here, t=top, b=bottom, p=page
\centering
\includegraphics[width=0.9\textwidth]{EVs/evmanon} 
% Bilddatei aus dem Unterverzeichnis bilder holen, skalieren auf 0.8*Satzspiegel
\caption {Einverständniserklärung: Proband 1} 
\end{figure}

\begin{figure}[htp]     % h=here, t=top, b=bottom, p=page
\centering
\includegraphics[width=0.9\textwidth]{EVs/evamdi} 
% Bilddatei aus dem Unterverzeichnis bilder holen, skalieren auf 0.8*Satzspiegel
\caption {Einverständniserklärung: Proband 2} 
\end{figure}

\begin{figure}[htp]     % h=here, t=top, b=bottom, p=page
\centering
\includegraphics[width=0.9\textwidth]{EVs/evaaron} 
% Bilddatei aus dem Unterverzeichnis bilder holen, skalieren auf 0.8*Satzspiegel
\caption {Einverständniserklärung: Proband 3} 
\end{figure}

\begin{figure}[htp]     % h=here, t=top, b=bottom, p=page
\centering
\includegraphics[width=0.9\textwidth]{EVs/evfynn} 
% Bilddatei aus dem Unterverzeichnis bilder holen, skalieren auf 0.8*Satzspiegel
\caption {Einverständniserklärung: Proband 4} 
\end{figure}





\end{document}