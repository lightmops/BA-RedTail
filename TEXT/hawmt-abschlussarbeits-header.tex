%%%%%%%%%%%%%%%%%%%%%%%%%%%%%%%%%%%%%%%%%%%%%%%%%
%---- LaTeX-Header fuer Abschlussarbeiten, Prof. Thomas Goerne, Dez. 2012/Aug. 2013 ----
%%%%%%%%%%%%%%%%%%%%%%%%%%%%%%%%%%%%%%%%%%%%%%%%%

\documentclass[pagesize,paper=A4,fontsize=12pt,utf8,numbers=noenddot,bibliography=totoc,listof=totoc,DIV=11,BCOR=1mm]{scrreprt}
% BCOR ist die Bindekorrektur (verlorener Rand am linken Blattrand)! Wert haengt von der Art der Heftung ab!!
% DIV ist eine Satzspiegeleinstellung von KOMA-Script / sccreprt.

\pagestyle{headings}

\usepackage[T1]{fontenc}
\usepackage[utf8]{inputenc} % Font Encoding fuer europaeische Schriften mit Umlauten (Unterstuetzung der Worttrennung)
\usepackage{lmodern} % PostScript-Varianten der TeX Computer Modern-Schriften laden
\usepackage[english,ngerman]{babel} % Spracheinstellungen fuer Englisch und Neudeutsch laden

\usepackage{graphicx} % Grafikeinbindung (fuer .JPG, .JPEG, .PNG und .PDF, falls pdflatex benutzt wird)
\usepackage[table]{xcolor} % ermoeglicht farbige Schrift und farbige Tabellenzeilen
\definecolor{black}{gray}{0} % Umdefinition der Farbe black, falls noetig (0=schwarz, 1=weiss)
\definecolor{dblue}{rgb}{0.1,0.2,0.6} % Dunkelblau, fuer Hyperlinks
\definecolor{lgray}{gray}{0.9} % Hellgrau, fuer Tabellen (0=schwarz, 1=weiss)

\usepackage{booktabs} % fuer schoene Tabellen

\usepackage[round,authoryear]{natbib} % Literaturverweise mit Name/Jahreszahl in runden Klammern
\bibpunct[:\,]{(}{)}{,}{a}{}{,~}  % Feinformatierung der Natbib-Zitierweise

\usepackage[hyphens]{url}
\usepackage[colorlinks=true,linkcolor=black,citecolor=dblue,urlcolor=dblue]{hyperref} 
\usepackage{hyperref}  
% die Pakete url und hyperref ermoeglichen anklickbare URLs im Quellenverzeichnis in definierter Farbe, 
% sie ermoeglichen den Zeilenumbruch bei langen URLs, und sie erzeugen Hyperlinks (Farbe s.o.) 
% zwischen Quellenverweis und Quellenverzeichnis sowie zwischen label und ref im PDF-Dokument

% Fonteinstellungen fuer Bildunterschriften: Unterschrift serifenlos, "Abbildung" fett (bfseries = bold face series)
\setkomafont{captionlabel}{\sffamily\bfseries}
\setkomafont{caption}{\sffamily}

%------------------------------------------------------------------------------------------------------------------
%------ Eigenstaendigkeitserklaerung im gerahmten Kasten (parbox in einer framebox) ------
%------------------------------------------------------------------------------------------------------------------

\newcommand{\eigen}{
\setlength{\fboxsep}{2ex}
\setlength{\fboxrule}{0.8pt} 
% Einstellungen fuer Rahmenabstand und Rahmendicke der Framebox
\begin{center}
	\fbox{
		\parbox{0.8\linewidth}{
		Ich versichere, die vorliegende Arbeit selbstst\"andig ohne fremde Hilfe verfasst 
		und keine anderen Quellen und Hilfsmittel als die angegebenen benutzt zu haben. 
		Die aus anderen Werken w\"ortlich entnommenen Stellen oder dem Sinn nach 
		entlehnten Passagen sind durch Quellenangaben eindeutig kenntlich gemacht.
		\par\bigskip\bigskip\bigskip\bigskip
		\hspace*{0.8cm}Ort, Datum \hfill \vorname~\nachname\hspace*{0.8cm}
		}
	}
\end{center}
}

%%%%%%%%%%%%%%%%%%%%%%%%%%%%%%%%%%%%%%%%%%%%%%%%%