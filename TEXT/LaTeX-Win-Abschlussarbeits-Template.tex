%%%%%%%%%%%%%%%%%%%%%%%%%%%%%%%%%%%%%%%%%%%%%%%%%
%------ LaTeX-Template für Abschlussarbeiten, Prof. Thomas Görne, Dezember 2012 --------
%%%%%%%%%%%%%%%%%%%%%%%%%%%%%%%%%%%%%%%%%%%%%%%%%

%---- Header (mit Formateinstellugen) laden, Inputencoding prüfen ------

\input{hawmt-abschlussarbeits-header}


%------------------------ Titelblatt-Layout laden ----------------------------------

\input{hawmt-bachelor-titelblatt}
%\input{hawmt-master-titelblatt}

%---------------------------- Titeldefinitionen --------------------------------------

\newcommand{\vorname}{Matthias}
\newcommand{\nachname}{Held}
\newcommand{\matrikelnummer}{2182712}

\newcommand{\titel}{"Red Tail":\\ Auswirkung eines zusätzlichen tiefroten Spektralanteils auf das Weißlicht von LED-Scheinwerfern\\[0.2ex] 
				\Large - am Beispiel der Beleuchtung von Hauttönen im TV-Bereich}

\newcommand{\erstpruef}{Prof. Dr. Roland Greule}
\newcommand{\zweitpruef}{Dipl.-Ing. Matthias Allhoff}

\date{vorläufige Fassung vom \today}   % praktisch für Vorab-Versionen. 
%\date{\sffamily Hamburg, 2. 2. 2020}  % Abgabedatum!

%--------------------------------------------------------------------------------------
%----------------------------- hier gehts los! --------------------------------------
%--------------------------------------------------------------------------------------

\begin{document}
\selectlanguage{ngerman}
\maketitle           % Titelseite erzeugen
\tableofcontents % Inhaltsverzeichnis erzeugen
\clearpage          % Seitenumbruch


%------------ Zusammenfassung / Abstract ------------------

\thispagestyle{empty}
\selectlanguage{english}
\section*{\centering\abstractname}
Form and layout of this \LaTeX-template incorporate the guidelines for theses in the Media Technology Department \glqq Richtlinien zur Erstellung schriftlicher Arbeiten, vorrangig Bachelor-Thesis (BA) und Master-Thesis (MA) im Department Medientechnik in der Fa\-kul\-t{\"a}t DMI an der HAW Hamburg\grqq\ in the version of December 6, 2012 by Prof.\ Wolfgang Willaschek. 

The thesis should be printed single-sided (simplex). The binding correction (loss at the left aper edge due to binding) might be adjusted, according to the type of binding. This template incorporates a binding correction as BCOR=1mm (suitable for adhesive binding) in the \LaTeX\ document header.

{\bfseries This is the english version of the opening abstract} (don't forget to set \LaTeX's language setting back to ngerman after the english text). 
 
 
\selectlanguage{ngerman}
\section*{\centering\abstractname}

Diese Arbeit befasst sich mit der Theorie der mo2 GmbH, den LED-Scheinwerfern fehle ein tiefroter Spektralanteil in ihrem kaltweißen Lichtspektrum, um Personen im Fernsehen natürlich aussehen zu lassen. Anhand von Messungen mit im TV-Bereich üblichen LED-Scheinwerfern soll die "Red Tail"-Theorie überprüft werden. Zusätzlich wird mit einer Umfrage eruiert, ob Personen, die mit einem zugemischten "Red Tail" beleuchtet werden, natürlicher aussehen.
In der Einleitung wird daher auf wichtige Kenngrößen der Lichttechnik eingegangen und verschiedene Leuchtmittel beschrieben, um auf die Probleme von LED-Leuchtmittel hinzuweisen. Zuerst wird mit einer Vormessung festgestellt welche Rot-Filter relevant sind und wie stark das Ausmaß eines zusätzlichen "Red Tail" im Spektrum ist, um davon abhängig eine Kamera und Bildschirmwahl zu begründen. Bei der Hauptmessung werden vorallendingen TLCI-Werte miteinander verglichen. Nach der Umfrage werden die Messergebnisse mit den Umfrageergebnissen in Verbindung gebracht 






%--------------------------- Text -------------------------------

\chapter{RechercheTeil}

\section{Unterkapitel mit Mathematik, Bildern und Querverweisen}

\chapter{Messungen}

\section{Unterkapitel mit Mathematik, Bildern und Querverweisen}

\chapter{Messergebnisse}

\section{Unterkapitel mit Mathematik, Bildern und Querverweisen}

\chapter{Umfrage}

\section{Unterkapitel mit Mathematik, Bildern und Querverweisen}

\chapter{Umfrageergebnisse}

\section{Unterkapitel mit Mathematik, Bildern und Querverweisen}

\chapter{Auswertung aller Ergebnisse}

\section{Unterkapitel mit Mathematik, Bildern und Querverweisen}

\chapter{Fazit}

\section{Unterkapitel mit Mathematik, Bildern und Querverweisen}





%--------------------- VERZEICHNISSE ----------------

\listoffigures % Abbildungsverzeichnis erzeugen
\listoftables % Tabellenverzeichnis erzeugen

%--------------------- LITERATURLISTE ---------------
% Die Einträge sollen alphabetisch sortiert sein.

\begin{thebibliography}{}

% Formatierung für Internetquelle
% Grundregel: Name, Vorname (falls vorhanden), VÖ-Jahr (falls vorhanden), Titel in Anführungszeichen, URL, Datum des letzten Aufrufs
% zur Formatierung der URL unbedingt den url-Befehl benutzen!!!
\bibitem[Blu-ray Disc Association(2005)]{bluray} 
Blu-ray Disc Association: 
\emph{White paper Blu-ray Disc Format 2.B Audio Visual Application, Format Specifications for BD-ROM}, 
\url{http://www.blu-raydisc.com/Assets/downloadablefile/2b_bdrom_audiovisualapplication_0305-12955-15269.pdf}, 2005, letzter Zugriff: 1. 10. 2012

% Formatierung für Aufsatz / Paper: Titel in Anführungszeichen, Zeitschriftentitel kursiv
\bibitem[Dooley \& Streicher(1982)]{dooley_streicher} 
Dooley, Wesley L.  \& Streicher, Ronald D.:
\glqq M--S Stereo: A Powerful Technique for Working in Stereo\grqq, 
\emph{Journ. Audio Engineering Society} vol. 30 (10), 1982

% Formatierung für Fachbuch, Diplomarbeit o.Ä.: Titel kursiv
\bibitem[Kuttruff(1991)]{kuttruff}
Kuttruff, Heinrich: 
\emph{Room Acoustics}, 3. Aufl., Elsevier 1991

% Formatierung für Fachbuch mit Herausgeber und mehreren Autoren
\bibitem[Spehr(2009)]{spehr}
Spehr, Georg (Hrsg.): 
\emph{Funktionale Klänge}, transcript 2009

% Formatierung für ein einzelnes Kapitel eines speziellen Autors aus einem Fachbuch mit mehreren Autoren
\bibitem[Sowodniok(2009)]{sowodniok}
Sowodniok, Ulrike: 
\glqq Funktionaler Stimmklang -- Ein Prozess mit Nachhalligkeit\grqq, 
in: Spehr, Georg (Hrsg.): \emph{Funktionale Klänge}, transcript 2009

% Formatierung für Aufsatz / Paper: Titel in Anführungszeichen, Zeitschriftentitel kursiv
\bibitem[Stephenson(1990)]{stephenson}
Stephenson, Uwe: 
\glqq Comparison of the Mirror Image Source Method and the Sound Particle Simulation Method\grqq, 
\emph{Applied Acoustics} vol. 29, 1990


\end{thebibliography}

%--------------------- EIGENSTÄNDIGKEITSERKLÄRUNG ---------------
\clearpage\thispagestyle{empty}
\eigen  % im header definiert
%--------------------------------------- ENDE ------------------------------------
\end{document}
%%%%%%%%%%%%%%%%%%%%%%%%%%%%%%%%%%%%











