%%%%%%%%%%%%%%%%%%%%%%%%%%%%%%%%%%%%%%%%%%%%%%%%%
%------ LaTeX-Template für Abschlussarbeiten, Prof. Thomas Görne, Dezember 2012 --------
%%%%%%%%%%%%%%%%%%%%%%%%%%%%%%%%%%%%%%%%%%%%%%%%%

%---- Header (mit Formateinstellugen) laden, Inputencoding prüfen ------

\input{hawmt-abschlussarbeits-header}


%------------------------ Titelblatt-Layout laden ----------------------------------

\input{hawmt-bachelor-titelblatt}
%\input{hawmt-master-titelblatt}

%---------------------------- Titeldefinitionen --------------------------------------

\newcommand{\vorname}{Matthias}
\newcommand{\nachname}{Held}
\newcommand{\matrikelnummer}{2182712}

\newcommand{\titel}{\glqq Red Tail\grqq\ :\\ Auswirkung eines zusätzlichen tiefroten Spektralanteils auf das Weißlicht von LED-Scheinwerfern\\[0.2ex] 
				\Large - am Beispiel der Beleuchtung von Hauttönen im TV-Bereich}

\newcommand{\erstpruef}{Prof. Dr. Roland Greule}
\newcommand{\zweitpruef}{Dipl. Ing. (FH) Matthias Allhoff}

\date{vorläufige Fassung vom \today}   % praktisch für Vorab-Versionen. 
%\date{\sffamily Hamburg, 2. 2. 2020}  % Abgabedatum!

%--------------------------------------------------------------------------------------
%----------------------------- hier gehts los! --------------------------------------
%--------------------------------------------------------------------------------------

\begin{document}
\selectlanguage{ngerman}
\maketitle           % Titelseite erzeugen
\tableofcontents % Inhaltsverzeichnis erzeugen
\clearpage          % Seitenumbruch


%------------ Zusammenfassung / Abstract ------------------

\thispagestyle{empty}
\selectlanguage{english}
\section*{\centering\abstractname}
Form and layout of this \LaTeX-template incorporate the guidelines for theses in the Media Technology Department \glqq Richtlinien zur Erstellung schriftlicher Arbeiten, vorrangig Bachelor-Thesis (BA) und Master-Thesis (MA) im Department Medientechnik in der Fa\-kul\-t{\"a}t DMI an der HAW Hamburg\grqq\ in the version of December 6, 2012 by Prof.\ Wolfgang Willaschek. 

The thesis should be printed single-sided (simplex). The binding correction (loss at the left aper edge due to binding) might be adjusted, according to the type of binding. This template incorporates a binding correction as BCOR=1mm (suitable for adhesive binding) in the \LaTeX\ document header.

{\bfseries This is the english version of the opening abstract} (don't forget to set \LaTeX's language setting back to ngerman after the english text). 
 
 
\selectlanguage{ngerman}
\section*{\centering\abstractname}

Diese Arbeit befasst sich mit der Auswirkung eines zusätzlichen tiefroten Spektralanteils auf das kaltweiße Lichtspektrum von LED-Scheinwerfern. Es soll dabei überprüft werden, ob Personen unter diesen Umständen im Kamerabild natürlicher aussehen, wie es in der \glqq Red Tail\grqq\ - Theorie der mo2 design GmbH angenommen wird.\\
Zunächst wird auf wichtige Kenngrößen der Lichttechnik eingegangen und verschiedene Leuchtmittel und lichttechnische Parameter werden erläutert. Im Folgeneden werden die Messungen beschrieben.\\
Bei diesen wird ein LED-Scheinwerfer und ein rotgefilterter PAR-Scheinwerfer, der den\glqq Red Tail\grqq\ simulieren soll, auf einen Messpunkt ausgerichtet. Der LED-Scheinwerfer wird zuerst allein auf eine kaltweiße Referenzlichtquelle bestmöglich abgeglichen und spektral vermessen. Anschließend wird der rotgefilterter PAR-Scheinwerfer dazugeschlatet und auch dieses Lichtgemisch wird auf die Referenzlichtquelle abgeglichen und spektral vermessen. 
Bei der Auswertung werden die gemessenen lichttechnischen Parameter betrachtet und zusätzlich werden bei einer Umfrage Bilder verglichen, auf denen Probanden verschiedener Hauttöne mit und ohne \glqq Red Tail\grqq\ beleuchtet wurden.




%--------------------------- Text -------------------------------

\chapter{Einleitung}

\chapter{Grundlagen und Kenngrößen der Lichttechnik} 

\section{Lichtstrom $\Phi$} \label{sec_lumen}

\section{Beleuchtungsstärke E}\label{sec_lux}

\section{Lichtstärke I}\label{sec_candela}

\section{Leuchtdichte L}\label{sec_candelamm}

\chapter{Farbe und Farbräume}

\section{Sehen mit dem Auge} \label{sec_auge}
Um Farben und Farbräume erklären zu können, werden in diesem Kapitel die Grundlagen der Farbwahrnehmung beschrieben.\\
Im Auge gibt es zwei Arten von lichtempfindlichen Rezeptoren in der Netzhaut, die für unsere Farbwahrnehmung verantwortlich sind: Zapfen und Stäbchen.\\
Die Stäbchen nehmen verschiedene Helligkeitseindrücke wahr, können aber keine Farben unterscheiden. Daher sind sie für das skotopische Sehen (von 3 x $10^{-6} \frac{cd}{m^{2}}$ bis 0,03$\frac{cd}{m^{2}}$) verantwortlich \footnote{\cite{doccheck sko}}.
Die verschiedenen spektralen Anteile des Lichts wirken sich auf die Zapfen aus und verantworten so den Farbeindruck. Außerdem sind die Zapfen für das photopische Sehen (ab einer Leuchtdichte von 3$\frac{cd}{m^{2}}$) zuständig \footnote{\cite{doccheck pho}}.
  



\begin{figure}[htp]     % h=here, t=top, b=bottom, p=page
\centering
\includegraphics[width=0.7\textwidth]{bilder/augespek} 
% Bilddatei aus dem Unterverzeichnis bilder holen, skalieren auf 0.8*Satzspiegel
\caption {Zapfen und Stäbchen im Auge\protect\footnotemark}\label{b_augespek}
\end{figure}

\footnotetext{\url{https://www.gigahertz-optik.de/assets/Uploads/Abb.-II.13-neu-v03.png}}


\section{Sichtbares Spektrum} \label{sec_spektrum}

\section{Farbe} \label{sec_farbe}

\section{RGB Farbraum} \label{sec_rgb}

\section{CIE-XYZ Farbraum} \label{sec_xyz}

\section{CIE-LUV Farbraum} \label{sec_luv}

\section{CIE-LAB Farbraum} \label{sec_lab}

\chapter{Lichtechnische Parameter}

\section{Color Rendering Index (CRI)} \label{sec_cri}

Da der Farbort allein keine eindeutige Aussage über die Zusammensetzung des Spektrums zulässt, wurde 1931 von der Commission Internationale de l'Eclairage ein Testverfahren entwickelt, mit dem man die Farbwiedergabe (Color Rendering Index) einer Leuchte bestimmen kann. Dafür hat man acht Referenzfarben festgelegt. Bei einer CRI-Messung überprüft man also, wie gut eine Lichtquelle diese Körperfarben wiedergeben kann. Es wird dabei zwischen einem schwarzen Strahler(< 5000K) und Tageslicht(> 5000K) differenziert. Die gemessenen Unterschiede zu den Referenzfarben werden mit Werten von 0 bis 100 gewichtet($R_{1}$-$R_{8}$), wobei ein Wert von 100 aussagt, dass die Farbe bestmöglich wiedergegeben wird. Zuerst werden die einzelnen Indexwerte $R_{i}$ aus den Farbdifferenzen $\Delta E_{i}$ berechnet (Gleichung \ref{gl_cri1})\footnote{\cite{davis_ohno}}.

	\begin{equation}\label{gl_cri1}
		R_{i} = 100 - 4,6 \cdot \Delta E_{i}
	\end{equation}
Diese acht Werte werden schließlich arithmetisch gemittelt und es ergibt sich der Gesamtwert $R_{a}$ (Gleichung \ref{gl_cri2})\footnote{\cite{production partner}}.
	\begin{equation}\label{gl_cri2}
		R_{a} =\frac{1}{8} \sum_{i=1}^{8} R_{i}
	\end{equation}
In der DIN 6169 werden zur besseren Beurteilung der Farbwiedergabe die $R_{a}$-Werte in verschiedene Stufen unterteilt (Tabelle \ref{t_cri}).

	\begin{table}[htp] 
		\rowcolors{1}{}{lgray} 
		\centering
		\begin{tabular}{rlcc}  % Spalten nach Ausrichtung: l, c, r, p{breite} 
		\toprule
		\multicolumn{3}{c}{\large\sffamily Stufen des CRI}\\ 							
		\midrule
		1A & $R_{a} \geq 90$ & sehr hohe Anforderung\\ 
		1B & 90 > $R_{a} \geq 80$ & sehr hohe Anforderung\\
		2A & 80 > $R_{a} \geq 70$ & hohe Anforderung\\
		2B & 70 > $R_{a} \geq 60$ & hohe Anforderung\\
		3 & 60 > $R_{a} \geq 40$ & mittlere Anforderung\\
		4 & 40 > $R_{a} \geq 20$ & geringe Anforderung\\
		\bottomrule
		\end{tabular}
		\caption{$R_{a}$ eingeteilt in verschiedene Stufen\protect\footnotemark}	
		\label{t_cri}
	\end{table}
	\footnotetext{\cite[111]{hentschel}}

Ein hoher $R_{a}$-Wert beschreibt aber nur bedingt die Farbwiedergabe einer Leuchte, da beispielsweise keine Angabe über die Sättigung der Farben gemacht wird. Außerdem sind die acht Referenzfarben nur Pastelltöne, weil der CRI damals für Glühlicht entwickelt wurde. Gesättigte Farben fließen nicht in die Bewertung mit ein.
Das wirkt sich auch auf die Vergleichbarkeit von Leuchten aus. Zwei Scheinwerfer mit dem selben $R_{a}$-Wert von 90 können sehr unterschiedliche Spektren haben und damit sehr unterschiedlich Farben darstellen, trotz gleichem Farbwiedergabeindex.
Außerdem kann man nur schwer eine Aussage darüber machen, ob sich eine Leuchte mit einem guten CRI für Personenbeleuchtung eignet, weil Rottöne und Hauttöne in diesem Bewertungsverfahren fehlen.\\\\
Leuchtstofflampen nutzten den CRI aus, indem durch gezielte schmalbandige Peaks im Spektrum die Referenzfarben getroffen werden. Auf diese Weise kann zwar ein hoher CRI-Werte erreicht werden, aber kein breitbandiges und ausgefülltes Lichtspektrum entstehen. Daher sah sich die CIE gezwungen den Farbwiedergabeindex zu erweitern. In dem neueren $R_{e}$-Wert gibt es nun auch gesättigte Farben und eine Hautfarbe wird miteinbezogen (Abb. \ref{b_cri}).

\begin{figure}[htp]     % h=here, t=top, b=bottom, p=page
\centering
\includegraphics[width=0.8\textwidth]{bilder/cri} 
% Bilddatei aus dem Unterverzeichnis bilder holen, skalieren auf 0.8*Satzspiegel
\caption {Alle Referenzfarben des Farbwiedergabeindexes: $R_{1}$ Altrosa, $R_{2}$ Senfgelb, $R_{3}$ Gelbgrün, $R_{4}$ Hellgrün, $R_{5}$ Türkisblau, $R_{6}$ Himmelblau, $R_{7}$ Asterviolett, $R_{8}$ Fliederviolett, $R_{9}$ Rot gesättigt, $R_{10}$ Gelb gesättigt, $R_{11}$ Grün gesättigt, $R_{12}$ Blau gesättigt und $R_{13}$ Rosa (Hautfarbe), $R_{14}$ Blattgrün \protect\footnotemark}\label{b_cri}
\end{figure}

\footnotetext{\url{https://www.elementalled.com/wp/wp-content/uploads/2015/08/CRI_chart.jpg}}


Bei einer warmweißen LED konnte ein CRI von 82 gemessen werden (Abbildung \ref{b_cri2}). Der $R_{e}$-Wert ist naturgemäß schlechter als der $R_{a}$-Wert, aber auch dieser ist mit 77 noch akzeptabel, wenn man bedenkt, dass der $R_{9}$-Wert nur 15 Punkte erbringt. Diese Leuchte entspricht \glqq sehr hohen Anforderungen\grqq (Tabelle \ref{t_cri}) und ist damit nach Definition sehr gut in der Farbwiedergabe. Jedoch ist der $R_{9}$-Wert ein Hinweis darauf, dass man mit dieser Aussage vorsichtig sein sollte.

\begin{figure}[htp]     % h=here, t=top, b=bottom, p=page
\centering
\includegraphics[width=1.0\textwidth]{bilder/cri2} 
% Bilddatei aus dem Unterverzeichnis bilder holen, skalieren auf 0.8*Satzspiegel
\caption {Messung einer warmweißen LED-Leuchte (Ausschnitt aus dem Demofile des Programmes \glqq LiVal\grqq von der Firma JETI): Links ist das Lichtspektrum der Leuchte dargestellt, rechts die gemessenen CRI-Werte  \protect\footnotemark}\label{b_cri2}
\end{figure}

 

 Daher ist auch mit einem einzigen Rot- und Hautton der CRI zu wenig ausschlaggebend, um damit eine Leuchte für Personenbeleuchtung zu bewerten (Kap. \ref{sec_auge}). Zusätzlich entsteht bei LED-Leuchtmitteln ein ähnliches Problem, wie bei den Leuchstoffröhren. Man kann das Spektrum mit den Peaks gut auf die Referenzfarben ausrichten, ohne das Gesamte Spektrum abdecken zu müssen. Gerade bei LED-Leuchten kann dieses Verhalten des CRI ausgenutzt werden, um kritische Bereiche zu verschleiern. Zusätzlich wird dies durch die arithmetische Mittlung der Referenzfarbwerte begünstigt. Ein, zwei schlechtere Werte mindern den $R_{a}$-Wert nicht beträchtlich. Beispielsweise wird bei Weißen-LEDs  der fehlende Rotanteil nur am niedrigen $R_{9}$-Wert sichtbar, aber im CRI-Wert sind diese Schwächen einer LED-Leuchte kaum erkennbar \footnote{\cite{davis_ohno}}. Der CRI kann daher eher als richtungsweisend betrachtet werden: Eine Leuchte mit guter Farbwiedergabe wird auch immer einen guten CRI-Wert haben. Zum Vergleich für Leuchten eignen sich andere Farbwiedergabewerte heutzutage besser \footnote{\cite{production partner}}.\\\\
Aus diesen Gründen und der Erkenntnis der CIE, \emph{\glqq dass die CRI-Methode generell nicht anwendbar ist, um eine Anzahl von Lichtquellen gemäß ihrer Farbwiedergabe einzuordnen, wenn weiße LEDs darunter sind\grqq}\footnote{\citep[VI]{CIE}}, wird sich diese Arbeit hauptsächlich auf andere Farbwiedergabewerte konzentrieren, den CRI aber mit aufführen, weil dieser in der Scheinwerfer- und Fernsehbranche (noch) einen hohen Stellenwert inne hat.

\section{Color Quality Scale (CQS)} \label{sec_cqs}

Der Color Quality Scale, der von dem National Institute of Standards and Technology (NIST) erarbeitet wurde, orientiert sich an der Grundidee des CRI und versucht dessen Probleme anzugehen und ihn zu ersetzen. So gibt es fünfzehn voll saturierte Referenzfarben, die auch auf LED-Leuchten anwendbar sind. Über Skaleneffekte soll der CQS auch indirekt eine Aussage über die Farbwiedergabe von Pastelltönen ermöglichen (Abb. \ref{b_cqs1}). 

\begin{figure}[htp]     % h=here, t=top, b=bottom, p=page
\centering
\includegraphics[width=0.8\textwidth]{bilder/cqs} 
% Bilddatei aus dem Unterverzeichnis bilder holen, skalieren auf 0.8*Satzspiegel
\caption {Alle Referenzfarben des CQS mit voller Sättigung\protect\footnotemark}\label{b_cqs1}
\end{figure}

\footnotetext{\url{https://www.lemoledlight.com/wp-content/uploads/2016/04/LED-Lighting-CRI-5.jpg}}
Bei dem Farbvergleich des CRI wurden weniger Punkte für eine Farbe vergeben, wenn diese übersättigt wurde, also die Leuchte eine höhere Farbigkeit hatte als das Referenzlicht des CRI. Wenn beispielsweise eine Oberfläche eines Objekts beleuchtet wird, kann eine übersättigte Farbe jedoch hilfreich sein und ist daher nicht pauschal negativ einzuordnen. Deswegen wertet der CQS eine Übersättigung der Farbe nicht, nur eine Abweichung von Farbton oder Helligkeit wird bestraft. Außerdem errechnet sich der CQS aus dem quadratischen Mittel (root-means-square) der einzelnen Farben und es ist deutlicher erkennbarer, wenn einzelne Farbe schlechte Werte erzielen (Gleichung \ref{gl_cqs1})\footnote{\cite{davis_ohno}}.

\begin{equation}\label{gl_cqs1}
		\Delta E_{rms} = \sqrt{\frac{1}{15} \sum_{i=1}^{15} \Delta E_{i} ^{2}} 
\end{equation}

Aus diesem Farbdifferenzwert wird ähnlich wie beim CRI (Gleichung \ref{gl_cri1}) ein Farbwiedergabewerte errechnet (Gleichung \ref{gl_cqs2}).

\begin{equation}\label{gl_cqs2}
		Q_{f,rms} = 100 - 3,0305 \cdot \Delta E_{rms} 
\end{equation}

Schließlich wird der CQS auf Werte von 0 bis 100 skaliert. Dadurch entfallen beim CQS negative Farbwerte, die beim CRI sehr schwierig zu interpretieren sind (Gleichung \ref{gl_cqs3}). 

\begin{equation}\label{gl_cqs3}
		Q_{f} = 10 \ln(e^{\frac{Q_{f,rms}}{10}}+1) 
\end{equation}\\

Der CQS wird mit seinen fünfzehn Referenzfarborten (abhängig von der Farbtemperatur) im CIELAB-Farbraum eingezeichnet. Da die Abstände von Farborten in diesem Farbraum in etwa wahrgenommenen Farbunterschieden entsprechen (Kap. \ref{sec_lab}), kann man gut erkennen, wie stark sich die Farbwiedergabe einer Leuchte den Referenzwerten ähneln(Abb. \ref{b_cqs2a} und \ref{b_cqs2b}).\\

\begin{figure}[htp]     % h=here, t=top, b=bottom, p=page
\centering
\includegraphics[width=0.9\textwidth]{bilder/cqs2a} 
% Bilddatei aus dem Unterverzeichnis bilder holen, skalieren auf 0.8*Satzspiegel
\caption {Ausschnitt aus dem Programm \glqq LiVal\grqq von der Firma JETI: Demo Spektrum einer warmweißen LED (2942K) mit $Q_{f} = 81$}\label{b_cqs2a}
\end{figure}

\begin{figure}[htp]     % h=here, t=top, b=bottom, p=page
\centering
\includegraphics[width=0.7\textwidth]{bilder/cqs2b} 
% Bilddatei aus dem Unterverzeichnis bilder holen, skalieren auf 0.8*Satzspiegel
\caption {Ausschnitt aus dem Programm \glqq LiVal\grqq von der Firma JETI: Die fünfzehn Referenzfarben(blau) im CIELAB-Farbraum im Vergleich zu den gemessenen Werten (rot)}\label{b_cqs2b}
\end{figure}

Auf die in den Abbildung \ref{b_cqs2a} und \ref{b_cqs2b} erwähnten Werte $Q_{a}$ (optimierter CQS-Wert für kaum übersättigte Farben), $Q_{p}$ (optimierter CQS-Wert für viele übersättigte Farben) und $Q_{g}$(optimierter CQS-Wert im Zusammenhang mit dem Gamut Area Index) wird in dieser Arbeit nicht weiter eingegangen, weil sie über den Rahmen dieser Bachelorarbeit hinaus gehen\footnote{\cite[60-62]{khanh}}. 

\newpage 
\section{Television Lighting Consistency Index (TLCI)} \label{sec_tlci}
Der CRI-Wert einer Leuchte ist im Fernsehbereich kaum aussagekräftig, weil kein Bezug zur Videokamera besteht und die Farbwiedergabe von menschlichen Hauttöne kaum gemessen wird. Daher hat die European Broadcast Union (EBU) 2012 einen neuen Farbewiedergabe bestimmt, der auf den Film- und Fernsehbereich zugeschnitten ist, den Television Lighting Consistency Index.  
Wie eine Messung des TLCI vonstattengeht ist in diesem Blockschaltbild der EBU verdeutlicht (\ref{b_tlci1}):

\begin{figure}[htp]     % h=here, t=top, b=bottom, p=page
\centering
\includegraphics[width=1.0\textwidth]{bilder/tlci1} 
% Bilddatei aus dem Unterverzeichnis bilder holen, skalieren auf 0.8*Satzspiegel
\caption {Blockschaltbild einer TLCI-Wertbestimmung \protect\footnotemark}\label{b_tlci1}
\end{figure}
\footnotetext{\citep[15]{roberts}}
Die von der Kamera gefilmten Farben werden dann in einem Datenfile gespeichert. Die Daten werden analysiert, um die Farbtemperatur zu bestimmen und so die Referenzdaten zu erstellen.

Zur Ermittlung des TLCI wird eine Testtafel mit 24 Farben von einer \glqq Standartkamera\grqq gefilmt. Diese Tafel wird von der zu testenden Leuchte bestrahlt. Die Kamera ist an einen \glqq Standartbildschirm\grqq angeschlossen, auf dem die TLCI-Merssergebnisse angezeigt werden.
Im ersten Schritt gewichtet die Kamera die reflektierten Farben mit ihren $\bar{r}$-, $\bar{g}$- und $\bar{b}$-Kamerakurven und die Farbtemperatur wird bestimmt. Die so entstandenen $R_{C}$, $G_{C}$ und $B_{C}$-Werte werden dann im zweiten Schritt farblich abgeglichen ($R_{Cb}$, $G_{Cb}$ und $B_{Cb}$) und mit einer linearen Matrix M bewertet, um die Werte des RGB-Signals zu erhalten .
\begin{equation}\label{gl_tlci1}
\begin{bmatrix} R \\ G \\ B \end{bmatrix}= 
\begin{bmatrix} 1,182 & -0,209 & 0,027 \\ 0,107 & 0,890 & 0,003 \\ 0,004 & -0,134 & 1,094 \end{bmatrix}
\begin{bmatrix} R_{Cb} \\ G_{Cb} \\ B_{Cb} \end{bmatrix}
\end{equation}\\
Ein Weißabgleich wird vorgenommen und die RGB-Werte werden in einer zweiten Matrix verrechnet, damit die Sättigungswerte der Farben stimmen (Empfehlung der EBU: 90 \% Sättigung). Im nächsten Schritt werden die $R_{M}$, $G_{M}$ und $B_{M}$-Werte der einzelnen Farben von der Gammakurve der Kamera vorverzerrt.
Beim Bildschirm angekommen werden die R'G'B'-Werte der Farben mit der Gammakurve des Bildschirm wieder entzerrt (Empfehlung der EBU: $\gamma$ = 2,4). Für die 24 Farben werden dann im vorletzten Schritt mit der XYZ()-Matrix  die Farbkoordinaten X,Y und Z für den Bildschirm errechnet. Schließlich wird mit den Referenzenfarbwerten der selben Farbtemperatur die Farbunterschiede ermittelt (Gleichung \ref{gl_tlci1}).
\begin{equation}\label{gl_tlci1}
		\Delta E_{a} ^{*} = \left( {\sum_{i=1}^{18}(\Delta E_{i} ^{*})^{4}}  \right)^{\frac{1}{4}} 
\end{equation}
Das Ergebnis wird als TLCI-Wert ausgegeben. Für optimale Werte wird mit $k = 3,16$ (eine Tageslichtleuchtstoffröhre erreicht dabei den TLCI-Wert 50) und $p = 4$ (für ein balanciertes Verhältnis zwischen hohen und niedrigen Werten) gerechnet\footnote{\cite[16-22]{roberts}} (Gleichung \ref{gl_tlci2}).

\begin{equation}\label{gl_tlci2}
		Q = \frac{100}{1+(\frac{\Delta E^{*}}{k})^{p}}
\end{equation}

Der TLCI lässt wie der CQS keine negativen Ergebnisse zu (Kapitel \ref{sec_cqs}) und die Werte von 0-100 sind für den Coloristen in der Nachbearbeitung des Videomaterials wie folgt zu deuten(Tabelle \ref{t_tlci}):

	\begin{table}[htp] 
		\rowcolors{1}{}{lgray} 
		\centering
		\begin{tabular}{rlcc}  % Spalten nach Ausrichtung: l, c, r, p{breite} 
		\toprule
		\multicolumn{2}{c}{\large\sffamily Abstufungen des TLCI}\\ 							
		\midrule
		$100 \geq  Q_{a} \geq 85$  & Farben korrigierbar bzw. nicht notwendig\\ 
		$85 > Q_{a} \geq 75$ & nach Korrektur noch akzeptabel\\
		$75 > Q_{a} \geq 50$ & Aufbereitung sehr zeitaufwendig\\
		$50 > Q_{a} \geq 25$ & nicht mehr zu retten - verbesserbar\\
		$25 > Q_{a} \geq 0$ & ist und bleibt nicht akzeptierbar\\
		\bottomrule
		\end{tabular}
		\caption{$Q_{a}$ eingeteilt in verschiedene Stufen\protect\footnotemark}	
		\label{t_tlci}
	\end{table}
	\footnotetext{\cite{production partner}}
Anhand der Tabelle ist eine Art Kostenvergleich möglich, in dem die Farbwiedergabequalität einer Leuchte gegen den Nachbearbeitungsaufwand des Coloristen gegengerechnet werden kann. Der TLCI gibt sogar eine Empfehlung ab, an welchen Paramtern der Colorist Verbesserungen vornehmen muss (Abbildung \ref{b_tlci2}).\newpage 

Die Messung des TLCI-Werts ergibt ein Ergebnisprotokoll, bestehend aus drei Abschnitten: eine Farbtafel mit den 24 Farbfeldern, eine Empfehlung für den Coloristen zur nachträglichen Bildbearbeitung und ein Vergleich von Referenz- und Testspektrum (Abbildung \ref{b_tlci2}):\\ 

\begin{figure}[htp]     % h=here, t=top, b=bottom, p=page
\centering
\includegraphics[width=1.0\textwidth]{bilder/tlci2} 
% Bilddatei aus dem Unterverzeichnis bilder holen, skalieren auf 0.8*Satzspiegel
\caption {TLCI-Ergebnisprotokoll eines Arri L7-C LED Fresnelscheinwerfers\protect\footnotemark}\label{b_tlci2}
\end{figure}
\footnotetext{\url{https://tech.ebu.ch/tlci-2012}}
Oben links ist der Name der Leuchte angegeben, die gemessene korrelierte Farbtemperatur (CCT) und die Abweichung vom Plank'schen Kurvenzug (\ref{sec_spektrum}) mit einer Gewichtung von 0.0054 (Empfehlung EBU). Ist der Abweichungswert kleiner als -1 wird die Zahl in magenta dargestellt (magentastichtiges weiß), ist sie größer als +1, in grün (grünstichiges weiß). Im Beispiel ist die Zahl daher schwarz. Eine Zeile darunter steht der gemessene TLCI-Wert. Der Arri L7-C ist mit $Q_{a}=86$ in die beste Farbwiedergabekategorie einzuordnen (Tabelle \ref{t_tlci}).\\
Oben rechts ist eine Tabelle mit Korrekturwerten für den Coloristen angegeben. Für 12 verschiedene Farbtöne wird jeweils ein Verbesserungsvorschlag für die Helligkeit, die Sättigung und die Farbtonabweichung angegeben. Da es nicht möglich ist, die Abweichung der Werte mit exakten Zahlen zu definieren, werden mit \glqq +\grqq, \glqq 0\grqq und \glqq -\grqq die verschiedenen Korrekturrichtungen aufgezeigt. Eine \glqq 0\grqq zeigt an, dass der Fehler zu klein ist, um ihn zu korrigieren. Die Anzahl der \glqq +\grqq und \glqq -\grqq wiederum ist ein Hinweis darauf, wie viel Aufwand der Colorist für die Anpassung benötigt. Der Arri L7-C hat beispielsweise Bedarf es vorallem im Bereich des Cyan, Blau/Magenta, Magenta und Magenta/Rot in der Farbtonabweichung einer Aufbesserung. Auch im Green/Cyan- und Cyan/Blau-Bereich sollte der Farbton angepasst werden. Die restlichen Verbesserungsvorschläge bei Helligkeit und Sättigung sollte der Colorist zügig bewältigen können.\\
Links unten ist eine Farbtafel mit den 24 Farben des TLCI sichtbar. Im großen Farbfeld ist die Farbe zusehen, wie das Licht des Arri L7-C diese Farbe wiedergibt. In der Mitte jeder Farbtafel ist ein kleineres Viereck, in dem die Referenzfarbe gezeigt wird. Je deutlicher also das Referenzviereck in dem Farbfeld zu sehen ist, desto schlechter ist die Farbwiedergabe der Testleuchte. Im Beispiel  ist im roten Farbfeld zu erkennen, dass der Arri L7-C diese Farbe nicht so gut wiedergibt wie andere Farben.\\ 
Rechts unten ist auf dem TLCI-Ergebnisprotokoll das Referenzspektrum von 380nm bis 740nm Wellenlänge abgebildet (schwarz) und dazu wird das geteste Spektrum geplotet (cyan). In dieser Ansicht kann man gut erkennen, inwieweit das Licht des Arri L7-C das Referenzspektrum abdeckt \footnote{\cite[15]{roberts}}.  


\section{IES Method for Evaluating Light Source Color Rendition (TM-30-15)} \label{sec_tm30}

\chapter{Leuchtmittel}

\section{Glühlampe} \label{sec_glühlampe}

\section{Halogenglühlampe} \label{sec_halogenglühlampe}

\section{Entladungslampen} \label{sec_entladungslampe}

\section{LEDs} \label{sec_led}

\chapter{Vormessungen}

\section{Ziel}

\section{Aufbau}

\section{Fazit aus der Vormessung}

\chapter{Hauptmessung}

\section{Messaufbau}

\chapter{Messergebnisse}

\section{Unterkapitel mit Mathematik, Bildern und Querverweisen}

\chapter{Umfrage}

\section{Unterkapitel mit Mathematik, Bildern und Querverweisen}

\chapter{Umfrageergebnisse}

\section{Unterkapitel mit Mathematik, Bildern und Querverweisen}

\chapter{Auswertung aller Ergebnisse}

\section{Unterkapitel mit Mathematik, Bildern und Querverweisen}

\chapter{Fazit}

\section{Unterkapitel mit Mathematik, Bildern und Querverweisen}





%--------------------- VERZEICHNISSE ----------------

\listoffigures % Abbildungsverzeichnis erzeugen
\listoftables % Tabellenverzeichnis erzeugen

%--------------------- LITERATURLISTE ---------------
% Die Einträge sollen alphabetisch sortiert sein.

\begin{thebibliography}{}

% Formatierung für Internetquelle
% Grundregel: Name, Vorname (falls vorhanden), VÖ-Jahr (falls vorhanden), Titel in Anführungszeichen, URL, Datum des letzten Aufrufs
% zur Formatierung der URL unbedingt den url-Befehl benutzen!!!


\bibitem[Commission Internationale de l'Eclairage(2007)]{CIE}
Commission Internationale de l'Eclairage:
\emph{\glqq Technical Report 177:2007 : Color Rendering of White LED Light Sources\grqq}
\url{https://de.scribd.com/document/125319182/CIE-177-2007}, 2007, letzter Zugriff 20.06.2018

\bibitem[Davis \& Ohno(2006)]{davis_ohno}
Davis, Wendy L. \& Ohno, Yoshihiro:
\emph{\glqq Development of a Color Quality Scale\grqq}
\url{http://citeseerx.ist.psu.edu/viewdoc/download?doi=10.1.1.568.8399&rep=rep1&type=pdf}, 08.02.2006, letzter Zugriff 20.06.2018

\bibitem[DocCheck Flexikon(2014)]{doccheck sko}
DocCheck Flexikon:
\emph{\glqq Skotopisches Sehen\grqq}
\url{http://flexikon.doccheck.com/de/Skotopisches_Sehen}, 24.01.2014, letzter Zugriff 18.06.2018

\bibitem[DocCheck Flexikon(2014)]{doccheck pho}
DocCheck Flexikon:
\emph{\glqq Photopisches Sehen\grqq}
\url{http://flexikon.doccheck.com/de/Photopisches_Sehen}, 10.05.2016, letzter Zugriff 18.06.2018

\bibitem[Production Partner(2018)]{production partner}
Production Partner:
\emph{\glqq Farbwiedergabe: TM-30-15, CRI und Co.\grqq}
\url{https://www.production-partner.de/basics/farbwiedergabe-tm-30-15-cri-und-co/}, 22.02.2018, letzter Zugriff 20.06.2018


\bibitem[Gigahertz-Optik(2012)]{Gigahertz}
Gigahertz-Optik:
\emph{\glqq Grundladen der Lichtmesstechnik\grqq}
\url{https://www.gigahertz-optik.de/de-de/grundlagen-lichtmesstechnik/}, letzter Zugriff 20.06.2018


% Formatierung für Aufsatz / Paper: Titel in Anführungszeichen, Zeitschriftentitel kursiv
\bibitem[Dooley \& Streicher(1982)]{dooley_streicher} 
Dooley, Wesley L.  \& Streicher, Ronald D.:
\glqq M--S Stereo: A Powerful Technique for Working in Stereo\grqq, 
\emph{Journ. Audio Engineering Society} vol. 30 (10), 1982

% Formatierung für Fachbuch, Diplomarbeit o.Ä.: Titel kursiv
\bibitem[Roberts(2015)]{roberts}
Roberts, Alan: 
\emph{TELEVISION LIGHTING CONSISTENCY INDEX (TLCI-2012)}, Version 2.015e, 18.04.2015

\bibitem[Hentschel(1993)]{hentschel}
Hentschel, Hans-Jürgen: 
\emph{Licht und Beleuchtung Theorie und Praxis der Lichttechnik}, 4. Aufl., Hüthig 1994

% Formatierung für Fachbuch mit Herausgeber und mehreren Autoren
\bibitem[Spehr(2009)]{spehr}
Spehr, Georg (Hrsg.): 
\emph{Funktionale Klänge}, transcript 2009

\bibitem[Greule(2014)]{greule}
Greule, Roland (Autor):
\emph{Licht und Beleuchtung im Medienbereich}, Hanser 2015 

\bibitem[Khanh \& Bodrogi \& Vinh(2007)]{khanh}
Khanh, Tran Quoc (Autor) \& Bodrogi, Peter (Autor) \& Vinh, Trinh Quang (Autor):
\emph{Color Quality of Semiconductor and Conventional Light Sources}, Wiley-VCH 2017


% Formatierung für ein einzelnes Kapitel eines speziellen Autors aus einem Fachbuch mit mehreren Autoren
\bibitem[Sowodniok(2009)]{sowodniok}
Sowodniok, Ulrike: 
\glqq Funktionaler Stimmklang -- Ein Prozess mit Nachhalligkeit\grqq, 
in: Spehr, Georg (Hrsg.): \emph{Funktionale Klänge}, transcript 2009




% Formatierung für Aufsatz / Paper: Titel in Anführungszeichen, Zeitschriftentitel kursiv
\bibitem[Stephenson(1990)]{stephenson}
Stephenson, Uwe: 
\glqq Comparison of the Mirror Image Source Method and the Sound Particle Simulation Method\grqq, 
\emph{Applied Acoustics} vol. 29, 1990


\end{thebibliography}

%--------------------- EIGENSTÄNDIGKEITSERKLÄRUNG ---------------
\clearpage\thispagestyle{empty}
\eigen  % im header definiert
%--------------------------------------- ENDE ------------------------------------
\end{document}
%%%%%%%%%%%%%%%%%%%%%%%%%%%%%%%%%%%%











